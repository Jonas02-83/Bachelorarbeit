
\chapter{Praxisergebnisse und Vergleich} \index{Praxisergebnisse und Vergleich}

TODO!!!!!!!!

\section{Besprechungsprotokoll} \index{Besprechungsprotokoll} \label{CompBesprechungsprotokoll}

Für die Erstellung des Besprechungsprotokoll wurde ein verschriftliches Gespräch verwendet [siehe Anhang!!!!!!!]. Mit 
diesem Gespräch und dem Prompt:

\begin{prompt}[H]
    \begin{tcolorbox}[colback=gray!20, colframe=gray!20, boxrule=0pt, sharp corners] 
        Erstell mir aus folgendem Gespräch ein erstes Besprechungsprotokoll für ein Projekt. Das Protokoll soll in 
        Form einer Tabelle mit den Spalten "Nummer": was eine eindeutige Nummer zur Identifikation ist, "Art": eine 
        Auswahl ob es eine Information, ein Auftrag, eine Feststellung oder eine Beschluss ist, "Beschreibung": Was 
        den Punkt kurz und präzise zusammenfasst, "Termin": bis wann der Auftrag erledigt sein muss und 
        "Verantwortlich": Welches Teammitglied verantwortlich ist, sein. Dabei müssen Aufträge mit einem 
        Fälligkeitsdatum und einem Verantwortlichen versehen sein. Beschlüsse müssen klar und unmissverständlich 
        formuliert werden. Feststellungen sind Beschlüsse, die keine Abstimmung benötigen und Informationen bieten 
        den Projektmitgliedern wichtige Hinweise: [siehe Besprechungsprotokoll.docx]
        \vfill
    \end{tcolorbox}
    \caption{Infotext LeChat, Quelle: Le Chat}
    \label{InfotextLeChat}
\end{prompt}

wurden die Eingaben getätigt. Wichtig ist dabei die einzelnen Spalten zu beschreiben. Ansonsten wird das Gespräch in dem 
Besprechungsprotokoll eher als Stichpunktliste zusammengefasst. Alle drei Tools sind dabei etwas ungenau mit der 
Beschreibung der Art. Die Anforderungen die der Kunde in dem Gespräch stellt, werden von ChatGPT und Gemini als 
Information festgelegt. Le Chat hingegen legt diese als Feststellung fest.\\
Alle drei legen teilweise für Aufträge kein genaues Datum fest, sondern schreiben hier zum Beispiel "+2 Wochen". Wenn 
man in den Prompt allerdings noch schreibt, dass ein genaues Datum benötigt wird, machen das auch alle drei Tools. Nur 
Le Chat legt dann als Datum

\section{ChatGPT} \index{ChatGPT} \label{ChatGPT}

\section{Gemini} \index{Gemini} \label{Gemini}

\section{Le Chat} \index{Le Chat} \label{Le Chat}

\section{Vergleich} \index{Vergleich} \label{Vergleich}