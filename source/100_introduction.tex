
\chapter{Einleitung} \index{Einleitung}

TODO!!!!!!!!!

\section{Motivation} \index{Motivation} \label{Motivation}

In einer Welt, die von immer komplexeren und dynamischeren Softwareanwendungen geprägt ist, wird die effektive Nutzung von Technologien zunehmend entscheidend für den Erfolg von  
Softwareprojekten. In diesem Zusammenhang gewinnen Werkzeuge des Natural Language Processing eine immer größere Bedeutung. Insbesondere die jüngsten Fortschritte im Bereich der Large  
Language Models (LLM) eröffnen neue Möglichkeiten für die Verbesserung des Software Engineering Prozesses.

Die Motivation hinter dieser Arbeit liegt in der Erkundung des Einsatzes von LLM-Tools im Software Engineering und der Analyse ihrer Auswirkungen auf verschiedene Phasen des  
Entwicklungszyklus. Die Frage, wie diese fortschrittlichen LLM-Tools genutzt werden können, um den Softwareentwicklungsprozess effizienter, präziser und insgesamt erfolgreicher zu  
gestalten, steht im Mittelpunkt dieser Untersuchung. 

TODO!!!!!!!!!!!!!!!!!!!!!!!!!!!!!!!!!!

\section{Zielsetzung der Arbeit} \index{Zielsetzung der Arbeit} \label{Zielsetzung der Arbeit}

\section{Struktur der Arbeit} \index{Struktur der Arbeit} \label{Struktur der Arbeit}


