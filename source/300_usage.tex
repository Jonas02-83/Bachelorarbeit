
\chapter{Anwendung von LLM Tools im Software Engineering} \index{Anwendung von LLM Tools im Software Engineering}

TODO!!!!!!!!!!

\section{Besprechungsprotokoll} \index{Besprechungsprotokoll} \label{Besprechungsprotokoll}

Besprechungsprotokolle sind ein unverzichtbares Element im Software Engineering Prozess, da sie die Kommunikation 
und Dokumentation innerhalb eines Teams strukturieren und nachvollziehbar machen. Diese Protokolle erfassen die 
wesentlichen Punkte, die in Meetings besprochen werden, einschließlich Entscheidungen, Aufgabenverteilungen und 
nächste Schritte. Traditionell werden Besprechungsprotokolle manuell von einem oder mehreren Teammitgliedern erstellt, 
was zeitaufwendig und fehleranfällig sein kann. Der Einsatz von LLM Tools kann diesen Prozess erheblich optimieren.\\

LLM Tools wie ChatGPT, Google Gemini und Le Chat bieten fortschrittliche Möglichkeiten zur Unterstützung bei der Erstellung 
und Bearbeitung von Besprechungsprotokollen. Da die kostenlosen Versionen dieser Tools keine Audio-Dateien transkribieren 
können und lediglich mit Texteingaben arbeiten, entfällt die Möglichkeit einer Echtzeit-Transkription. Man könnte jedoch 
andere kostenfreie Sprach-zu-Text Tools nutzen, um das Meeting in Textform vorliegen zu haben. Auf diese Weise wären LLM 
Tools in der Lage, die wichtigsten Punkte aus dem Text zu extrahieren und strukturiert zusammenzufassen.\\

Ein praktisches Beispiel für den Einsatz von LLM Tools ist die automatische Identifizierung von Aufgaben, die an bestimmte 
Teammitglieder verteilt wurden, und die direkte Einfügung dieser Informationen in das Protokoll. Dies erleichtert die 
Nachverfolgung und stellt sicher, dass keine wichtigen Punkte übersehen werden.\\

Eine weitere Möglichkeit für den Einsatz von LLM Tools ist die Durchsuchung historischer Besprechungsprotokolle, um relevante 
Informationen schnell zu finden. Dies ist besonders in großen Projekten nützlich, in denen viele Meetings stattfinden und eine
Vielzahl von Informationen verwaltet werden muss. Durch die Suchfunktion können Teammitglieder schnell auf vergangene 
Entscheidungen und Diskussionspunkte zugreifen.

\section{Projekthandbuch} \index{Projekthandbuch} \label{Projekthandbuch}

Das Projekthandbuch ist ein zentrales Dokument im Software Engineering Prozess, das alle wesentlichen Informationen und 
Richtlinien eines Projekts zusammenfasst. Es dient als Referenzwerk für das Projektteam und stellt sicher, dass alle 
Beteiligten ein gemeinsames Verständnis der Projektziele, -anforderungen und -prozesse haben. Das Handbuch umfasst 
typischerweise Abschnitte zu den Projektzielen, den beteiligten Personen, den Kommunikationswegen, den zu verwendenden 
Tools und Technologien sowie den Vorgehensweisen und Standards, die im Projekt eingehalten werden sollen.\\

Der Einsatz von LLM Tools kann durch die Fähigkeiten, wie die automatisierte Generierung, Aktualisierung 
und Verwaltung von Projektdokumentationen, an dieser Stelle gut unterstützen.\\

Ein Möglichkeit der Nutzung von LLM Tools ist, das Projekthandbuch kontinuierlich aktuell zu 
halten. Änderungen und Ergänzungen können automatisch erkannt und in das Handbuch eingepflegt werden. Beispielsweise 
können neue Anforderungen oder geänderte Spezifikationen, die in Besprechungen oder anderen Kommunikationskanälen 
besprochen wurden, direkt in das Projekthandbuch übernommen werden. Dies stellt sicher, dass das Handbuch stets den 
aktuellen Projektstand widerspiegelt und als verlässliche Quelle für alle Teammitglieder dient.

\section{Risikoliste} \index{Risikoliste} \label{Risikoliste}

Die Risikoliste ist ein zentrales Element im Software Engineering Prozess, das potenzielle Risiken identifiziert, 
bewertet und Maßnahmen zu deren Minimierung festlegt. Eine sorgfältig erstellte Risikoliste trägt dazu bei, mögliche 
Probleme frühzeitig zu erkennen und geeignete Gegenmaßnahmen zu planen, um den Projekterfolg zu sichern. Traditionell 
wird die Risikoliste manuell von Projektmanagern und Teammitgliedern erstellt, was zeitaufwendig sein kann und oft 
eine systematische und kontinuierliche Überprüfung erfordert.\\

Der Einsatz von LLM Tools im Kontext der Risikoliste kann diesen Prozess erheblich verbessern. LLM Tools wie ChatGPT, 
Google Gemini und Le Chat bieten fortschrittliche Möglichkeiten zur automatisierten Erstellung, Aktualisierung und 
Verwaltung von Risikolisten.\\

Ein bedeutender Vorteil der Nutzung von LLM Tools ist die Fähigkeit, große Mengen an Informationen zu analysieren und 
potenzielle Risiken zu identifizieren. Diese Tools können Daten aus verschiedenen Quellen wie Projektdokumentationen und 
Besprechungsprotokollen durchforsten, um mögliche Risiken zu erkennen und in die Risikoliste aufzunehmen. Dies würde 
nicht nur Zeit sparen, sondern würde auch die Genauigkeit und Vollständigkeit der Risikoliste erhöhen.\\

Darüber hinaus können LLM Tools dabei helfen, Risiken zu bewerten und zu priorisieren. Basierend auf historischen Daten 
und aktuellen Projektinformationen können diese Tools die Wahrscheinlichkeit und den möglichen Einfluss von Risiken 
einschätzen. Dies ermöglicht eine fundierte Entscheidungsfindung und hilft dem Projektteam, sich auf die wichtigsten 
Risiken zu konzentrieren.\\

Ein weiterer Vorteil ist die Fähigkeit der LLM Tools, kontinuierlich neue Informationen zu überwachen und die 
Risikoliste entsprechend zu aktualisieren. Wenn sich im Verlauf des Projekts neue Risiken ergeben oder bestehende 
Risiken ändern, können diese Änderungen automatisch erkannt und in die Risikoliste integriert werden. Dies stellt 
sicher, dass die Risikoliste stets aktuell ist und das Projektteam rechtzeitig auf neue Entwicklungen reagieren kann.\\

LLM Tools können zudem bei der Entwicklung von Maßnahmen zur Risikominimierung unterstützen. Basierend auf den 
identifizierten Risiken und den Erfahrungen aus früheren Projekten können diese Tools Vorschläge für geeignete 
Gegenmaßnahmen machen. Dies erleichtert es dem Projektteam, proaktiv zu handeln und potenzielle Probleme frühzeitig 
anzugehen.

\section{Anforderungsspezifikation} \index{Anforderungsspezifikation} \label{Anforderungsspezifikation}

Die Anforderungsspezifikation ist ein zentrales Dokument im Software Engineering Prozess, das die funktionalen und 
nicht-funktionalen Anforderungen an ein Softwareprojekt definiert. Es bildet die Basis für das Design, die 
Implementierung und das Testen der Software. Eine genaue und umfassende Anforderungsspezifikation ist daher 
unerlässlich, um Missverständnisse zu vermeiden und sicherzustellen, dass das Endprodukt den Erwartungen der 
Stakeholder entspricht. Die Bestandteile der Anforderungsspezifikation wurden bereits im Unterkapitel 
\ref{Anforderungsdefinition} beschrieben und erläutert.\\

Ein bedeutender Vorteil der Nutzung von LLM-Tools liegt in ihrer Fähigkeit, natürliche Sprache zu verstehen und in 
strukturierte Anforderungen zu übersetzen. Dies kann genutzt werden, um beispielsweise das fachliche Datenmodell 
oder die Entitätsdiagramme für die Anwendungsfälle zu erstellen. Dadurch kann man die Zeit sparen, die ansonsten 
für die manuelle Erstellung dieser Diagramme erforderlich wäre, und diese lediglich in Textform dokumentieren. 
Einfachere und weniger komplexe Diagramme könnten sogar ohne ausführliche Beschreibung erstellt werden.\\

Für das Anwendungsfalldiagramm kann das umfassende Wissen der LLM-Tools genutzt werden. Diese Tools können Vorschläge 
für Anwendungsfälle für die einzelnen Rollen im System unterbreiten, was zu einer Zeitersparnis und einer Erhöhung 
der Genauigkeit und Vollständigkeit des Anwendungsfalldiagramms führt. Zudem können die Rollen in Beziehung 
zueinander gesetzt werden.\\

Darüber hinaus können LLM-Tools dabei helfen, Unklarheiten und Widersprüche in den Anforderungen zu erkennen. 
Durch die Analyse der erfassten Anforderungen können diese Tools potenzielle Konflikte identifizieren und auf 
Inkonsistenzen hinweisen. Dies ermöglicht es dem Projektteam, frühzeitig Korrekturen vorzunehmen und die Qualität 
der Anforderungsspezifikation zu verbessern.

\section{Architekturdokument} \index{Architekturdokument} \label{Architekturdokument}

Das Architekturdokument beschreibt die grundlegende Struktur und die Designentscheidungen eines Softwareprojekts. Es 
dient als detaillierte Vorlage für die Entwicklung und stellt sicher, dass alle Teammitglieder ein einheitliches 
Verständnis der Softwarearchitektur haben. Das Dokument umfasst in der Regel Informationen über die Hauptkomponenten, 
deren Interaktionen, die verwendeten Technologien und die Designprinzipien, die im Projekt verfolgt werden. Auch hier 
wurden die Bestandteile bereits im Unterkapitel \ref{System- und Softwareentwurf} aufgezählt und erklärt.\\

Auf Grundlage der Systemübersicht aus dem Unterkapitel \ref{Anforderungsspezifikation} könnte man sich die Technische Infrastruktur
erstellen lassen und die aktuellsten Versionen von der Hardware ausgeben lassen. Ebenfalls könnten die LLM-Tools die
Komponenten erklären und die benötigten Hardware Voraussetzungen definieren.

Das Komponentenmodell könnte man ähnlich wie die Diagramme aus dem Unterkapitel \ref{Anforderungsspezifikation} Beschreiben
und erstellen lassen. Dabei könnte man auch die Erklärung der Komponenten automatisch erstellen lassen.

\section{Testspezifikation} \index{Testspezifikation} \label{Testspezifikation}