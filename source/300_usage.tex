\chapter{Anwendung von LLM Tools im Software Engineering} \index{Anwendung von LLM Tools im Software Engineering}

Dieses Kapitel untersucht ausführlich die potenziellen Anwendungsfälle von LLM Tools im Software Engineering. 
Zunächst wird eine umfassende Übersicht der verschiedenen Dokumente vorgestellt, die im Verlauf des 
Entwicklungsprozesses erstellt werden. Dabei wird jedes Dokument einzeln betrachtet und aufgezeigt, wie 
LLM Tools bei der Erstellung und Verwaltung dieser Dokumente eingesetzt werden können. Es wird detailliert 
beschrieben, in welchen Bereichen und auf welche Weise diese Tools die Entwickler unterstützen, um die 
Effizienz und Qualität des gesamten Entwicklungsprozesses zu verbessern.

\section{Besprechungsprotokoll} \index{Besprechungsprotokoll} \label{Besprechungsprotokoll}

Besprechungsprotokolle sind ein unverzichtbares Element im Software Engineering Prozess, da sie die Kommunikation 
und Dokumentation innerhalb eines Teams strukturieren und nachvollziehbar machen. Diese Protokolle erfassen die 
wesentlichen Punkte, die in Meetings besprochen werden, einschließlich Entscheidungen, Aufgabenverteilungen und 
nächste Schritte. Traditionell werden Besprechungsprotokolle manuell von einem oder mehreren Teammitgliedern erstellt, 
was zeitaufwendig und fehleranfällig sein kann. Der Einsatz von LLM Tools kann diesen Prozess erheblich optimieren.\\

LLM Tools wie ChatGPT, Google Gemini und Le Chat bieten fortschrittliche Möglichkeiten zur Unterstützung bei der Erstellung 
und Bearbeitung von Besprechungsprotokollen. Da die kostenlosen Versionen dieser Tools keine Audio-Dateien transkribieren 
können und lediglich mit Texteingaben arbeiten, entfällt die Möglichkeit einer Echtzeit-Transkription. Man könnte jedoch 
andere kostenfreie Sprach-zu-Text Tools nutzen, um das Meeting in Textform vorliegen zu haben. Auf diese Weise wären die LLM 
Tools in der Lage, die wichtigsten Punkte aus dem Text zu extrahieren und strukturiert zusammenzufassen.\\

Ein praktisches Beispiel für den Einsatz von LLM Tools ist die automatische Identifizierung von Aufgaben, die an bestimmte 
Teammitglieder verteilt wurden, und die direkte Einfügung dieser Informationen in das Protokoll. Dies erleichtert die 
Nachverfolgung und stellt sicher, dass keine wichtigen Punkte übersehen werden.\\

Eine weitere Möglichkeit für den Einsatz von LLM Tools ist die Durchsuchung historischer Besprechungsprotokolle, um relevante 
Informationen schnell zu finden. Dies ist besonders in großen Projekten nützlich, in denen viele Meetings stattfinden und eine
Vielzahl von Informationen verwaltet werden muss. Durch die Suchfunktion können Teammitglieder schnell auf vergangene 
Entscheidungen und Diskussionspunkte zugreifen.

\section{Projekthandbuch} \index{Projekthandbuch} \label{Projekthandbuch}

Das Projekthandbuch ist ein zentrales Dokument im Software Engineering Prozess, das alle wesentlichen Informationen und 
Richtlinien eines Projekts zusammenfasst. Es dient als Referenzwerk für das Projektteam und stellt sicher, dass alle 
Beteiligten ein gemeinsames Verständnis der Projektziele, -anforderungen und -prozesse haben. Das Handbuch umfasst 
typischerweise Abschnitte zu den Projektzielen, den beteiligten Personen, den Kommunikationswegen, den zu verwendenden 
Tools und Technologien sowie den Vorgehensweisen und Standards, die im Projekt eingehalten werden sollen.\\

Der Einsatz von LLM Tools kann durch die Fähigkeiten, wie die automatisierte Generierung, Aktualisierung 
und Verwaltung von Projektdokumentationen, an dieser Stelle gut unterstützen.\\

Ein Möglichkeit der Nutzung von LLM Tools ist, das Projekthandbuch kontinuierlich aktuell zu 
halten. Änderungen und Ergänzungen können automatisch erkannt und in das Handbuch eingepflegt werden. Beispielsweise 
können neue Anforderungen oder geänderte Spezifikationen, die in Besprechungen oder anderen Kommunikationskanälen 
besprochen wurden, direkt in das Projekthandbuch übernommen werden. Dies stellt sicher, dass das Handbuch stets den 
aktuellen Projektstand widerspiegelt und als verlässliche Quelle für alle Teammitglieder dient.

\section{Risikoliste} \index{Risikoliste} \label{Risikoliste}

Die Risikoliste identifiziert und bewertet potenzielle Risiken und legt Maßnahmen zu deren Minimierung fest.
Eine sorgfältig erstellte Risikoliste trägt dazu bei, mögliche Probleme frühzeitig zu erkennen und geeignete 
Gegenmaßnahmen zu planen, um den Projekterfolg zu sichern. Traditionell wird die Risikoliste manuell von 
Projektmanagern und Teammitgliedern erstellt, was zeitaufwendig sein kann und oft eine systematische und 
kontinuierliche Überprüfung erfordert.\\

Durch den Einsatz von LLM Tools im Kontext der Risikoliste könnte dieser Prozess erheblich verbessert werden. LLM Tools wie ChatGPT, 
Google Gemini und Le Chat bieten fortschrittliche Möglichkeiten zur automatisierten Erstellung, Aktualisierung und 
Verwaltung von Risikolisten.\\

Ein bedeutender Vorteil der Nutzung von LLM Tools ist die Fähigkeit, große Mengen an Informationen zu analysieren und 
potenzielle Risiken zu identifizieren. Diese Tools können Daten aus verschiedenen Quellen wie Projektdokumentationen und 
Besprechungsprotokollen durchforsten, um mögliche Risiken zu erkennen und in die Risikoliste aufzunehmen. Dies würde 
nicht nur Zeit sparen, sondern würde auch die Genauigkeit und Vollständigkeit der Risikoliste erhöhen.\\

Darüber hinaus können LLM Tools dabei helfen, Risiken zu bewerten und zu priorisieren. Basierend auf historischen Daten 
und aktuellen Projektinformationen können diese Tools die Wahrscheinlichkeit und den möglichen Einfluss von Risiken 
einschätzen. Dies ermöglicht eine fundierte Entscheidungsfindung und hilft dem Projektteam, sich auf die wichtigsten 
Risiken zu konzentrieren.\\

Ein weiterer Vorteil ist die Fähigkeit der LLM Tools, kontinuierlich neue Informationen zu überwachen und die 
Risikoliste entsprechend zu aktualisieren. Wenn sich im Verlauf des Projekts neue Risiken ergeben oder bestehende 
Risiken ändern, können diese Änderungen automatisch erkannt und in die Risikoliste integriert werden. Dies stellt 
sicher, dass die Risikoliste stets aktuell ist und das Projektteam rechtzeitig auf neue Entwicklungen reagieren kann.\\

LLM Tools können zudem bei der Entwicklung von Maßnahmen zur Risikominimierung unterstützen. Basierend auf den 
identifizierten Risiken und den Erfahrungen aus früheren Projekten können diese Tools Vorschläge für geeignete 
Gegenmaßnahmen machen. Dies erleichtert es dem Projektteam, proaktiv zu handeln und potenzielle Probleme frühzeitig 
anzugehen.

\section{Anforderungsspezifikation} \index{Anforderungsspezifikation} \label{Anforderungsspezifikation}

Die Anforderungsspezifikation definiert die funktionalen und nicht-funktionalen Anforderungen an einem Softwareprojekt.
Es bildet die Basis für das Design, die Implementierung und das Testen der Software. Eine genaue und umfassende 
Anforderungsspezifikation ist daher unerlässlich, um Missverständnisse zu vermeiden und sicherzustellen, dass 
das Endprodukt den Erwartungen der Stakeholder entspricht. Die Bestandteile der Anforderungsspezifikation wurden 
bereits im \autoref{Anforderungsdefinition} beschrieben und erläutert.\\

Ein bedeutender Vorteil der Nutzung von LLM-Tools liegt in ihrer Fähigkeit, natürliche Sprache zu verstehen und in 
strukturierte Anforderungen zu übersetzen. Dies kann genutzt werden, um das fachliche Datenmodell, die 
Entitätsdiagramme für die Anwendungsfälle und die Dialoge zu erstellen. Dadurch kann man die Zeit sparen, die ansonsten 
für die manuelle Erstellung dieser Diagramme und Skizzen erforderlich wäre, und muss diese lediglich in Textform dokumentieren. 
Einfachere und weniger komplexe Diagramme könnten sogar ohne ausführliche Beschreibung erstellt werden.\\

Für das Anwendungsfalldiagramm kann das umfassende Wissen der LLM-Tools genutzt werden. Diese Tools können Vorschläge 
für Anwendungsfälle für die einzelnen Rollen im System unterbreiten, was zu einem Zeitersparnis und einer Erhöhung 
der Genauigkeit und Vollständigkeit des Anwendungsfalldiagramms führt. Zudem können die Rollen in Beziehung 
zueinander gesetzt werden.\\

Darüber hinaus können LLM-Tools dabei helfen, Unklarheiten und Widersprüche in den Anforderungen zu erkennen. 
Durch die Analyse der erfassten Anforderungen können diese Tools potenzielle Konflikte identifizieren und auf 
Inkonsistenzen hinweisen. Dies ermöglicht es dem Projektteam, frühzeitig Korrekturen vorzunehmen und die Qualität 
der Anforderungsspezifikation zu verbessern.

\section{Architekturdokument} \index{Architekturdokument} \label{Architekturdokument}

Das Architekturdokument beschreibt die grundlegende Struktur und die Designentscheidungen eines Softwareprojekts. Es 
dient als detaillierte Vorlage für die Entwicklung und stellt sicher, dass alle Teammitglieder ein einheitliches 
Verständnis der Softwarearchitektur haben. Das Dokument umfasst in der Regel Informationen über die Hauptkomponenten, 
deren Interaktionen, die verwendeten Technologien und die Designprinzipien, die im Projekt verfolgt werden. Auch hier 
wurden die Bestandteile bereits im \autoref{System- und Softwareentwurf} aufgezählt und erklärt.\\

LLM Tools können bei der Erstellung von Diagrammen und anderen visuellen Darstellungen der Softwarearchitektur helfen. 
Durch die automatische Generierung von Architekturdiagrammen auf Basis der erfassten Informationen tragen diese Tools 
dazu bei, die Struktur und Interaktionen der verschiedenen Komponenten klar und übersichtlich darzustellen. Dies erleichtert 
das Verständnis der Architektur und unterstützt die Kommunikation innerhalb des Teams.\\

Ein praktisches Beispiel für den Einsatz von LLM Tools ist die automatische Erstellung von Architekturdokumenten basierend 
auf den initialen Anforderungen und Designentscheidungen. Diese Tools dokumentieren die grundlegenden Komponenten und deren 
Interaktionen sowie die verwendeten Technologien und Designprinzipien. Dadurch wird eine solide Grundlage für die weitere 
Entwicklung geschaffen und sichergestellt, dass alle Teammitglieder ein einheitliches Verständnis der Architektur haben.\\

Basierend auf der Systemübersicht aus dem \autoref{Anforderungsspezifikation} kann die technische Infrastruktur erstellt 
und die aktuellsten Versionen der Hardware spezifiziert werden. Zudem könnten LLM-Tools die Hardwarekomponenten erläutern und die 
erforderlichen Hardwarevoraussetzungen definieren. Das Komponentendiagramm und die Sequenzdiagramme lassen sich ähnlich 
wie die Diagramme aus dem \autoref{Anforderungsspezifikation} beschreiben und erstellen. Beim Komponentendiagramm können 
die Komponenten und Schnittstellen erklärt und definiert werden. Für die Sequenzdiagramme der einzelnen Anwendungsfälle 
könnten die benötigten Funktionen innerhalb der Komponenten inklusive der Übergabeparameter definiert und der Ablauf 
beschrieben werden.\\

\section{Testspezifikation} \index{Testspezifikation} \label{Testspezifikation}

Die Testspezifikation ist ein essenzielles Dokument im Software Engineering Prozess, das die Teststrategie und die spezifischen 
Testfälle für ein Softwareprojekt definiert. Sie stellt sicher, dass alle funktionalen und nicht-funktionalen Anforderungen 
umfassend getestet werden, um die Qualität und Zuverlässigkeit der Software zu gewährleisten. Was die Testspezifikation
umfasst, wurde bereits im \autoref{Implementierung und Modultests} und \autoref{Integration und Systemtests} aufgelistet.\\

Ein wesentlicher Vorteil der Nutzung von LLM Tools ist ihre Fähigkeit, Testfälle automatisch aus den Anforderungsspezifikationen 
zu generieren. Diese Tools können die Anforderungen analysieren und darauf basierend Testfälle erstellen, die sicherstellen, 
dass alle Aspekte der Software umfassend getestet werden. Dies spart nicht nur Zeit, sondern erhöht auch die Vollständigkeit 
und Genauigkeit der Testspezifikation.\\

Darüber hinaus können LLM Tools dabei helfen, Testskripte für automatisierte Tests zu erstellen. Basierend auf den definierten 
Testfällen können diese Tools Testskripte generieren, die in verschiedenen Testumgebungen ausgeführt werden können. Dies würde 
die Automatisierung von Testprozessen erleichtern und zur Effizienzsteigerung im Testzyklus bei tragen.\\

Ein weiterer Vorteil ist die Unterstützung bei der Verwaltung und Aktualisierung der Testspezifikation. LLM Tools können Änderungen 
in den Anforderungen oder im Code automatisch erkennen und die entsprechenden Testfälle und Testskripte aktualisieren. Dies stellt 
sicher, dass die Testspezifikation stets den aktuellen Stand der Softwareentwicklung widerspiegelt und alle neuen oder geänderten 
Funktionen angemessen getestet werden.\\

LLM Tools können auch dabei helfen, die Testabdeckung zu analysieren und zu verbessern. Durch die Überprüfung der vorhandenen Testfälle 
und die Identifizierung von Lücken können diese Tools Vorschläge zur Ergänzung der Testspezifikation machen, um eine umfassende 
Testabdeckung zu gewährleisten. Dies trägt dazu bei, potenzielle Fehler frühzeitig zu erkennen und die Qualität der Software zu erhöhen.\\

Ein praktisches Beispiel für den Einsatz von LLM Tools ist die automatische Erstellung von Testberichten. Nach der Ausführung der Tests 
können diese Tools die Testergebnisse analysieren und detaillierte Testberichte erstellen, die die Testabdeckung, die gefundenen Fehler 
und die insgesamt erreichte Qualität der Software dokumentieren. Dies erleichtert die Nachverfolgung von Fehlern und die Kommunikation 
der Testergebnisse an alle Stakeholder.
