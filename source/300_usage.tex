
\chapter{Anwendung von LLM Tools im Software Engineering} \index{Anwendung von LLM Tools im Software Engineering}

TODO!!!!!!!!!!

\section{Besprechungsprotokoll} \index{Besprechungsprotokoll} \label{Besprechungsprotokoll}

Besprechungsprotokolle sind ein unverzichtbares Element im Software Engineering Prozess, da sie die Kommunikation 
und Dokumentation innerhalb eines Teams strukturieren und nachvollziehbar machen. Diese Protokolle erfassen die 
wesentlichen Punkte, die in Meetings besprochen werden, einschließlich Entscheidungen, Aufgabenverteilungen und 
nächste Schritte. Traditionell werden Besprechungsprotokolle manuell von einem oder mehreren Teammitgliedern erstellt, 
was zeitaufwendig und fehleranfällig sein kann. Der Einsatz von LLM Tools kann diesen Prozess erheblich optimieren. 

LLM Tools wie ChatGPT, Google Gemini und Le Chat bieten fortschrittliche Möglichkeiten zur Unterstützung bei der Erstellung 
und Bearbeitung von Besprechungsprotokollen. Da die kostenlosen Versionen dieser Tools keine Audio-Dateien transkribieren 
können und lediglich mit Texteingaben arbeiten, entfällt die Möglichkeit einer Echtzeit-Transkription. Man könnte jedoch 
andere kostenfreie Sprach-zu-Text Tools nutzen, um das Meeting in Textform vorliegen zu haben. Auf diese Weise wären LLM 
Tools in der Lage, die wichtigsten Punkte aus dem Text zu extrahieren und strukturiert zusammenzufassen. Dies führt zu einer 
erheblichen Reduktion des Zeitaufwands und einer Erhöhung der Genauigkeit der Protokolle.

Ein praktisches Beispiel für den Einsatz von LLM Tools ist die automatische Identifizierung von Aufgaben, die an bestimmte 
Teammitglieder verteilt wurden, und die direkte Einfügung dieser Informationen in das Protokoll. Dies erleichtert die 
Nachverfolgung und stellt sicher, dass keine wichtigen Punkte übersehen werden.

Eine weitere Möglichkeit für den Einsatz von LLM Tools ist die Durchsuchung historischer Besprechungsprotokolle, um relevante 
Informationen schnell zu finden. Dies ist besonders in großen Projekten nützlich, in denen viele Meetings stattfinden und eine
Vielzahl von Informationen verwaltet werden muss. Durch die Suchfunktion können Teammitglieder schnell auf vergangene 
Entscheidungen und Diskussionspunkte zugreifen, was die Effizienz und Transparenz im Projektmanagement erhöht.

\section{Projekthandbuch} \index{Projekthandbuch} \label{Projekthandbuch}

\section{Risikoliste} \index{Risikoliste} \label{Risikoliste}

\section{Anforderungsspezifikation} \index{Anforderungsspezifikation} \label{Anforderungsspezifikation}


\section{Architekturdokument} \index{Architekturdokument} \label{Architekturdokument}

\section{Testspezifikation} \index{Testspezifikation} \label{Testspezifikation}