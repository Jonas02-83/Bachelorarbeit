
\begin{abstract}


    %% Anfang des Abstract-Texts
     
    Im Rahmen dieser Arbeit wird untersucht, inwieweit die drei LLM-Tools ChatGPT, Google Gemini und Mistral Le Chat im Bereich des Software 
    Engineerings eingesetzt werden können. Dabei wird betrachtet, wie geeignet die drei unterschiedlichen LLM Tools sind, die in einem Software-Engineering-Prozess 
    verwendeten Dokumente selbstständig zu erstellen. Die Dokumente werden am Beispiel eines „Mensch ärgere dich nicht“-Spiels erstellt.

    Nach einer kurzen Beschreibung der drei LLM-Tools und einem Überblick über die Inhalte der verschiedenen Dokumente wird untersucht, inwieweit 
    die Tools im Rahmen eines Software-Engineering-Prozesses unterstützen können. Unter zahlreichen Unterstützungsmöglichkeiten, wird sich innerhalb der 
    vorliegenden Arbeit auf eine generische Dokumentenerstellung konzentriert.
    
    In der Analyse werden die Ergebnisse der Tools, die auf dem beiliegenden USB-Stick zu finden sind, besprochen und auf ihre Fehler untersucht. 
    Dabei stellen insbesondere das Format der UML-Diagramme und die Konsistenz zwischen den einzelnen Dokumenten eine Herausforderung dar. 
    Anschließend werden auch die Funktionen für die Schnittstelle mit ChatGPT erstellt, da dieses Tool die Dokumente am erfolgreichsten erstellt hat.
    
    Abschließend gibt es eine Zusammenfassung der Ergebnisse und eine Beschreibung, wie gut mit den einzelnen Tools gearbeitet werden konnte. Danach 
    folgt ein Abschnitt zur Kritik an dieser Arbeit und der Arbeitsweise. Besonders aufgrund der vielen unterschiedlichen Aspekte, die bei der 
    Erstellung der Dokumente berücksichtigt werden müssen, und der unterschiedlichen Ausgaben der drei Tools war es schwierig, den Überblick zu 
    behalten. Zum Schluss folgt ein kurzer Ausblick auf offene Punkte, die in diesem Bereich noch untersucht werden können. Da die drei Tools noch 
    relativ neu sind und die Entwicklung in den letzten Jahren erst richtig Fahrt aufgenommen hat, ist zu erwarten, dass sich in diesem Bereich noch 
    viel tun wird.

    %% Ende des Abstract-Texts
    
    
\end{abstract}
