
\chapter{Grundlagen} \index{Grundlagen}

TODO!!!!!!!!!

\section{LLM Tools} \index{LLM Tools} \label{LLM Tools}

In dieser Arbeit wird größtenteils mit drei unterschiedlichen LLM-Tools gearbeitet: ChatGPT, Google Gemini und  
Le Chat. Im Folgenden wird kurz auf die einzelnen Tools eingegangen.

%https://weissenberg-group.de/was-ist-ein-large-language-model/#:~:text=Zum%20IPA%20Whitepaper-,Wie%20funktioniert%20ein%20LLM%3F,zwischen%20den%20Token%20zu%20ermitteln.

Artikel in eigenen Worten wiedergeben? Selbes bei 'Le Chat'

\subsection{ChatGPT} \index{ChatGPT} \label{ChatGPT}

ChatGPT wurde von dem Unternehmen OpenAI im Jahr 2022 veröffentlicht. Es gibt eine kostenfreie Version und  
eine Version, die ein kostenpflichtiges Abo benötigt. In der Arbeit wird mit der kostenfreien Version gearbeitet,  
um für die Mehrheit relevant zu sein. Die kostenfreie Version ist die GPT-3.5. Diese arbeitet mit Trainingsdaten,  
die bis Januar 2022 gehen. Man arbeitet lediglich mit Texteingaben und -ausgaben. Außerdem werden die Eingaben  
zu Trainingszwecken genutzt und gespeichert. Die kostenpflichtige GPT-4-Version arbeitet hingegen mit Daten  
bis 2023 und kann über Bing auf aktuelle Informationen zugreifen. Außerdem hat sie die Fähigkeit, auch Bilder  
zu verarbeiten und zu generieren. Daneben hat man auch die Möglichkeit, die Verarbeitung der Eingaben auszuschalten. 

\subsection{Google Gemini} \index{Google Gemini} \label{Google Gemini}

Google Gemini (ehemals Bard) wurde von DeepMind entwickelt, was ein Tochterunternehmen von Google ist.  
Das Erstveröffentlichungsdatum war der 6. Dezember 2023 und basiert auf der Meena-Architektur von Google AI,  
die 2020 veröffentlicht wurde.

\subsection{Le Chat} \index{Le Chat} \label{Le Chat}

Le Chat ist ein, von dem französischem KI-Startup Mistral AI veröffentlichter Chatbot. Mistral AI startete mit  
seinen frei verwendbaren Sprachmodellen, also auf der Grundlage von Open Source, erfolgreich durch. Nun(????)  
hat Mistral AI sein bislang größtes Modell 'Mistral Large' veröffentlicht. Diesmal allerdings nicht auf der  
Basis von Open Source, sondern ausschließlich über die eigene Webseite und der KI-Infrastruktur Microsoft  
Azure. Es lassen sich allerdings API-Keys für Programmierschnittstellen erstellen, um z.B. Mistral Large über  
seinen eigenen Server laufen zu lassen und so für andere User auf der eigenen Homepage verfügbar zu machen.  
Mit Mistral Large wurde auch Le Chat veröffentlicht, welcher aktuell(?????????) kostenfrei verwendet werden kann.
Le Chat bietet derzeit noch sehr wenige Funktionen an. Es stehen Lediglich Texteingabe und -ausgabe zur Verfügung.  
Die Datengrundlage reicht nur bis 2021, weshalb es auch hier, für die Jahre 2022 bis heute, zu der Problematik der  
Halluzination kommen kann.
Grundsätzlich kann man zwischen drei Sprachmodellen auswählen: Large, Next und Small. Large bietet überlegene Denkfähigkeit,  
Next ist ein Prototyp-Modell für erhöhte Kürze und Small arbeitet schnell und kosteneffektiv. !!!!!Da Next ein Prototyp ist, könnte  
dieses besser arbeiten als Large, kann aber auch zu vermehrten Fehlern führen. Daher wird in der Arbeit mit dem Large  
Sprachmodell gearbeitet.!!!!! *VERMUTUNG??*


https://www.unidigital.news/mistral-ai-franzoesisches-ki-startup-stellt-chatbot-le-chat-und-kooperation-mit-microsoft-azure-vor/

\section{Software Engineering Prozess}

\subsection{einzelne Phasen} \index{einzelne Phasen} \label{einzelne Phasen}
