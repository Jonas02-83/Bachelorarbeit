
\chapter{Grundlagen} \index{Grundlagen}

TODO!!!!!!!!!

\section{LLM Tools} \index{LLM Tools} \label{LLM Tools}

In dieser Arbeit wird größtenteils mit drei unterschiedlichen LLM-Tools gearbeitet: ChatGPT, Google Gemini und  
Mistral AI. Im Folgenden wird kurz auf die einzelnen Tools eingegangen.

\subsection{ChatGPT} \index{ChatGPT} \label{ChatGPT}

ChatGPT wurde von dem Unternehmen OpenAI im Jahr 2022 veröffentlicht. Es gibt eine kostenfreie Version und  
eine Version, die ein kostenpflichtiges Abo benötigt. In der Arbeit wird mit der kostenfreien Version gearbeitet,  
um für die Mehrheit relevant zu sein. Die kostenfreie Version ist die GPT-3.5. Diese arbeitet mit Trainingsdaten,  
die bis Januar 2022 gehen. Man arbeitet lediglich mit Texteingaben und -ausgaben. Außerdem werden die Eingaben  
zu Trainingszwecken genutzt und gespeichert. Die kostenpflichtige GPT-4-Version arbeitet hingegen mit Daten  
bis 2023 und kann über Bing auf aktuelle Informationen zugreifen. Außerdem hat sie die Fähigkeit, auch Bilder  
zu verarbeiten und zu generieren. Daneben hat man auch die Möglichkeit, die Verarbeitung der Eingaben auszuschalten. 

\subsection{Google Gemini} \index{Google Gemini} \label{Google Gemini}

Google Gemini (ehemals Bard) wurde von DeepMind entwickelt, was ein Tochterunternehmen von Google ist.  
Das Erstveröffentlichungsdatum war der 6. Dezember 2023 und basiert auf der Meena-Architektur von Google AI,  
die 2020 veröffentlicht wurde.

\subsection{Mistral AI} \index{Mistral AI} \label{Mistral AI}

Todo!!!!!!!!!!!!!!!!!!!

\section{Software Engineering Prozess}
