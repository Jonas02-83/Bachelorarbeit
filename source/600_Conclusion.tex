
\chapter{Zusammenfassung und Ausblick} \index{Zusammenfassung und Ausblick}

In diesem Kapitel werden die Ergebnisse der Tools nochmal zusammengefasst. Anschließend folgt ein Ausblick 
und eine grobe Einschätzung, wie sich die LLM-Tools in der Zukunft noch entwickeln.
TODO!!!!!!!!

\section{Zusammenfassung der Ergebnisse} \index{Zusammenfassung der Ergebnisse} \label{Zusammenfassung der Ergebnisse}

Es lässt sich grundsätzlich sagen, dass alle drei Tools es geschafft haben, bei den meisten Dokumenten eine solide 
Basis zu erstellen, die auch richtig ist. Dabei fällt vor allem auf, dass einfachere Sachen wie die Erstellung von 
Texten oder das Designen von Dialogen sehr gut klappt. Schwierige Diagramme, die vielleicht auch aufeinander 
aufbauen klappen dahingegen noch nicht so gut und beinhalten häufig noch einige Fehler. Auch die Konsistenz 
zwischen den einzelnen Dokumenten klappt mal mehr und mal weniger gut.

- alle drei Tools bei meisten Dokumente eine solide Basis erstellt
- Einfache Sachen, wie Erstellung von Texten oder Dialogen sehr gut
- Komplexere Sachen wie Komponentenmodell oder Sequenzdiagramme problematisch
- Konsistenz zwischen den Dokumenten lässt zu wünschen übrig (Besonders bei AF) -> Erinnerung an frühere 
Dokumente lässt irgendwann nach
- Probiert, einfache Eingaben zu machen, die auch Leute ohne Vorwissen erstellen würden
-> Genaue Beschreibung teilweise nötig. Bei Diagrammen hat eigentlich gereicht zu sagen, was man für eins 
möchte
- Frage ist ob Leute ohne Vorwissen es schaffen die Lösung der Tools zu verstehen, zu korrigieren und passende 
Nachfragen zu stellen
-> Man erhält aber Idee, was erstellt werden soll
- Manchmal schwierig, die Tools von einer Implementierung oder Formulierung weg zu bekommen (Bsp. Komponentenmodell)
- Alle drei haben immer unterschiedliche Ausgaben erzeugt, trotz gleicher Eingabe
-> Falls mal nur mist erzeugt wird, neu generieren lassen (Besonders bei Gemini ein problem)
- Erzeugung mit ChatGPT hat deutlich am besten geklappt, liegt vermutlich auch an 4o-Version
- Gemini war sehr problematisch, da kaum Erinnerungsvermögen
-> Musste in einer ``Session'' erstellt werden, da sich Gemini nicht mehr an Ausgaben vom vor Tag erinnert hat
- Gemini auch sehr stark von der Eingabe abhängig, wenn man ihn was gefragt hat, wurde häufig nur geantwortet. Bei den 
anderen Tools wurde die Ausgabe direkt angepasst
- Gemini wechselt teilweise einfach auf Englisch um
- LeChat erstellt immer Link auf [[[[[]]]]] mit, was aber nur ein Platzhalter Link ist
- Desweiteren teilweise komische Warnung vor Eingabe: Bsp. "Der Inhalt kann gefährliche oder sensible Themen beinhalten" für Eingabe:
"Formulier die Beschreibung der Aktivitäten etwas genauer. Beschreibe welcher Wert wo gespeichert wird mithilfe des Fachlichen Datenmodells."
- ChatGPT-4o Problematisch mit begrenzter Eingabe Anzahl 
-> Wenn man Zeit hat kein Problem, wenn nicht schon da auch spürbarer Qualitätsunterschied zwischen 3.5 und 4o vorhanden ist


\section{Implikationen für die Praxis} \index{Implikationen für die Praxis} \label{Implikationen für die Praxis}

Schlecht wenn im Nachhinein noch Grobe Fehler auffallen. (FDM Gemini Chat und Punktesystem)

Schlecht für große Zusammenhängende Projekte, da Erinnerungsvermögen nicht so lange hält. Bei manchen Tools besser als bei anderen 
aber grundsätzlich sich da alle begrenzt.
TODO!!!!!!!!!!

\section{Ausblick und zukünftige Entwicklungen} \index{Ausblick und zukünftige Entwicklungen} \label{Ausblick und zukünftige Entwicklungen}

Ich hätte vor der Untersuchung festlegen sollen, wieviele eingaben gemacht werden soll. So ist es bei jedem Tool etwas unterschiedlich
was dazu führt dass das Endergebnis, bei Tools mit mehr Eingaben, vielleicht etwas besser ist als bei Tools mit weniger eingaben.

Alles Produkte von Unternehmen, welche Gewinnorientiert sind. Daher Diebstahl von Infos und Herangehensweisen der Projekte 
wenn man diese mit dem Tool zusammenfassen lässt. Grade im militärischen Bereich.

Lokale LLM Tools

TODO!!!!!!!!!!