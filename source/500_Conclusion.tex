
\chapter{Zusammenfassung und Ausblick} \index{Zusammenfassung und Ausblick}

Im Verlauf dieser Arbeit wurde deutlich, wie LLM-Tools bei der Erstellung von Dokumenten im Software Engineering 
Prozess unterstützen können. Im abschließenden Kapitel werden die Ergebnisse der Analyse zusammengefasst, um die 
Erkenntnisse und die Effektivität der eingesetzten Tools zu bewerten. Anschließend wird ein Ausblick gegeben, der 
eine grobe Einschätzung zur zukünftigen Entwicklung und dem Potenzial von LLM-Tools im Software Engineering liefert. 
Hierbei wird sowohl auf die erwarteten technologischen Fortschritte als auch auf mögliche neue Anwendungsfelder 
eingegangen.

\section{Zusammenfassung der Ergebnisse} \index{Zusammenfassung der Ergebnisse} \label{Zusammenfassung der Ergebnisse}

Es lässt sich grundsätzlich sagen, dass alle drei Tools es geschafft haben, bei den meisten Dokumenten eine solide 
und korrekte Basis zu erstellen. Dabei fällt vor allem auf, dass einfachere Sachen wie die Erstellung von 
Texten oder das Designen von Dialogen sehr gut klappt. Schwierige Diagramme, die vielleicht auch aufeinander 
aufbauen, klappen hingegen noch nicht so gut und beinhalten häufig noch Fehler. Auch die Konsistenz 
zwischen den einzelnen Dokumenten ist durchwachsen. Dies stellt sich besonders bei den 
Sequenzdiagrammen raus, da diese recht wenig zu dem dazugehörigen Aktivitätsdiagramm und dem Fachlichen 
Datenmodell passen. Daher sind die Tools aktuell noch eher ungeeignet für große zusammenhängende Projekte. Wobei 
man beachten muss, dass ChatGPT und Le Chat da deutlich weniger Probleme hatten als Gemini. 
Zusätzlich ist es manchmal schwierig, die Tools von einer gewissen Formulierung oder 
Implementierung weg zu bekommen, was man besonders bei dem Komponentenmodell und dem zuvor verwendeten Begriff 
``Komponente'' im Rahmen der Technischen Infrastruktur gemerkt hat. Wenn sich die Tools mal mit einer Ausgabe in 
die komplett falsche Richtung bewegen, kann es helfen, sich die erste Ausgabe zu einem Dokument nochmal ausgeben zu 
lassen. Da diese immer etwas verschieden waren, konnte man so die Tools leicht korrigieren. Dies war besonders 
beim Besprechungsprotokoll hilfreich. 

Es wurde probiert, die Anfangseingaben einfach zu halten und so zu formulieren, dass auch Nutzer ohne Vorwissen 
diese stellen würden. Dabei muss beachtet werden, dass wenn man spezielle Anforderungen an die Ausgabe hat, 
diese mit in die Eingabe eingefügt werden müssen. Besonders bei Tabellen, die ein gewissen Format haben sollen, 
muss dieses mit angegeben werden. Auch falls zusätzliche Informationen, die die Tools nicht wissen können,
müssen diese mit in die Eingabe geschrieben werden. Fraglich ist es dann teilweise, ob unwissende Nutzer es 
schaffen, die Ausgaben auf ihre Korrektheit zu prüfen und nachfolgend richtige und passende Eingaben zu tätigen.
Aber auch ohne das erhält man eine grobe Idee was gefordert ist und wie die Inhalte erstellt werden müssen. 

Im Bezug auf die Handhabung, war ChatGPT sehr angenehm und hat wenig Probleme gemacht. Es lässt sich leicht 
eine Antwort neu generieren und auch alte Eingaben konnten nach Belieben nochmal geändert werden. Auch die 
Qualität der Ausgaben war von ChatGPT am besten, was vermutlich auch mit an der Version 4o liegt. Negativ ist 
dabei allerdings, dass man nur eine gewisse Anzahl an Eingaben machen kann. Sobald das Eingabelimit erreicht ist, kann 
man mit der 3.5er Version weiter arbeiten. Nach einem gewissen Zeitintervall lässt sich mit 4o weiterarbeiten. Alternativ 
muss ein Abonnement abgeschlossen werden, um unbegrenzt mit 4o arbeiten zu können. Da die Ausgaben der 3.5er Version allerdings einen spürbaren 
negativen Qualitätsunterschied im Vergleich zu 4o haben, ist der Rückgriff auf 3.5 eher unangenehm. Wenn man Zeit hat ist es besser 
einfach zu warten und dann mit 4o weiter zu arbeiten. Im Verlauf meiner Arbeit hat es sich bewährt, erst mit 
den anderen Tools, vor allem Le Chat, die Eingaben zu machen, da die Eingaben häufig umgeschrieben werden mussten, bis man
das gewünschte Ergebnis erhalten hat. Anschließend kann man die ``fertige'' Eingabe ChatGPT geben. Dadurch verbraucht man 
keine ``Aufladungen'' bei ChatGPT mit Eingaben, die einen zunächst nicht weiter bringen, sondern gibt ChatGPT nur die 
Qualitativen und wichtigen Eingaben.

Die Arbeit mit Gemini war sehr problematisch, da das Erinnerungsvermögen vergleichsweise kurz war. Bereits nach ein oder 
zwei Tagen wusste ChatGPT nicht mehr, was vorher erstellt wurde, obwohl man im selben Chat weiter arbeitet. Teilweise 
wurde sogar direkt nach einer Eingabe das erstellte wieder vergessen. Aus diesem Grund musste die Erstellung mit Gemini 
in einem Zug gemacht werden, was aber in der Praxis im Alltag nicht umsetzbar ist, da man vermutlich über mehrere Wochen 
oder sogar Monaten an so einem Projekt arbeitet. Besonders als Student verläuft so ein Projekt meistens während eines Semesters oder 
Trimesters. Die Alternative wäre, Gemini bei jeder Eingabe den Rahmen des Projektes mit den benötigten Informationen mit zu geben.
Dies ist aber grade zum Ende hin nicht umsetzbar, da die Dokumente von immer mehr anderen Dokumenten abhängig werden.
Des Weiteren ist Gemini sehr stark von den Eingaben abhängig und hat bei Fragen häufig nur eine Antwort mit Erklärung 
geliefert und nicht in Kontext mit den vorherigen Eingaben gesetzt. Man musste extra die Anweisungen geben dies zu tun.
Zusätzlich hat Gemini gerne mal die Ausgaben auf Englisch geschrieben. Außerdem kann man nur die letzte Eingabe nochmal verändern 
und es gab auch keine Funktion, um eine Antwort neu generieren zu lassen. Hier musste dann der Punkt am Ende eines Satzes entfernt werden, 
damit dies nochmal gemacht wurde. Dazu kommt noch, dass Ausgaben irgendwann einfach abbrechen. Diese Aspekte führten dazu, dass die Arbeit mit 
Gemini sehr mühsam und schwierig war. 

Le Chat war eine kleine Mischung aus den ChatGPT und Gemini. Bei der Qualität der Ausgaben hat man häufig gemerkt, dass Le Chat
noch nicht so weit entwickelt ist wie ChatGPT-4o, allerdings ist die Bedienung ähnlich angenehm wie mit ChatGPT und auf jeden Fall 
besser als mit Gemini. Grundsätzlich gab es hier keine besonderen Vorkommnisse bei der Arbeit mit Le Chat. Lediglich wurden bei den 
Ausgaben häufig ein Link mit eingefügt, der zu einem Bild führen soll, allerdings mit ``/XyXyXy'' geendet hat und nur ein Platzhalter war.
Dies war zu Beginn sehr verwirrend, da der Link nicht sichtbar war, sondern nur gesehen werden konnte, wenn man die gesamte Eingabe kopiert.
Des Weiteren ist manchmal die Warnung, ``Der Inhalt kann gefährliche oder sensible Themen beinhalten'', für die Eingabe, 
``Formulier die Beschreibung der Aktivitäten etwas genauer. Beschreibe welcher Wert wo gespeichert wird mithilfe des Fachlichen 
Datenmodells.'' angezeigt worden. 

\section{Kritik an der Arbeit} \index{Kritik an der Arbeit} \label{Kritik an der Arbeit}

Die Kritik an meiner Arbeit dient dazu, die Schwächen meiner Vorgehensweise zu reflektieren und Bereiche für zukünftige 
Verbesserungen zu identifizieren.\\
Dabei gibt es zwei große Bereiche: Die Erstellung der Dokumente bzw. die Arbeit mit den LLM Tools und die Analyse der 
Ergebnisse.

Bei der Arbeit der LLM Tools wäre es sinnvoller gewesen, im Vorhinein festzulegen, wie umfangreich und wie viele Eingaben 
für ein Dokument für ein Tool getätigt werden sollen. Der Gedanke, die selben Eingaben zu benutzen war schon gut, durch 
die unterschiedlichen Ausgaben ist es allerdings nicht möglich, dies für ein Inhalt fortzuführen. Daher hätte man 
entweder festlegen sollen, dass nur die erste Ausgabe analysiert wird oder sich auf eine gewisse Anzahl an Eingaben 
und die Genauigkeit festzulegen. Bei beiden Methoden werden allerdings die Möglichkeiten der Tools eingegrenzt und auch der 
Bezug zur Realität wäre geschmälert worden. Dafür könnten die Ergebnisse und die Tools allerdings deutlich besser verglichen werden.
Hier muss man entscheiden, welcher Aspekt wichtiger ist. Eine weitere Möglichkeit wäre eine Art Mischung zu machen. 
Man könnte sich auf eine grobe Anzahl an Eingaben festlegen, was aber keine feste Grenze ist. Dadurch wären die Ergebnisse 
etwas vergleichbarer und man würde nicht den Aspekt der Verbesserung außer acht lassen. In jedem Fall sollte sich aber darüber 
im Vorhinein Gedanken gemacht und festgelegt werden. 

Bei der Analyse hätte man sich ebenfalls zunächst im Vorhinein überlegen sollen, wie man vorgeht. Durch die unterschiedlichen
Ergebnisse und die vielen Anforderungen, die vom Kunden festgelegten als auch die Spielregeln, gab es viele Dinge zu bedenken.
Dass die Tools dann die Dokumente nicht konsistent erstellt haben, hat die Analyse zusätzlich erschwert. Dadurch gab es Punkte, die 
beim Erstellen der Prompts oder bei der Analyse der Ausgaben nicht bedacht wurden, dann beim Schreiben der Texte allerdings erkannt wurden
und nicht nachträglich geändert werden konnten. Auch wenn in späteren Dokumenten auf einmal Aspekte aufkommen, welche bei den 
vorherigen Dokumenten nicht näher betrachtet wurden, lässt sich dies schwer korrigieren. Dafür hätte es geholfen, sich eine 
Liste mit allen Dingen, die implementiert werden müssen, zu schreiben und in welchen Dokumenten diese vorkommen müssen um dann 
strukturiert sagen zu können, welche Aspekte fehlen, welche inkonsistent zu vorherigen Angaben sind und welche richtig enthalten 
sind. Dadurch hätte man die Analyse strukturierter und vollständiger schreiben können und es müsste weniger nachträglich verbessert werden.

\section{Ausblick und zukünftige Entwicklungen} \index{Ausblick und zukünftige Entwicklungen} \label{Ausblick und zukünftige Entwicklungen}

Zunächst wird ein kleiner Ausblick auf mögliche Fortführungen und Weiterentwicklungen zu diesem Thema gegeben und anschließend 
eine Einschätzung, wo die Reise mit LLM Tools noch hingehen könnte.

Zum einen wurde es nicht geschafft, die Testspezifikation erstellen zu lassen. Dies sollte größtenteils kein Problem sein, da 
in dem Dokument nur textuell beschrieben wird, wie der Testablauf aussieht und einzelne Tests schriftlich erstellt werden. 
In den bisherigen Dokumenten waren solche Beschreibung nie ein Problem, weshalb es nicht zu erwarten ist, dass es hier anders ist.
Lediglich Gemini könnte Probleme bei der Erstellung der Testdaten bekommen, da es sich hierbei häufig um eine sehr große Tabelle handelt.
Bei Gemini könnte es wieder passieren, dass die Generierung der Ausgabe einfach abgebrochen wird.\\
Anschließend könnte man die Implementierung des Spiels als Java-Code fortsetzten.\\
Auch die Abschnitte und Inhalte, die hier bisher nicht betrachtet wurden, da sie keine allzu große Relevanz haben, könnten noch 
erstellt werden.\\
Als zusätzliche Unterstützungsmöglichkeiten könnte man sich die Pflege der Dokumente angucken. Schaffen es die Tools zum Beispiel 
die Risikoliste passend zum aktuellen Abschnitt anzupassen? Oder diese bei einem eingetretenen oder gelösten Risiko anzupassen? Können 
die Tools zum Beispiel die Anwendungsfälle nachträglich überarbeiten, so dass diese zu den Sequenzdiagrammen passen? All dies wären weitere 
Unterstützungsmöglichkeiten, die die Tools bereitstellen könnten.

Die Unterstützungsmöglichkeiten lassen sich grundsätzlich auch in Unternehmen und Behörden einsetzten. Hier sollte man 
allerdings beachten, dass die in dieser Arbeit vorgestellten LLM Tools Produkte von Unternehmen sind. Hier wird immer 
der Gewinn an erster Stelle stehen und daher wird sich vermutlich auch an den Daten, die als Eingaben benutzt werden, 
bedient. Daher sollte man aufpassen, was man als Eingabe gibt. Eine Lösung könnten zukünftig Lokale LLM Tools bieten,
die vollständig auf einem eigenen System laufen. Innerhalb Unternehmen und Behörden, wo meistens auch mehr Rechenleistung 
zur Verfügung steht, können auch größere LLMs als Inselsystem ohne Außenanbindung laufen (z.B. das neue LLM der Bundeswehr). 
Besonders in Behörden, wo mit eingestuften Inhalten gearbeitet wird, wäre das eine gute Möglichkeit, die Vorteile 
von LLM Tools mit einzubinden.\\
Eine weitere Möglichkeit wären speziell trainierte LLM Tools. Es könnten irgendwann LLM Tools zum Kauf zur 
Verfügung stehen, die nur allgemeine Informationen zum Trainieren erhalten haben, damit die Grundfähigkeiten zur 
Verfügung stehen. Anschließend könnte man das LLM Tool auf einen speziellen Anwendungsbereich hin trainieren. In diesem 
Fall zum Beispiel für das Software Engineering. Dadurch könnten sich die Fähigkeiten des LLM Tools in diesem Bereich 
verstärken und weniger Rechenleistung benötigen, als ein Tool, welches alles ein bisschen kann.

Um nochmal zurück auf den Studenten und auf Schüler zu kommen: Es lässt sich mit ziemlicher Sicherheit sagen, 
dass in Zukunft die Bedenken, dass KI für diese nicht förderlich sondern eher schädlich sind, nicht abnehmen 
werden. Wenn allerdings die LLM Tools in der Arbeitswelt eine unabdingbare Rolle spielen, ist es von Vorteil 
wenn Schüler und Studenten sich mit diesen Tools schon im Vorhinein auseinander setzten. Sie lernen mit den Tools 
umzugehen und wie Eingaben gestellt werden müssen, um die gewünschte Antworten/Ergebnisse zu erhalten. Des Weiteren 
wird auch geübt die Ausgaben auf ihre Korrektheit zu analysieren.

Auf jeden fall sind die Large Lange Modells auf dem Weg, einen großen Einfluss auf das zukünftige Leben zu nehmen.