Basics:
    ChatGPT:
    Die kontinuierliche Weiterentwicklung und Verbesserung von ChatGPT zeigt die beeindruckenden Stärken und Fähigkeiten der 
    künstlichen Intelligenz, birgt jedoch auch Herausforderungen und Schwächen. Wenn man ChatGPT nach seinen Stärken und Schwächen 
    fragt, erhält man die Antwort in Abbildung \ref{StäSchwäChatGPT}
    
    \begin{figure} 
        \begin{tcolorbox}[colback=gray!20, colframe=gray!20, boxrule=0pt, sharp corners] 
        Als künstliche Intelligenz habe ich bestimmte Stärken und Schwächen, die durch meine Programmierung und den Umfang meines Trainings bestimmt werden. Hier sind einige davon:\\
    
        \textbf{Stärken}
        \begin{itemize}
            \item \textbf{Breites Wissen}: Ich habe Zugang zu einer großen Menge an Informationen und Wissen, die bis zu meinem letzten Trainingsdatum im Jahr 2023 reichen. Das ermöglicht mir, in vielen Bereichen fundierte Antworten zu geben.
            \item \textbf{Sprachverständnis}: Ich kann komplexe Sprachstrukturen verstehen und verarbeiten, was mir erlaubt, präzise und kohärente Antworten zu formulieren.
            \item \textbf{Multilinguale Fähigkeiten}: Ich kann in vielen verschiedenen Sprachen kommunizieren, was mich für ein globales Publikum nützlich macht.
            \item \textbf{24/7 Verfügbarkeit}: Ich bin jederzeit verfügbar und kann sofort auf Anfragen reagieren, was mich zu einem praktischen Werkzeug für viele Anwendungsfälle macht.
            \item \textbf{Anpassungsfähigkeit}: Ich kann auf verschiedene Arten von Anfragen reagieren, von einfachen Informationsanfragen bis hin zu komplexen Problemlösungen oder kreativen Aufgaben.
        \end{itemize}
    
        \textbf{Schwächen}
        \begin{itemize}
            \item \textbf{Aktualität der Informationen}: Mein Wissen ist auf den Stand von 2023 begrenzt. Neuere Entwicklungen, Ereignisse oder Änderungen nach diesem Datum sind mir unbekannt.
            \item \textbf{Kontextbegrenzung}: Ich verstehe den Kontext von Anfragen nur so gut, wie er mir durch die eingegebenen Informationen vermittelt wird. Missverständnisse können auftreten, wenn der Kontext unklar ist.
            \item \textbf{Fehlende Emotionen und Einfühlungsvermögen}: Obwohl ich menschenähnlich formulieren kann, fehlen mir echte Emotionen und die Fähigkeit zur Empathie. Das kann bei sensiblen Themen als unpersönlich empfunden werden.
            \item \textbf{Keine physischen Fähigkeiten}: Im Gegensatz zu Robotern oder physischen Maschinen kann ich keine physischen Aufgaben ausführen.
            \item \textbf{Abhängigkeit von Benutzereingaben}: Ich bin darauf angewiesen, dass Benutzer klare und präzise Fragen stellen. Vage oder unklare Anfragen können zu weniger hilfreichen Antworten führen.
            \item \textbf{Potentielle Verzerrungen}: Meine Antworten basieren auf den Daten, mit denen ich trainiert wurde. Diese Daten können unvollständig oder voreingenommen sein, was sich in meinen Antworten widerspiegeln kann.
        \end{itemize}
    
        Durch das Bewusstsein über diese Stärken und Schwächen kann ich hoffentlich besser auf deine Bedürfnisse eingehen und realistische Erwartungen setzen.
        \vfill
        \end{tcolorbox}
        \caption{Quelle: ChatGPT}
        \label{StäSchwäChatGPT}
    \end{figure}

Comparison:
    Projekthandbuch:
        Für die Erstellung des Besprechungsprotokoll wurde ein verschriftliches Gespräch verwendet [siehe Anhang!!!!!!!]. Mit 
        diesem Gespräch und dem Prompt:
    
        \begin{prompt}[H]
            \begin{tcolorbox}[colback=gray!20, colframe=gray!20, boxrule=0pt, sharp corners] 
                Erstell mir aus folgendem Gespräch ein erstes Besprechungsprotokoll für ein Projekt. Das Protokoll soll in 
                Form einer Tabelle mit den Spalten "Nummer": was eine eindeutige Nummer zur Identifikation ist, "Art": eine 
                Auswahl ob es eine Information, ein Auftrag, eine Feststellung oder eine Beschluss ist, "Beschreibung": Was 
                den Punkt kurz und präzise zusammenfasst, "Termin": bis wann der Auftrag erledigt sein muss und 
                "Verantwortlich": Welches Teammitglied verantwortlich ist, sein. Dabei müssen Aufträge mit einem 
                Fälligkeitsdatum und einem Verantwortlichen versehen sein. Beschlüsse müssen klar und unmissverständlich 
                formuliert werden. Feststellungen sind Beschlüsse, die keine Abstimmung benötigen und Informationen bieten 
                den Projektmitgliedern wichtige Hinweise: [siehe Besprechungsprotokoll.docx]
                \vfill
            \end{tcolorbox}
            \caption{Infotext LeChat, Quelle: Le Chat}
            \label{InfotextLeChat}
        \end{prompt}
    
        wurden die Eingabe getätigt. Wichtig ist dabei die einzelnen Spalten zu beschreiben. Ansonsten wird das Gespräch in dem 
        Besprechungsprotokoll eher als Stichpunktliste zusammengefasst. Alle drei Tools sind dabei etwas ungenau mit der 
        Beschreibung der Art. Die Anforderungen die der Kunde in dem Gespräch stellt, werden von ChatGPT und Gemini als 
        Information festgelegt. Le Chat hingegen legt diese als Feststellung fest. Die Aufträge werden dahingegen gut 
        erkannt, jedoch legen alle drei Tools für Aufträge teilweise kein genaues Datum fest, sondern schreiben hier zum 
        Beispiel "+2 Wochen". Wenn man in den Prompt allerdings noch schreibt, dass ein genaues Datum benötigt wird, machen 
        das auch alle drei Tools. Nur Le Chat bleibt dann 

Comparison:
    Bewertung:
        Sehr gut:
        - Inhalt ist Fachlich richtig
        - Format ist so wie gewünscht
        -> Brauch nur in Datei eingefügt zu werden
        
        Gut:
        - Inhalt fehlerhaft, lässt sich mit einpaar Eingaben korrigieren
        - Format ist wie gewünscht
        
        oder
        - Inhalt richtig
        - Format ist fehlerhaft, man kann aber benötigten Inhalt einfach raus kopieren und passend einfügen.
        
        Befriedigend:
        - Inhalt fehlerhaft, lässt sich mit einpaar Eingaben korrigieren
        - Format ist fehlerhaft, lässt sich korrigieren oder man kann richtigen Inhalt raus kopieren
        
        Ausreichend:
        - Inhaltlich fehlen Teile, aber mit weiteren genauen Beschreibungen, von dem was man will, wird es verbessert, aber 
        immernoch nicht vollständig
        
        Mangelhaft:
        - Inhaltlich falsch oder nicht verständlich und lässt sich auch nicht korrigieren.
        
        Besprechungsprotokoll:
        - Nur wichtige Punkte (mit gestelltem Gespräch nicht mehr als 20 Punkte, sonst sofort Mangelhaft)
        - Format soll der beschriebenen Tabelle entsprechen
        - Art darf nur die 4 festgelegten und beschrieben sein
        - Termin muss!! ein Datum sein
        - Verantwortliche Person namentlich nennen
        - muss alle Anforderungen enthalten
        
        Projekthandbuch:
        - Texte sollen zum Abschnitt \& Kapitel passen.
        - Die richtigen Informationen sollen im Text enthalten sein (Redaktion und Verteiler)
        
        Risikoliste:
        - Risiken \& Maßnahmen sollen stimmig sein
        - Attribute sollen grob passen (Habe keine echten Erfahrungswerte diese zu beschreiben)
        - Format der Tabelle soll passen
        - Risiko- \& Restmaß richtig berechnet
        - Restklasse richtig zugewiesen
        
        Anforderungsspezifikation:
            Einleitung:
            - Texte sollen zum Abschnitt \&  Kapitel passen
        
            Systemübersicht:
            - Nicht zu detailiert
            - "Gezeichnetes" Diagramm wäre schön (so als Ausgabe oder PlantUML, auch nach Nachfrage sehr gut)
            - Passender Text
        
            Fachliches Datenmodell:
            - Format eines UML-Klassendiagrammes
            - Zeichnung wäre schön, aber ausführliche Beschreibung würde auch reichen
            - Format und Beschreibung für jeden Wert beschreiben
        
            Anwendungsfalldiagramm:
            - Zeichnung wäre schön, beschreibung der Rollen mit dazugehörigen AF würde aber auch reichen
            - Falls besonderheiten, diese Textuell beschreiben
        
            AF:
            - Zeichnung wäre schön, aber ausführliche Beschreibung würde auch reichen
            - Endzustände auf Clientseite wichtig
            - Aktivitätsdiagramm richtiges Format (auch bei Beschreibung)
            - Aktivitäten beschreiben
        
            Benutzungsschnittstelle:
            - Zeichnung wäre schön, aber ausführliche Beschreibung würde auch reichen
            - Beschreibung, Zeichnung sinnvoll und korrekt. Alle Verbindungen eingezeichnet
        
            Dialog:
            - Zu Benutzungsschnittstelle passende Beschreibung, wann sich Dialog öffnet.
            - Dialog aufzeichenen oder gut beschreiben
    
    
    Besprechungsprotokoll:
        Stichpunkte
        - [Prompt] und Gespräch als Eingabe
        - Teilweise sehr unterschiedliche Ausgaben obwohl selbe Prompts. Auch in neuen Chats bei gleichem Tool.
        Einzelne Einträge mal zu einem Eintrag zusammengefasst, mal in mehreren geschrieben. (Anz. der Einträge immer unterschiedlich).
        Beispiel Server-Client, JavaFX und RMI.
        - genaue Beschreibung der Spalten wird benötigt, da sonst eher eine Stichpunkteliste erstellt wird
        - Einzelnen Arten müssen beschrieben werden, da diese sonst sehr ungenau festgelegt werden. Aber auch mit
        Beschreibung der Arten ein Problem. Tools machen das unterschiedlich und auch in neuen Chats wird die Zuweisung immer 
        bisschen unterschiedlich gemacht. Obwohl gleiche Eingabe.
        - Im Prompt muss stehen das "Termin" ein genaues Datum benötigt. Sonst wird z.B. mit "+2 Wochen" gearbeitet. Le Chat 
        macht das dann trotzdem nicht, da muss als neue Eingabe "Der Termin soll ein genaues Datum sein" geschrieben werden.
        Datum dann allerdings in der Vergangenheit (z.B. 31.10.2023)?? Schreibt aber das die Termine beispielhaft gewählt sind
        und sich am aktuellen Datum, 29.03.2023??, gewählt sind.
        - Gemini hat großes Problem nur die wichtigsten Punkte aus dem Gespräch rauszulesen. Häufig Besprechungsprotokolle mit etwa 
        30 Punkten. Dabei erstellt er als neue Art "Frage" und füllt diese z.B. mit "Maximale Spieleranzahl" um anschließend eine 
        Information aufzulisten mit der Beschreibung "Maximale Spieleranzahl: 4". Wenn selben Prompt in neuem Chat immer wieder 
        eingegeben wird, kommt manchmal eine "kurze" Version, welche immernoch länger ist als die Protokolle der anderen beiden Tools.
        Gemini schreibt sehr kleinkariert und mach für jede einzelne Information einen neuen Punkt. Beispiel Server-Client, JavaFX und RMI.
        - Auch zusätzlicher Hinweis im Prompt, nur die wichtigen Punkte in der Tabelle aufzunehmen, ändert bei Gemini nix daran auszuschweifen.
        - Fazit?? Selber drüber lesen und schauen ob das passt. Vorallem auf "Art" gucken und ob der Punkt wirklich wichtig ist. Le Chat
        hält sich eher zu knapp mit den Punkten, daher hier schauen ob alle wichtigen Informationen enthalten sind.

    Projekthandbuch:
