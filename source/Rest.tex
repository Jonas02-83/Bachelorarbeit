Basics:
    ChatGPT:
    Die kontinuierliche Weiterentwicklung und Verbesserung von ChatGPT zeigt die beeindruckenden Stärken und Fähigkeiten der 
    künstlichen Intelligenz, birgt jedoch auch Herausforderungen und Schwächen. Wenn man ChatGPT nach seinen Stärken und Schwächen 
    fragt, erhält man die Antwort in Abbildung \ref{StäSchwäChatGPT}
    
    \begin{figure} 
        \begin{tcolorbox}[colback=gray!20, colframe=gray!20, boxrule=0pt, sharp corners] 
        Als künstliche Intelligenz habe ich bestimmte Stärken und Schwächen, die durch meine Programmierung und den Umfang meines Trainings bestimmt werden. Hier sind einige davon:\\
    
        \textbf{Stärken}
        \begin{itemize}
            \item \textbf{Breites Wissen}: Ich habe Zugang zu einer großen Menge an Informationen und Wissen, die bis zu meinem letzten Trainingsdatum im Jahr 2023 reichen. Das ermöglicht mir, in vielen Bereichen fundierte Antworten zu geben.
            \item \textbf{Sprachverständnis}: Ich kann komplexe Sprachstrukturen verstehen und verarbeiten, was mir erlaubt, präzise und kohärente Antworten zu formulieren.
            \item \textbf{Multilinguale Fähigkeiten}: Ich kann in vielen verschiedenen Sprachen kommunizieren, was mich für ein globales Publikum nützlich macht.
            \item \textbf{24/7 Verfügbarkeit}: Ich bin jederzeit verfügbar und kann sofort auf Anfragen reagieren, was mich zu einem praktischen Werkzeug für viele Anwendungsfälle macht.
            \item \textbf{Anpassungsfähigkeit}: Ich kann auf verschiedene Arten von Anfragen reagieren, von einfachen Informationsanfragen bis hin zu komplexen Problemlösungen oder kreativen Aufgaben.
        \end{itemize}
    
        \textbf{Schwächen}
        \begin{itemize}
            \item \textbf{Aktualität der Informationen}: Mein Wissen ist auf den Stand von 2023 begrenzt. Neuere Entwicklungen, Ereignisse oder Änderungen nach diesem Datum sind mir unbekannt.
            \item \textbf{Kontextbegrenzung}: Ich verstehe den Kontext von Anfragen nur so gut, wie er mir durch die eingegebenen Informationen vermittelt wird. Missverständnisse können auftreten, wenn der Kontext unklar ist.
            \item \textbf{Fehlende Emotionen und Einfühlungsvermögen}: Obwohl ich menschenähnlich formulieren kann, fehlen mir echte Emotionen und die Fähigkeit zur Empathie. Das kann bei sensiblen Themen als unpersönlich empfunden werden.
            \item \textbf{Keine physischen Fähigkeiten}: Im Gegensatz zu Robotern oder physischen Maschinen kann ich keine physischen Aufgaben ausführen.
            \item \textbf{Abhängigkeit von Benutzereingaben}: Ich bin darauf angewiesen, dass Benutzer klare und präzise Fragen stellen. Vage oder unklare Anfragen können zu weniger hilfreichen Antworten führen.
            \item \textbf{Potentielle Verzerrungen}: Meine Antworten basieren auf den Daten, mit denen ich trainiert wurde. Diese Daten können unvollständig oder voreingenommen sein, was sich in meinen Antworten widerspiegeln kann.
        \end{itemize}
    
        Durch das Bewusstsein über diese Stärken und Schwächen kann ich hoffentlich besser auf deine Bedürfnisse eingehen und realistische Erwartungen setzen.
        \vfill
        \end{tcolorbox}
        \caption{Quelle: ChatGPT}
        \label{StäSchwäChatGPT}
    \end{figure}

Comparison:
    Projekthandbuch:
        Für die Erstellung des Besprechungsprotokoll wurde ein verschriftliches Gespräch verwendet [siehe Anhang!!!!!!!]. Mit 
        diesem Gespräch und dem Prompt:
    
        \begin{prompt}[H]
            \begin{tcolorbox}[colback=gray!20, colframe=gray!20, boxrule=0pt, sharp corners] 
                Erstell mir aus folgendem Gespräch ein erstes Besprechungsprotokoll für ein Projekt. Das Protokoll soll in 
                Form einer Tabelle mit den Spalten "Nummer": was eine eindeutige Nummer zur Identifikation ist, "Art": eine 
                Auswahl ob es eine Information, ein Auftrag, eine Feststellung oder eine Beschluss ist, "Beschreibung": Was 
                den Punkt kurz und präzise zusammenfasst, "Termin": bis wann der Auftrag erledigt sein muss und 
                "Verantwortlich": Welches Teammitglied verantwortlich ist, sein. Dabei müssen Aufträge mit einem 
                Fälligkeitsdatum und einem Verantwortlichen versehen sein. Beschlüsse müssen klar und unmissverständlich 
                formuliert werden. Feststellungen sind Beschlüsse, die keine Abstimmung benötigen und Informationen bieten 
                den Projektmitgliedern wichtige Hinweise: [siehe Besprechungsprotokoll.docx]
                \vfill
            \end{tcolorbox}
            \caption{Infotext LeChat, Quelle: Le Chat}
            \label{InfotextLeChat}
        \end{prompt}
    
        wurden die Eingabe getätigt. Wichtig ist dabei die einzelnen Spalten zu beschreiben. Ansonsten wird das Gespräch in dem 
        Besprechungsprotokoll eher als Stichpunktliste zusammengefasst. Alle drei Tools sind dabei etwas ungenau mit der 
        Beschreibung der Art. Die Anforderungen die der Kunde in dem Gespräch stellt, werden von ChatGPT und Gemini als 
        Information festgelegt. Le Chat hingegen legt diese als Feststellung fest. Die Aufträge werden dahingegen gut 
        erkannt, jedoch legen alle drei Tools für Aufträge teilweise kein genaues Datum fest, sondern schreiben hier zum 
        Beispiel "+2 Wochen". Wenn man in den Prompt allerdings noch schreibt, dass ein genaues Datum benötigt wird, machen 
        das auch alle drei Tools. Nur Le Chat bleibt dann 