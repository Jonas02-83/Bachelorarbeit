Basics:
    ChatGPT:
    Die kontinuierliche Weiterentwicklung und Verbesserung von ChatGPT zeigt die beeindruckenden Stärken und Fähigkeiten der 
    künstlichen Intelligenz, birgt jedoch auch Herausforderungen und Schwächen. Wenn man ChatGPT nach seinen Stärken und Schwächen 
    fragt, erhält man die Antwort in Abbildung \ref{StäSchwäChatGPT}
    
    \begin{figure} 
        \begin{tcolorbox}[colback=gray!20, colframe=gray!20, boxrule=0pt, sharp corners] 
        Als künstliche Intelligenz habe ich bestimmte Stärken und Schwächen, die durch meine Programmierung und den Umfang meines Trainings bestimmt werden. Hier sind einige davon:\\
    
        \textbf{Stärken}
        \begin{itemize}
            \item \textbf{Breites Wissen}: Ich habe Zugang zu einer großen Menge an Informationen und Wissen, die bis zu meinem letzten Trainingsdatum im Jahr 2023 reichen. Das ermöglicht mir, in vielen Bereichen fundierte Antworten zu geben.
            \item \textbf{Sprachverständnis}: Ich kann komplexe Sprachstrukturen verstehen und verarbeiten, was mir erlaubt, präzise und kohärente Antworten zu formulieren.
            \item \textbf{Multilinguale Fähigkeiten}: Ich kann in vielen verschiedenen Sprachen kommunizieren, was mich für ein globales Publikum nützlich macht.
            \item \textbf{24/7 Verfügbarkeit}: Ich bin jederzeit verfügbar und kann sofort auf Anfragen reagieren, was mich zu einem praktischen Werkzeug für viele Anwendungsfälle macht.
            \item \textbf{Anpassungsfähigkeit}: Ich kann auf verschiedene Arten von Anfragen reagieren, von einfachen Informationsanfragen bis hin zu komplexen Problemlösungen oder kreativen Aufgaben.
        \end{itemize}
    
        \textbf{Schwächen}
        \begin{itemize}
            \item \textbf{Aktualität der Informationen}: Mein Wissen ist auf den Stand von 2023 begrenzt. Neuere Entwicklungen, Ereignisse oder Änderungen nach diesem Datum sind mir unbekannt.
            \item \textbf{Kontextbegrenzung}: Ich verstehe den Kontext von Anfragen nur so gut, wie er mir durch die eingegebenen Informationen vermittelt wird. Missverständnisse können auftreten, wenn der Kontext unklar ist.
            \item \textbf{Fehlende Emotionen und Einfühlungsvermögen}: Obwohl ich menschenähnlich formulieren kann, fehlen mir echte Emotionen und die Fähigkeit zur Empathie. Das kann bei sensiblen Themen als unpersönlich empfunden werden.
            \item \textbf{Keine physischen Fähigkeiten}: Im Gegensatz zu Robotern oder physischen Maschinen kann ich keine physischen Aufgaben ausführen.
            \item \textbf{Abhängigkeit von Benutzereingaben}: Ich bin darauf angewiesen, dass Benutzer klare und präzise Fragen stellen. Vage oder unklare Anfragen können zu weniger hilfreichen Antworten führen.
            \item \textbf{Potentielle Verzerrungen}: Meine Antworten basieren auf den Daten, mit denen ich trainiert wurde. Diese Daten können unvollständig oder voreingenommen sein, was sich in meinen Antworten widerspiegeln kann.
        \end{itemize}
    
        Durch das Bewusstsein über diese Stärken und Schwächen kann ich hoffentlich besser auf deine Bedürfnisse eingehen und realistische Erwartungen setzen.
        \vfill
        \end{tcolorbox}
        \caption{Quelle: ChatGPT}
        \label{StäSchwäChatGPT}
    \end{figure}

Comparison:
    Projekthandbuch:
        Für die Erstellung des Besprechungsprotokoll wurde ein verschriftliches Gespräch verwendet [siehe Anhang!!!!!!!]. Mit 
        diesem Gespräch und dem Prompt:
    
        \begin{prompt}[H]
            \begin{tcolorbox}[colback=gray!20, colframe=gray!20, boxrule=0pt, sharp corners] 
                Erstell mir aus folgendem Gespräch ein erstes Besprechungsprotokoll für ein Projekt. Das Protokoll soll in 
                Form einer Tabelle mit den Spalten "Nummer": was eine eindeutige Nummer zur Identifikation ist, "Art": eine 
                Auswahl ob es eine Information, ein Auftrag, eine Feststellung oder eine Beschluss ist, "Beschreibung": Was 
                den Punkt kurz und präzise zusammenfasst, "Termin": bis wann der Auftrag erledigt sein muss und 
                "Verantwortlich": Welches Teammitglied verantwortlich ist, sein. Dabei müssen Aufträge mit einem 
                Fälligkeitsdatum und einem Verantwortlichen versehen sein. Beschlüsse müssen klar und unmissverständlich 
                formuliert werden. Feststellungen sind Beschlüsse, die keine Abstimmung benötigen und Informationen bieten 
                den Projektmitgliedern wichtige Hinweise: [siehe Besprechungsprotokoll.docx]
                \vfill
            \end{tcolorbox}
            \caption{Infotext LeChat, Quelle: Le Chat}
            \label{InfotextLeChat}
        \end{prompt}
    
        wurden die Eingabe getätigt. Wichtig ist dabei die einzelnen Spalten zu beschreiben. Ansonsten wird das Gespräch in dem 
        Besprechungsprotokoll eher als Stichpunktliste zusammengefasst. Alle drei Tools sind dabei etwas ungenau mit der 
        Beschreibung der Art. Die Anforderungen die der Kunde in dem Gespräch stellt, werden von ChatGPT und Gemini als 
        Information festgelegt. Le Chat hingegen legt diese als Feststellung fest. Die Aufträge werden dahingegen gut 
        erkannt, jedoch legen alle drei Tools für Aufträge teilweise kein genaues Datum fest, sondern schreiben hier zum 
        Beispiel "+2 Wochen". Wenn man in den Prompt allerdings noch schreibt, dass ein genaues Datum benötigt wird, machen 
        das auch alle drei Tools. Nur Le Chat bleibt dann 

Comparison:
    Bewertung:
        Sehr gut:
        - Inhalt ist Fachlich richtig
        - Format ist so wie gewünscht
        -> Brauch nur in Datei eingefügt zu werden
        
        Gut:
        - Inhalt fehlerhaft, lässt sich mit einpaar Eingaben korrigieren
        - Format ist wie gewünscht
        
        oder
        - Inhalt richtig
        - Format ist fehlerhaft, man kann aber benötigten Inhalt einfach raus kopieren und passend einfügen.
        
        Befriedigend:
        - Inhalt fehlerhaft, lässt sich mit einpaar Eingaben korrigieren
        - Format ist fehlerhaft, lässt sich korrigieren oder man kann richtigen Inhalt raus kopieren
        
        Ausreichend:
        - Inhaltlich fehlen Teile, aber mit weiteren genauen Beschreibungen, von dem was man will, wird es verbessert, aber 
        immernoch nicht vollständig
        
        Mangelhaft:
        - Inhaltlich falsch oder nicht verständlich und lässt sich auch nicht korrigieren.
        
        Besprechungsprotokoll:
        - Nur wichtige Punkte (mit gestelltem Gespräch nicht mehr als 20 Punkte, sonst sofort Mangelhaft)
        - Format soll der beschriebenen Tabelle entsprechen
        - Art darf nur die 4 festgelegten und beschrieben sein
        - Termin muss!! ein Datum sein
        - Verantwortliche Person namentlich nennen
        - muss alle Anforderungen enthalten
        
        Projekthandbuch:
        - Texte sollen zum Abschnitt \& Kapitel passen.
        - Die richtigen Informationen sollen im Text enthalten sein (Redaktion und Verteiler)
        
        Risikoliste:
        - Risiken \& Maßnahmen sollen stimmig sein
        - Attribute sollen grob passen (Habe keine echten Erfahrungswerte diese zu beschreiben)
        - Format der Tabelle soll passen
        - Risiko- \& Restmaß richtig berechnet
        - Restklasse richtig zugewiesen
        
        Anforderungsspezifikation:
            Einleitung:
            - Texte sollen zum Abschnitt \&  Kapitel passen
        
            Systemübersicht:
            - Nicht zu detailiert
            - "Gezeichnetes" Diagramm wäre schön (so als Ausgabe oder PlantUML, auch nach Nachfrage sehr gut)
            - Passender Text
        
            Fachliches Datenmodell:
            - Format eines UML-Klassendiagrammes
            - Zeichnung wäre schön, aber ausführliche Beschreibung würde auch reichen
            - Format und Beschreibung für jeden Wert beschreiben
        
            Anwendungsfalldiagramm:
            - Zeichnung wäre schön, beschreibung der Rollen mit dazugehörigen AF würde aber auch reichen
            - Falls besonderheiten, diese Textuell beschreiben
        
            AF:
            - Zeichnung wäre schön, aber ausführliche Beschreibung würde auch reichen
            - Endzustände auf Clientseite wichtig
            - Aktivitätsdiagramm richtiges Format (auch bei Beschreibung)
            - Aktivitäten beschreiben
        
            Benutzungsschnittstelle:
            - Zeichnung wäre schön, aber ausführliche Beschreibung würde auch reichen
            - Beschreibung, Zeichnung sinnvoll und korrekt. Alle Verbindungen eingezeichnet
        
            Dialog:
            - Zu Benutzungsschnittstelle passende Beschreibung, wann sich Dialog öffnet.
            - Dialog aufzeichenen oder gut beschreiben
    
    
    Besprechungsprotokoll:
        Stichpunkte
        - [Prompt] und Gespräch als Eingabe
        - Teilweise sehr unterschiedliche Ausgaben obwohl selbe Prompts. Auch in neuen Chats bei gleichem Tool.
        Einzelne Einträge mal zu einem Eintrag zusammengefasst, mal in mehreren geschrieben. (Anz. der Einträge immer unterschiedlich).
        Beispiel Server-Client, JavaFX und RMI.
        - genaue Beschreibung der Spalten wird benötigt, da sonst eher eine Stichpunkteliste erstellt wird
        - Einzelnen Arten müssen beschrieben werden, da diese sonst sehr ungenau festgelegt werden. Aber auch mit
        Beschreibung der Arten ein Problem. Tools machen das unterschiedlich und auch in neuen Chats wird die Zuweisung immer 
        bisschen unterschiedlich gemacht. Obwohl gleiche Eingabe.
        - Im Prompt muss stehen das "Termin" ein genaues Datum benötigt. Sonst wird z.B. mit "+2 Wochen" gearbeitet. Le Chat 
        macht das dann trotzdem nicht, da muss als neue Eingabe "Der Termin soll ein genaues Datum sein" geschrieben werden.
        Datum dann allerdings in der Vergangenheit (z.B. 31.10.2023)?? Schreibt aber das die Termine beispielhaft gewählt sind
        und sich am aktuellen Datum, 29.03.2023??, gewählt sind.
        - Gemini hat großes Problem nur die wichtigsten Punkte aus dem Gespräch rauszulesen. Häufig Besprechungsprotokolle mit etwa 
        30 Punkten. Dabei erstellt er als neue Art "Frage" und füllt diese z.B. mit "Maximale Spieleranzahl" um anschließend eine 
        Information aufzulisten mit der Beschreibung "Maximale Spieleranzahl: 4". Wenn selben Prompt in neuem Chat immer wieder 
        eingegeben wird, kommt manchmal eine "kurze" Version, welche immernoch länger ist als die Protokolle der anderen beiden Tools.
        Gemini schreibt sehr kleinkariert und mach für jede einzelne Information einen neuen Punkt. Beispiel Server-Client, JavaFX und RMI.
        - Auch zusätzlicher Hinweis im Prompt, nur die wichtigen Punkte in der Tabelle aufzunehmen, ändert bei Gemini nix daran auszuschweifen.
        - Fazit?? Selber drüber lesen und schauen ob das passt. Vorallem auf "Art" gucken und ob der Punkt wirklich wichtig ist. Le Chat
        hält sich eher zu knapp mit den Punkten, daher hier schauen ob alle wichtigen Informationen enthalten sind.

    Projekthandbuch:
        - Zwei eingaben. Einmal für Einleitung mit dem Zweck des Dokuments, der Redaktion und dem Verteiler. Und einmal für die 
        Projektdefinition mit ausschließlich den Kapiteln Vorgeschichte und Inhaltliche Kurzdarstellung.
        - Muss genau Beschrieben werden was man für Abschnitte haben will, sonst wird ein eigenes Projekthandbuch erstellt, welches 
        nicht ganz den Anforderungen, welche in dieser Arbeit beschrieben wurde, entspricht.
        - Für die Einleitung werden spezifische Informationen benötigt, welche im Prompt angegeben werden müssen. In diesem Fall wurde 
        angegeben, dass die Projektleiterinfür das Projekt verantwortlich ist und als Verteiler ein Discord-Channel verwendet wird.
        - Abschnitte wurden gut erstellt. ChatGPT bennent im Abschnitt für den Verteiler einmal alle Teammitglieder, frage ob unbedingt nötig.
        - Zweite Abschnitt benötigt keine weiteren Informationen, da diese im Gespräch bereits vorgegeben sind. 
        - Abschnitte wurden hier auch gut erstellt. Die benötigten Informationen haben sich die Tools gut aus dem Gespräch raus gezogen.
        - Gemini hat teilweise dreher in den Namen drinne. Einmal hat Gemini geschrieben, dass Frau Schmidt (Teamcheffin) mit dem Herrn Müller 
        (Kunde) aufnahm. -> muss Probegelesen werden.


    Risikoliste:
        - Auch hier zwei einzelne Eingaben. Einmal für die Tabelle mit den Risiken und einmal für die mit den Maßnahmen.
        - Prompt für Risikotabelle sehr umfangreich. Jede Spalte erklärt damit die Tabelle, die erstellt wird, den Vorgaben entspricht. 
        Auch das Risiken jeweils zu den Oberbegriffen Team-, Technische-, Methodische-, Kunden-, Fachliche-, Produkt-, Management- und 
        Planungsrisiken erstellt werden sollen und die Risiken dazu mit einer leer Zeile getrennt werden sollen für eine bessere Übersicht 
        und um die Risiken leichter Zuordnen zu können.
        - Die Tools haben die Zuweisung der Risikoklasse zunächst mit den Werten niedrig, mittel und hoch erstellt. Daher wurde der 
        Prompt um die Anweisung, dass zwischen tolerierbar, unerwünscht, kritisch und katastrophal gewählt werden soll ergänzt.
        Insgesamt war die Zuordnung jedoch ohne richtiges Konzept und schwerwiegende Risiken eine geringe Risikoklasse als eher unwichtigere
        Risiken. Daher wurde die definition, wann welche Risikoklasse zu wählen ist, ebenfalls ergänzt.
        - Damit Schaden und die Risikoklassen richtig gewählt werden und die Tools vergleichbar sind, wurde das Projektvolumen auf 500000€
        festgelegt. Ansonsten wählen die Tools ein eigenes, was zu Ungenauigkeiten führt.
        - Die Erstellung der Risiken war kein Problem, eher bei der Formatierung der Tabelle und der Zuweisung der restlichen Attirbute haben
        Gemini und Le Chat Schwierigkeiten.
        
        - ChatGPT hat bei der ersten eingabe die Risikoklassen falsch zugewiesen. Bei Hinweis darauf wurde dies jedoch richtig korriegiert.
        Autor eher der Verantwortliche für das Risiko. Macht aber irgendwo Sinn, da die "Experten" auf ihrem Gebiet, am ehesten die Risiken
        für ihr Gebiet kennen und erstellen können.
        
        - Gemini legt den Autor, auf den am ehesten Verantwortlichen für das Risiko, fest. Also Teamrisiken werden von Frau Schmidt definiert,
        während zum Beispiel Anforderungsrisiken vom Anforderungsanalysten (Herr Schneider) und Kundenrisiken vom Kunden (Herr Müller) geschrieben
        werden. Außerdem berechnet Gemini das Risikomaß nicht richtig. Das Komma der Werte ist um eine Stelle zu weit links. Es muss alles *10 
        gerechnet werden. Die ID setzt Gemini einfach auf 1, 2, 3, ... was grundsätzlich aber nicht schlimm ist, da kein großer Mehraufwand, dies 
        selbst zu ändern. Gemini kann nicht gesamte Tabelle aufeinmal Generieren. Es wird einfach mitten drinn aufgehört. Bei nachfrage ob er nur 
        noch Produkt-, Management- und Planungsrisiken erstellen kann, macht er dies, fügt aber keinen Schaden und Maß mehr ein. Wenn ich ihm sag, 
        das er den Schaden und Maß noch festlegen soll, sagt er nur, dass er das ohne detaillierte Informationen über das Projekt und die potenziellen
        Risiken nicht kann.
        
        - Le Chat hatte wieder das Problem mit dem Datum, dass dieses auf den 01.04.2023 gesetzt wurde. Manchmal war außerdem das Format der Tabelle 
        nicht richtig und die Risiken wurde manchmal den Oberbegriffen nicht zugeordnet sondern zufällig erstellt. Musste schritt für schritt beschreiben 
        was Le Chat ändern soll. Auch Le Chat bestimmt die Risikoklasse falsch. Auch nach Hinweis, dass diese nicht stimmen sind diese noch falsch 
        und es wird der Schaden bei den Risiken geändert. Auch hier ist der Autor eher der Verantwortliche.
        
        
        - Prompt für Maßnahmen ähnlich, dass jede Spalte der Tabelle einmal beschrieben wird und anschließend gesagt wird, dass die ID des dazugehörigen
        Risikos übernommen werden soll.
        
        - ChatGPT nur ein Fehler bei Restklasse. Sonst passt alles.
        
        - Gemini erstellt für die Begründung, warum ein Risiko als nicht mehr relevant eingestuft wird eine eigene Spalte. Ist aber nicht so schlimm, da 
        zu beginn eh leer. Formatierung schlecht, da bei jeder 2. Zeile alle Daten ab Spalte "Begründung" um eine Stelle nach links verschoben wird. Neu 
        generierung und auch Hinweis darauf ändert nichts. Trigger auch nicht gefüllt. Außerdem ist der Typ immer abwechselnd "Verhindern" und "Lindern".
        Auch hier ist Teilweise das Restmaß falsch berechnet. Hört bei Risiko 4.2 auf mit der generierung der Maßnahmen. Maßnahmen passen aber zu Risiken.
        
        - Le Chat erstellt die Maßnahmen nicht immer zu den Risiken in der Risikotabelle, sonder zu den Oberbegriffen Team-, Technische-, Methodische-, 
        Kunden-, Fachliche-, Produkt-, Management- und Planungsrisiken. Maßnahme 1-4 scheinen sehr allgemein formuliert, Maßnahmen 5-8 würden
        zu den 1. Risiken des jeweiligen Oberbegriffes passen. Auch hier scheinen die Attribute sehr repetitiv. Wahrscheinlichkeit immer bei 5\%,
        die Risikoklasse außer bei der ersten Maßnahme immer Unerwünscht und der Restschaden auch meistens bei den 2 Maßnahmen für ein Risiko gleich.




Fachliches Datenmodell:
- Prompt: Information, das ein UML-Klassendiagramm verwendet werden soll, um entsprechendes Layout zu erhalten. 
Außerdem beschreibung, welche Infos benötigt werden. Inhalt muss nicht beschrieben werden.

- ChatGPT: 
    - Pfeile zeigen in gleiche Richtung -> sagt "stimmt sollten Bidirektional sein,..." und generiert neu. Fehler bleibt aber
    - Spiel-Spieler Beziehung 4..*? Sollen doch 0..4 Teilnehmen oder garkeine. -> Ändert auf 1..4, richtige Bezeichnung? weiß ich auch nicht.
    - spielBrett als String? -> Wird entfernt bei Nachfrage
    - Wofür zugNummer? -> Reihenfolge der Spielzüge innerhalb eines Spiels festzuhalten
    - Beziehungen von Spielfigur im Diagramm nicht richtig dargestellt. -> Bei Hinweis wird nicht korrigiert. ``Kann'' Beziehungen immer mit 1..
    - Feld braucht Attribut ob belegt oder nicht -> Korrigiert
    - Chat nicht richtig im UML dargestellt, Beziehungen falsch -> Bei Hinweis wird nicht korrigiert
    - Spieler in User und Player aufteilen? -> Macht er richtig. Frage nur ob dann pos in Spieler oder figur?
    - status in Spiel zu einem Enum statt String?
    - farbe in Spieler zu Enum statt String?
    - Spielzug: figurID statt spielfigur? Beziehung zu Spieler?
    - Spielfigur: braucht farbe wenn Spieler schon farbe hat?
    - Bei Prompt Nr. 7, bekommt Spieler ein Attribut pos: int?? und Spieler hat keine Farbe mehr? -> Bei Hinweis Farbe wieder eingefügt und Position entfernt
    - Ändert bei Neugenerierung immer wieder ein oder zwei Beziehungen. -> Lassen sich aber Händisch vervollständigen
    - Für Würfel wird zunächst eine eigene Entität erstellt, nach Frage ob nicht auch einfach Attribut in Spiel reicht, wird dies geändert

    - Hat keine Einschränkungen zu den Attributen festgelegt. -> Wird nach Hinweis gemacht
    - Legt nicht alle Beziehungen fest.

    Prompts:
        - Erstell mir nun für die Anforderungsspezifikation das Fachliche Datenmodell. Dieses soll mit Hilfe eines UML Klassendiagrammes mit ergänzenden Beschreibungen bzw. EInschränkungen spezifiziert werden. Das Modell soll alle Entitätstypen mit deren Eigenschaften, Beziehungen und Einschränkungen besitzen.
        - Im UML-Diagramm zeigen immer beide Pfeile in eine richtung, müssen diese nicht bidirektional zeigen?
        - Du hast die Beziehung "Ein Spiel hat mehrere Spieler" auf 4..* festgelegt. Sollte diese nicht 1 - 4 betragen? Es sollen ja maximal 4 Spieler teilnehmen können.
        - Braucht das Feld nicht noch ein Attribut ob es belegt ist oder nicht?
        - Macht es nicht auch sinn Spieler in zwei Entitäten aufzuteilen? Einmal "User" und einmal "Player".  "User" beinhaltet dabei den Namen, die Email, das Passwort und die Punkte. "Spieler" beinhaltet dann die nötigen Attribute zum Spielen wie SpielerID, die Farbe und die position seiner Spielfiguren.
        - Erstell mir zu den Attributen noch Einschränkungen, in dem du zu jedem Attribut eine Erklärung schreibst und ein Format definierst.
        - Benötigt man für jede entität unbedingt eine ID? Außerdem sind einige Beziehungen falsch: Ein User kann ein Spieler sein (0..*), Ein Spiel kann mehrere Spielzüge haben (0..*), ein Feld kann von einer Spielfigur besetzt sein (0..1), Ein Spieler kann an einem Spiel teilnehmen (1) und es fehlen die Beziehungsbeschreibung von Spieler zu Spielfigur, Spieler zu Chat, Spiel zu Chat, Spielfigur zu Spielzug. Dadurch, dass die Spielfigur nur genau einem Spieler zugeordnet ist, braucht die Spielfigur kein Attribut Farbe oder nicht?
        - Wieso hat Spieler keine Farbe mehr
        - Wieso hat Spieler ein Attribut Position
        - Wie setzt du das Spielbrett als String um? Und warum muss dieses gespeichert werden? Kann man das nicht einfach als Graphische Oberfläche implementieren?
        - Muss man den Würfel in das Fachliche Datenmodell übernehmen? 
        - Reicht es nicht den Würfel als Attribut in Spiel aufzunehmen?


- Le Chat:
    - Nur beschreibung
    - Hat auch nur Spieler, "user" und "player" würden sinn machen.
    - Spielfigur an einem oder mehreren Spielen teilnehmen?
    - Spieler-Spielfigur: Ein Spieler hat genau 4 Figuren, nicht weniger und nicht mehr
    - Beziehung zwischen Spiel-Spielfigur? wenn ja, Spieler * 4 Spielfiguren.
    - Punkte werde nicht gespeichert.
    - Player hat position? Player kann an mehreren Spielen teilnehmen? -> Nach Hinweis geändert
    - Player und Spielfigur haben positionen.
    - Spiel-Spielfigur Beziehung sinnvoll?

    - Probleme mit der Beschreibung der Beziehungen zwischen den Klassen

    - Ein paar grundsätzliche Einschränkungen aufgelistet, aber nicht zu jedem Attribut -> Sagt nur ob Ganzzahl oder Text,
    keine Begrenzung in der Länge.

    Prompts:
        - Erstell mir nun für die Anforderungsspezifikation das Fachliche Datenmodell. Dieses soll mit Hilfe eines UML Klassendiagrammes mit ergänzenden Beschreibungen bzw. EInschränkungen spezifiziert werden. Das Modell soll alle Entitätstypen mit deren Eigenschaften, Beziehungen und Einschränkungen besitzen.
        - Macht es nicht auch sinn "Spieler" in zwei Klassen
        aufzuteilen? Einmal "User" und einmal ``Player''.  "User" beinhaltet dabei den Namen, die Email, das Passwort und die Punkte. "Spieler" beinhaltet dann die nötigen Attribute zum Spielen wie SpielerID, die Farbe und die position seiner Spielfiguren.
        - Ein Player hat 4 Figuren. Außerdem hat der Player keine Positionen, diese haben ja die Figuren. Ein Player soll auch nicht an mehreren Spielen teilnehmen, sondern der User kann mehrere Player sein und ein Player spielt ein Spiel. Ein Spiel besteht aus 1 bis 4 Playern.




Anwendungsfalldiagramm:

- Prompt: Reicht zu sagen, dass das Diagramm alle Anwendungsfälle beinhalten soll und den Rollen zugewiesen werden soll.

- ChatGPT:
    - Spiel erstellen: wird zum Gastgeber. Besondere rechte? wenn ja, warum keine neue Rolle. -> Nach Hinweis neue Rolle
    - Als User Chat nutzen? In Anforderungen wurde nur von Chat während des Spiels gesprochen. -> Nach Hinweis wurde entfernt.
    - Was gehört zu Zug machen, wenn Spielfigur bewegen eigener Anwendungsfall ist -> gesamte Ablauf wenn Spieler am Zug ist
    (Würfeln, Auswählen und Bewegen der Spielfigur, Schlagen gegnerischer Figuren und Beenden des Zuges), AF "Spielfigur bewegen" 
    war doppelt. Wurde aber nach Hinweis entfernt. Sehr großer Anwendungsfall, vielleicht lieber in kleinere AF aufteilen??
    - Erstellt Code um in einem PlantUML-Editor oder einem Online Renderer das Diagramm als Bild zu erstellen
    - Erstellt Skizze in PlantUML ganz gut, sieht lediglich aus als ob auch ``Spieler'' von ``User'' erbt.
    -> Grundgerüst wird ganz gut erstellt, muss aber mit ein paar nachfragen in richtige Richtung lenken.
