
\chapter{Einleitung} \index{Einleitung}

Die rasante Entwicklung im Bereich der Künstlichen Intelligenz hat in den letzten Jahren bedeutende Fortschritte ermöglicht, 
insbesondere durch den Einsatz von Large Language Models (LLM). Diese mächtigen Tools, die auf umfangreichen Textkorpora 
trainiert wurden, haben das Potenzial, vielfältige Aufgaben im Software Engineering Prozess zu optimieren. 
In dieser Bachelorarbeit wird untersucht, wie LLM-Tools im Software Engineering eingesetzt werden können, um die Effizienz und 
Qualität der Entwicklungsprozesse zu verbessern.

In diesem ersten Kapitel wird zunächst auf die Motivation hinter dieser Arbeit eingegangen. Anschließend wird die Zielsetzung 
und dann die Struktur der Arbeit beschrieben.

\section{Motivation} \index{Motivation} \label{Motivation}

KIs ersetzen irgendwann die Menschen! Dies ist eine Aussage, die man mittlerweile recht häufig hört. Unter anderem durch Filme 
und Serien ist das Ansehen von KIs nicht immer positiv, und einige Menschen sehen eine große Gefahr darin. Besonders durch die 
rasante Entwicklung in den letzten zwei Jahren haben sogenannte Large Language Models (LLMs) viel Aufmerksamkeit auf sich gezogen 
und die Möglichkeiten, zum Beispiel in der Informationsbeschaffung, stark vergrößert. Gerade in Schulen und Universitäten haben 
solche Tools mittlerweile eine große Relevanz erlangt, was nicht von jedem positiv gesehen wird. Viele Pädagogen behaupten, dass 
LLM-Tools schlecht für die Entwicklung von Schülern und Studenten sind, da man nun Hausaufgaben oder ähnliche Arbeiten einfach an 
LLM-Tools abgeben und in wenigen Sekunden automatisch erstellen lassen kann. Doch wie gut sind eigentlich die Ausgaben von LLM Tools?

In einer vorherigen Projektarbeit zum Thema ``Erzeugung von Algorithmen mit LLM-Tools für Java'' wurde bereits untersucht 
wie gut drei unterschiedliche Tools einfache Algorithmen zum Sortieren von Listen und einen etwas komplexeren Algorithmus 
für Spiele wie Tic-Tac-Toe oder Vier Gewinnt erstellen können. Dies hat, zumindest bei den einfacheren Algorithmen, bereits 
sehr gut funktioniert, solange man bei funktionierendem Code nicht immer weiter nachfragt und etwas verbessern lassen will.
Angestoßen durch ein begleitendes Modul zum Thema 'Secure Software Engineering' entstand das Interesse an der Frage, inwieweit 
LLMs auch bei der Erstellung von Software-Dokumentation unterstützen können. Insbesondere stellt sich dabei die Frage, ob LLMs 
in der Lage sind, eine vollständige und aussagekräftige Dokumentation autonom zu generieren oder ob sie eher als Hilfsmittel zur 
Unterstützung menschlicher Autoren dienen können.

\section{Zielsetzung der Arbeit} \index{Zielsetzung der Arbeit} \label{Zielsetzung der Arbeit}

Das Ziel der Arbeit ist es, herauszufinden, wie gut LLM-Tools geeignet sind, Dokumente aus dem Software-Engineering-Prozess zu 
erstellen oder bei der Erstellung zu unterstützen. Der Fokus liegt dabei vor allem auf der Erstellung der Kernabschnitte der 
einzelnen Dokumente, wie zum Beispiel dem fachliche Datenmodell und den einzelnen UML-Diagrammen. Des Weiteren sollen Studenten 
einen Eindruck erhalten, wie sie die Eingaben formulieren müssen und worauf es dabei ankommt. Zusätzlich soll eine Tendenz 
abgegeben werden, welche Tools für die Erstellung besonders gut geeignet sind und welche Tools weniger geeignet erscheinen. 

\section{Struktur der Arbeit} \index{Struktur der Arbeit} \label{Struktur der Arbeit}

Grundsätzlich befasst sich die Arbeit mit drei ausgewählten LLM-Tools: ChatGPT und Google Gemini als vermutlich die 
bekanntesten Tools sowie Le Chat von Mistral AI als kleine europäische Alternative. Mit diesen Tools wird versucht, 
die Dokumente für ein ``Mensch ärgere dich nicht''-Spiel selbstständig zu erstellen. Dabei wird ausschließlich mit den 
Ausgaben der Tools gearbeitet, ohne sie zu verändern oder zu verbessern. Stattdessen wird mit Hilfe von Eingaben 
versucht, die Tools zu korrigieren. Die zu erstellenden Dokumente und die Anforderungen an das Projekt orientieren 
sich an den Vorgaben des Moduls ``Secure Software Engineering'' an der Universität der Bundeswehr München, in der 
Fakultät ``ETTI6'' am Institut für ``Software Engineering''. Die verantwortliche Professorin ist Prof. Dr. rer. nat. Andrea Baumann, 
die mit ihren Kontaktdaten im Literaturverzeichnis aufgeführt ist.

Die Arbeit ist wie folgt strukturiert: Zunächst wird eine Einführung in die Historie der drei betrachteten Tools gegeben. 
Anschließend erfolgt eine detaillierte Beschreibung der in einem Software-Engineering-Prozess typischerweise erstellten 
Dokumente, einschließlich ihrer Inhalte und ihres Zwecks. Darauf aufbauend wird untersucht, wie die eingesetzten LLMs bei 
der Erstellung dieser Dokumente unterstützen können. Dieses Kapitel formuliert zugleich die 
zentrale Fragestellung der Analyse.\\
Im folgenden Kapitel werden die Ergebnisse der praktischen Anwendung der LLMs präsentiert und vergleichend betrachtet. 
Dabei wird insbesondere auf die Qualität der generierten Inhalte sowie auf mögliche Fehler und Limitationen eingegangen.\\
Die Arbeit schließt mit einer Zusammenfassung der wichtigsten Erkenntnisse ab. Zudem werden kritisch die Schwächen der 
durchgeführten Analyse und der Arbeitsweise reflektiert sowie, in einem Ausblick, mögliche Ansatzpunkte für zukünftige Forschungsarbeiten aufgezeigt.

Die erstellten Ausgaben, welche im Kapitel 4 analysiert werden, befinden sich auf dem beigelegten USB-Stick 
in Form von PDF-Dokumenten [\autoref{BeigelegteDateien}]. In der Überschrift der Dokumente steht jeweils, 
um was es sich dabei handelt und dann in runden Klammern das Tool, welches für die Erstellung verwendet 
wurde, und eine Zahl, die wievielte Eingabe dies zu diesem Thema war. In den einzelnen Abschnitten wird als erstes der 
Prompt niedergeschrieben, der für die Erstellung der Ausgabe genutzt wurde und anschließend folgt die Ausgabe. In der Analyse wird immer auf 
das Kapitel innerhalb eines Dokuments und die Seite verwiesen. Der Verweis bezieht sich dabei immer auf 
den äquivalenten Abschnitt im beiliegenden Dokument. Als Beispiel: Wenn im Abschnitt ``Anforderungsspezifikation'' auf einen Inhalt verwiesen wird, 
ist dieser in dem Dokument ``04\_Anforderungsspezifikation.pdf'' zu finden. Zu beginn jedes Unterkapitels wird aber auch nochmal 
erwähnt, in welchem Dokument sich die Ausgaben befinden.

Im Rahmen der Arbeit wurde eine Tabelle erstellt, die als Übersicht dient. In den einzelnen Kapiteln wird diese Tabelle 
schrittweise aufgedeckt, um einen schnellen Überblick über die Ergebnisse des jeweiligen Kapitels zu liefern.