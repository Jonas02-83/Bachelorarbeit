
\chapter{Einleitung} \index{Einleitung}

TODO!!!!!!!!!

\section{Motivation} \index{Motivation} \label{Motivation}

KIs ersetzten irgendwann die Menschen! Dies ist eine Aussage die man mittlerweile recht häufig hört. Unteranderem durch Filme und Serien ist das ansehen von KIs nicht 
immer positiv und einige Menschen sehen eine große Gefahr darin. Besonders durch die rasante Entwicklung in den letzten zwei Jahren, haben sogenannte Large Language Modells, Abkegürzt LLM, 
viel Aufmerksamkeit auf sich gezogen und die Möglichkeiten, zum Beispiel in der Informationsbeschaffung, stark vergrößert. Grade in Schulen und 
Universitäten haben solche Tools mittlerweile eine große Relevanz erhalten, was nicht von jedem positiv gesehen wird. Viele Pädagogen behaupten, dass LLM Tools 
schlecht für die Entwicklung von Schülern und Studenten sind, da man diese nun Hausaufgaben oder ähnliche Arbeiten einfach an LLM Tools abdrücken und in wenigen Sekunden 
autmatisch erstellen lassen können. Auch in meinem Alltag als Student habe ich mir schon das ein oder andere mal mit LLM Tools Fragen beantworten lassen und zum Beispiel 
Code erklären lassen. Doch wie gut sind eigentlich die Ausgaben von LLM Tools?

Dazu gibt es schon zahlreiche Benchmark Tests oder andere professionell erstellte Analysen in den unterschiedlichen Bereichen, die die Tools an ihre Gerenzen bringen.
Doch diese betrachten eher sehr spezifische Bereiche und absolut optimierte Eingaben. Mich interessiert es eher, wie es mit einfachen Anliegen und Eingaben aus
sieht, die einen Schüler oder Studenten beschäftigen und erstellen? Wie gut können die Tools diesen Personengruppen helfen?

Dazu habe ich bereits in meinem letzten Trimester eine Projektarbeit zum Thema "Erzeugung von Algorithmen mit LLM Tools für Java" ausgearbeitet. Dort habe ich mir 
angeguckt, wie gut drei unterschiedliche Tools, einfache Algorithmen zum Sortieren von Listen und einen etwas komplexeren Algorithmus für Spiele wie TicTacToe oder
4 Gewinnt, erstellen können. Dies hatte, zumindest bei den einfacheren Algorithmen, bereits sehr gut funktioniert, solange man bei richtigem Code nicht immer weiter nachfragt 
und etwas verbessern lassen will. Durch ein Modul welches ich im Trimester vor der Projektarbeit belegt habe, Secure Software Engineering, habe ich gefallen an der Entwicklung 
von Projekten gefunden und bin so auf die Frage gekommen, wie gut LLM Tools selbstständig die Dokumenten aus einem Software Engineering Prozess erstellen können oder zumindest 
Unterstützen können? So bin ich zu meinem Thema für diese Arbeit gekommen: "Einsatz von LLM Tools im Software Engineering Prozess".

\section{Zielsetzung der Arbeit} \index{Zielsetzung der Arbeit} \label{Zielsetzung der Arbeit}

Das Ziel der Arbeit ist es, herauszufinden wie gut LLM Tools geeignet sind Dokumente aus dem Software Engineering Prozess zu erstellen oder bei der Erstellung zu 
Unterstützen. Desweiteren sollen Studenten einen eindruck erhalten, wie Sie die Eingaben formulieren müssen und warauf es dabei ankommt. Der Fokus liegt dabei 
vorallem auf der Erstellung von den Kernabschnitten der einzelnen Dokumente. Also zum Beispiel das Fachliche Datenmodell und die einzelnen UML Diagramme.
Zusätzlich soll eine Tendenz abgegeben werden, welche Tools für die Erstellung vielleicht besonders gut geeignet sind und welche Tools nicht so gut zum Arbeiten sind.
Dies ist, zumindes teilweise, eine Subjektive Meinung und ich kann hier nur eine Empfehlung anhand meiner Erfahrungen abgeben. Es kann natürlich immer sein, dass
ich mit einem Tool zurecht komme und jemand anderes damit Probleme hat und anders rum.


\section{Struktur der Arbeit} \index{Struktur der Arbeit} \label{Struktur der Arbeit}

Grundsätzlich befasst sich die Arbeit mit drei ausgewählten LLM Tools: ChatGPT und Google Gemini als vermutlich die bekanntesten Tools und als kleine europäische 
Alternative Le Chat von Mistral AI. Mit diesen Tools wird versucht die Dokumente für ein "Mensch ärgere dich nicht"-Spiel selbstständig zu erstellen. Dabei wird 
lediglich mit den Ausgaben gearbeitet. Ich werde daran nichts ändern oder verbessern sondern nur mithilfe von Eingaben die Tools zu korrigieren. Die zu erstellenden 
Dokumente und die Vorraussetzungen an das Projekt beziehen sich auf die Vorgaben aus dem Modul "Secure Software Engineering" an der Universität der Bundeswehr München 
an der Faklutät "ETTI6" am Institut "Software Engineering". Bei der Verantwortlichen(?) Professorin handelt es sich um Prof. Dr. rer. nat. Andrea Baumann. Diese ist 
im Literaturverzeichnis mit den Kontaktdaten nochmal aufgelistet.\\

Der Aufbau der Arbeit sieht dabei wie folgt aus: Zu beginn kommt eine Einführung in die Historie der drei Tools und anschließend eine Beschreibung, welche 
Dokumente in einem Software Engineering Prozess erstellt werden müssen und welche Inhalte diese beinhalten. Danach folgt eine Beschreibung, wie die LLM Tools 
bei der erstellung unterstützen können. Dieses Kapitel bildet zeitgleich die "Aufgabenstellung" für die Analyse dar. Die Analyse folgt dann anschließend im 
Kapitel "??????" wo auf die Ausgaben und die Fehler darin eingegangen wird. Zu guter letzt folgt ein abschließend Kapitel mit einer Zusammenfassung der Ergebnisse,
einem Abschnitt wo ich die Kritik an der Arbeit und meiner Herangehensweise übe und was man besser machen kann und zuletzt einen Ausblick auf offene Punkte gebe.
