
\chapter{Praxisergebnisse und Vergleich} \index{Praxisergebnisse und Vergleich}

TODO!!!!!!!!

Generell:
    - Schwierigkeiten in einem gorßen/langen Chat, die Tools von einer Formulierung weg zu bringen. 


\section{Bewertung} \index{Bewertung} \label{BewertungLLMTools}

Um die Ausgaben der Tools besser zu vergleichen und ein nachvollziehbares Fazit zu ziehen, wurde ein Bewertungssystem 
entwickelt, das fünf Stufen umfasst. Diese Bewertungsstufen ermöglichen eine strukturierte und transparente Methode 
zur Beurteilung der Qualität der Ausgaben von LLM Tools. Die fünf Bewertungsstufen sind wie folgt definiert:\\

``Sehr gut'' wird vergeben, falls der Inhalt der Ausgabe Fachlich richtig ist und auch das Format den Vorgaben entspricht.
Also, dass die Ausgabe genauso in das entsprechende Dokument eingefügt werden kann.\\
Die Ausgabe wird als ``Gut'' eingestuft, wenn der Inhalt etwas fehlerhaft ist, man aber mit einpaar wenigen Eingaben 
den Inhalt korrigieren kann und das Format den Vorgaben entspricht. Eine Ausgabe wird auch als ``Gut'' bewertet, wenn 
der Inhalt richtig ist, aber das Format etwas fehlerhaft ist. Mann sollte aber den richtigen Inhalt einfach aus der 
Ausgabe kopieren können und in das entsprechende Dokument mit dem richtigen Format einfügen können.\\
Als ``Befriedigend'' wird eine Ausgabe bewertet, wenn der Inhalt und das Format etwas fehlerhaft sind, sich diese aber 
mit einpaar wenigen Eingaben korrigieren lassen oder man den richtigen Inhalt aus der Ausgabe kopieren kann.\\
``Ausreichend'' wird vergeben, wenn in der Ausgabe Inhaltlich Teile fehlen, man aber mit ganz genauen Eingaben, von dem 
was man will, die Ausgabe etwas verbessert, aber diese immer noch nicht vollständig ist.\\
Zuletzt wird eine Ausgabe als ``Mangelhaft'' bewertet, wenn der Inhalt der Ausgabe falsch oder nicht verständlich ist
und es sich auch nicht korrigieren lässt.\\



% Sehr gut:
% - Inhalt ist Fachlich richtig
% - Format ist so wie gewünscht
% -> Brauch nur in Datei eingefügt zu werden

% Gut:
% - Inhalt fehlerhaft, lässt sich mit einpaar Eingaben korrigieren
% - Format ist wie gewünscht

% oder
% - Inhalt richtig
% - Format ist fehlerhaft, man kann aber benötigten Inhalt einfach raus kopieren und passend einfügen.

% Befriedigend:
% - Inhalt fehlerhaft, lässt sich mit einpaar Eingaben korrigieren
% - Format ist fehlerhaft, lässt sich korrigieren oder man kann richtigen Inhalt raus kopieren

% Ausreichend:
% - Inhaltlich fehlen Teile, aber mit weiteren genauen Beschreibungen, von dem was man will, wird es verbessert, aber 
% immernoch nicht vollständig

% Mangelhaft:
% - Inhaltlich falsch oder nicht verständlich und lässt sich auch nicht korrigieren.

Außerdem fließen auch spezifische Aspekte, die nur für das jeweilige Aspekt wichtig sind, in die Bewertung mit ein. 
Diese sehen wie folgt aus:

Besprechungsprotokoll:
- Nur wichtige Punkte (mit gestelltem Gespräch nicht mehr als 20 Punkte, sonst sofort Mangelhaft)
- Format soll der beschriebenen Tabelle entsprechen
- Art darf nur die 4 festgelegten und beschrieben sein
- Termin muss!! ein Datum sein
- Verantwortliche Person namentlich nennen
- muss alle Anforderungen enthalten

Projekthandbuch:
- Texte sollen zum Abschnitt \& Kapitel passen.
- Die richtigen Informationen sollen im Text enthalten sein (Redaktion und Verteiler)

Risikoliste:
- Risiken \& Maßnahmen sollen stimmig sein
- Attribute sollen grob passen (Habe keine echten Erfahrungswerte diese zu beschreiben)
- Format der Tabelle soll passen
- Risiko- \& Restmaß richtig berechnet
- Restklasse richtig zugewiesen

Anforderungsspezifikation:
    Einleitung:
    - Texte sollen zum Abschnitt \&  Kapitel passen

    Systemübersicht:
    - Nicht zu detailiert
    - "Gezeichnetes" Diagramm wäre schön (so als Ausgabe oder PlantUML, auch nach Nachfrage sehr gut)
    - Passender Text

    Fachliches Datenmodell:
    - Format eines UML-Klassendiagrammes
    - Zeichnung wäre schön, aber ausführliche Beschreibung würde auch reichen
    - Format und Beschreibung für jeden Wert beschreiben

    Anwendungsfalldiagramm:
    - Zeichnung wäre schön, beschreibung der Rollen mit dazugehörigen AF würde aber auch reichen
    - Falls besonderheiten, diese Textuell beschreiben

    AF:
    - Zeichnung wäre schön, aber ausführliche Beschreibung würde auch reichen
    - Endzustände auf Clientseite wichtig
    - Aktivitätsdiagramm richtiges Format (auch bei Beschreibung)
    - Aktivitäten beschreiben

    Benutzungsschnittstelle:
    - Zeichnung wäre schön, aber ausführliche Beschreibung würde auch reichen
    - Beschreibung, Zeichnung sinnvoll und korrekt. Alle Verbindungen eingezeichnet

    Dialog:
    - Zu Benutzungsschnittstelle passende Beschreibung, wann sich Dialog öffnet.
    - Dialog aufzeichenen oder gut beschreiben


\section{Besprechungsprotokoll} \index{Besprechungsprotokoll} \label{CompBesprechungsprotokoll}

Für die Erstellung des Besprechungsprotokoll wurde ein verschriftliches Gespräch verwendet [\autoref{Kundengespräch1} 
und \autoref{Kundengespräch2}]. Mit diesem Gespräch und dem Prompt:
    
\begin{prompt}[H]
    \begin{tcolorbox}[colback=gray!20, colframe=gray!20, boxrule=0pt, sharp corners] 
        Erstell mir aus folgendem Gespräch ein erstes Besprechungsprotokoll für ein Projekt. Das Protokoll soll nur 
        die wichtigen Punkte in Form einer Tabelle mit den Spalten "Nummer": was eine eindeutige Nummer zur 
        Identifikation ist, "Art": eine Auswahl ob es eine Information, ein Auftrag, eine Feststellung oder eine 
        Beschluss ist, "Beschreibung": Was den Punkt kurz und präzise zusammenfasst, "Termin": ein genaues Datum bis 
        wann der Auftrag erledigt sein muss und "Verantwortlich": Welches Teammitglied verantwortlich ist, enthalten. 
        Dabei müssen Aufträge mit einem genauen Fälligkeitsdatum und einem Verantwortlichen versehen sein. Beschlüsse 
        müssen klar und unmissverständlich formuliert werden. Feststellungen sind Beschlüsse, die keine Abstimmung 
        benötigen und Informationen bieten den Projektmitgliedern wichtige Hinweise: 
        [siehe Besprechungsprotokoll.docx]
        \vfill
    \end{tcolorbox}
    \caption{Prompt Besprechungsprotokoll}
    \label{Prompt Besprechungsprotokoll}
\end{prompt}

wurden dann bei ChatGPT, Gemini und Le Chat die Eingabe getätigt. Die Ergebnisse [[[[[[[[[[[[[[[[]]]]]]]]]]]]]]]].\\

Zunächst fiel auf, dass trotz der Verwendung desselben Prompts die Ausgaben der verschiedenen Tools teilweise erheblich abwichen. 
Dies betraf sowohl die Anzahl der erstellten Einträge im Besprechungsprotokoll als auch die Art und Weise, wie die Informationen 
zusammengefasst wurden. In einigen Fällen wurden mehrere Punkte zu einem Eintrag zusammengefasst, während in anderen Fällen mehrere 
Einträge daraus entstanden. Dies trat insbesondere bei der Anforderung auf, dass die Anwendung mit JavaFX als Server-Client-Architektur 
mit RMI entworfen werden sollte. Hier haben die Tools teilweise einen Eintrag dafür erstellt und manchmal drei einzelne Einträge. 
Dieses Verhalten zeigte sich auch, wenn derselbe Prompt in neuen Chats mit demselben Tool verwendet wurde. Besonders bei Gemini trat 
dieses Problem sehr häufig und extrem auf.\\

Ein wichtiger Aspekt bei der Erstellung des Prompts war die Notwendigkeit einer genauen Beschreibung der Spalten und der verschiedenen 
Arten (Information, Auftrag, Feststellung, Beschluss). Ohne diese genaue Beschreibung tendierten die Tools dazu, lediglich 
Stichpunktlisten anstelle einer strukturierten Tabelle zu erstellen. Doch auch die genaue Definition der Arten führte nicht immer zu 
konsistenten Ergebnissen, da die Zuweisung der Kategorien unterschiedlich vorgenommen wurde.\\

Auf die Spalte ``Termin'' musste ebenfalls ein besonderes Augenmerk gelegt werden. Wenn im Prompt nicht explizit angegeben war, dass ein 
genaues Datum erforderlich ist, arbeiteten die Tools oft mit relativen Angaben wie ``+2 Wochen''. Selbst mit der Klarstellung, dass der 
Termin ein genaues Datum sein sollte, traten bei Le Chat Probleme auf: Es wurden immer Termine gewählt, die in der Vergangenheit lagen. 
Dazu wurde angegeben, dass die Termine beispielhaft gewählt und auf das aktuelle Datum, den 29.03.2023, bezogen seien.\\

Google Gemini hatte, wie bereits erwähnt, erhebliche Schwierigkeiten, nur die wichtigsten Punkte aus dem Gespräch herauszufiltern. 
Häufig wurden Besprechungsprotokolle mit etwa 30 Punkten erstellt, wobei jede einzelne Information separat und auch unwichtige 
Informationen aufgeführt wurden. Gemini fügte zudem eine neue Art, ``Frage'', selbstständig hinzu, wodurch im Besprechungsprotokoll 
teilweise mehrere Punkte für eine Information erstellt wurden. Bei erneuten Eingaben variierte die Länge des Besprechungsprotokolls 
stark. Trotz des Hinweises, nur die wichtigen Punkte in das Protokoll aufzunehmen, neigte Gemini weiterhin dazu, sehr detaillierte und 
kleinteilige Protokolle zu erstellen.\\

Le Chat hingegen hielt sich häufig zu knapp. Dadurch wurden immer wieder wichtige Anforderungen nicht in das Besprechungsprotokoll 
aufgenommen, und es musste besonders darauf geachtet werden, ob alle relevanten Informationen enthalten waren.

%Stichpunkte
% - [Prompt] und Gespräch als Eingabe
% - Teilweise sehr unterschiedliche Ausgaben obwohl selbe Prompts. Auch in neuen Chats bei gleichem Tool.
% Einzelne Einträge mal zu einem Eintrag zusammengefasst, mal in mehreren geschrieben. (Anz. der Einträge immer unterschiedlich).
% Beispiel Server-Client, JavaFX und RMI.
% - genaue Beschreibung der Spalten wird benötigt, da sonst eher eine Stichpunkteliste erstellt wird
% - Einzelnen Arten müssen beschrieben werden, da diese sonst sehr ungenau festgelegt werden. Aber auch mit
% Beschreibung der Arten ein Problem. Tools machen das unterschiedlich und auch in neuen Chats wird die Zuweisung immer 
% bisschen unterschiedlich gemacht. Obwohl gleiche Eingabe.
% - Im Prompt muss stehen das "Termin" ein genaues Datum benötigt. Sonst wird z.B. mit "+2 Wochen" gearbeitet. Le Chat 
% macht das dann trotzdem nicht, da muss als neue Eingabe "Der Termin soll ein genaues Datum sein" geschrieben werden.
% Datum dann allerdings in der Vergangenheit (z.B. 31.10.2023)?? Schreibt aber das die Termine beispielhaft gewählt sind
% und sich am aktuellen Datum, 29.03.2023??, gewählt sind.
% - Gemini hat großes Problem nur die wichtigsten Punkte aus dem Gespräch rauszulesen. Häufig Besprechungsprotokolle mit etwa 
% 30 Punkten. Dabei erstellt er als neue Art "Frage" und füllt diese z.B. mit "Maximale Spieleranzahl" um anschließend eine 
% Information aufzulisten mit der Beschreibung "Maximale Spieleranzahl: 4". Wenn selben Prompt in neuem Chat immer wieder 
% eingegeben wird, kommt manchmal eine "kurze" Version, welche immernoch länger ist als die Protokolle der anderen beiden Tools.
% Gemini schreibt sehr kleinkariert und mach für jede einzelne Information einen neuen Punkt. Beispiel Server-Client, JavaFX und RMI.
% - Auch zusätzlicher Hinweis im Prompt, nur die wichtigen Punkte in der Tabelle aufzunehmen, ändert bei Gemini nix daran auszuschweifen.
% - Fazit?? Selber drüber lesen und schauen ob das passt. Vorallem auf "Art" gucken und ob der Punkt wirklich wichtig ist. Le Chat
% hält sich eher zu knapp mit den Punkten, daher hier schauen ob alle wichtigen Informationen enthalten sind.

\section{Projekthandbuch} \index{Projekthandbuch} \label{CompProjekthandbuch}

Das Projekthandbuch wurde in zwei Schritten erstellt. Zunächst wurde die Einleitung mit dem Zweck des Dokuments, der Redaktion 
und dem Verteiler verfasst. Hierzu wurde der folgende Prompt im selben Chat eingegeben, in dem auch das Besprechungsprotokoll 
erstellt wurde:

\begin{prompt}[H]
    \begin{tcolorbox}[colback=gray!20, colframe=gray!20, boxrule=0pt, sharp corners] 
        Erstell mir für dieses Projekt die Einleitung für das Projekthandbuch. Die Einleitung besteht aus einem Abschnitt für 
        den Zweck des Dokuments, einen Abschnitt zur Redaktion, in welchem geklärt wird, wer für das Dokument verantwortlich ist 
        und einen Abschnitt zu dem Verteiler, also wer bei Änderungen zu informieren ist. Verantwortlich für das Dokument ist der 
        Projektleiter und über Änderungen wird das gesamte Team informiert. Dazu wird eine entsprechende Nachricht in den Discord 
        Channel geschrieben.
        \vfill
    \end{tcolorbox}
    \caption{Prompt Einleitung Projekthandbuch}
    \label{Prompt Einleitung Projekthandbuch}
\end{prompt}

Für die Einleitung sind spezifische Informationen erforderlich, die im Prompt angegeben werden müssen. In diesem Projekt sind das die 
Verantwortlichkeit des Projektleiters und der Verteiler über den Discord-Channel. Die Tools konnten die Abschnitte gut erstellen, wobei 
jedoch auffiel, dass ChatGPT im Abschnitt Verteiler alle Teammitglieder nannte, was nicht unbedingt nötig ist. Ansonsten wurden die 
Abschnitte gut erstellt [Verweis!!!!!!!!!!!!!!!!!!!!!!].\\

Der zweite Teil betrifft das Kapitel „Projektdefinition“ mit den Abschnitten „Vorgeschichte“ und „Inhaltliche Kurzdarstellung“. Dazu wurde 
der folgende Prompt verwendet, welcher nach der Ausgabe der Einleitung im Chat eingegeben wurde:

\begin{prompt}[H]
    \begin{tcolorbox}[colback=gray!20, colframe=gray!20, boxrule=0pt, sharp corners] 
        Erstelle mir nun das Kapitel "Projektdefinition" des Projekthandbuches. Der erste Abschnitt soll die Vorgeschichte des Projekts 
        beschreiben und anschließend soll ein Abschnitt eine inhaltliche Kurzdarstellung beschreiben.
        \vfill
    \end{tcolorbox}
    \caption{Prompt Projektdefinition Projekthandbuch}
    \label{Prompt Projektdefinition Projekthandbuch}
\end{prompt}

Diese zweite Eingabe erforderte keine weiteren Informationen, da diese im Gespräch bereits vorgegeben waren und die Tools darauf zugreifen 
können sollten. Auch hier wurden die Abschnitte gut erstellt, und die Tools konnten die benötigten Informationen aus dem Gespräch gut 
extrahieren [Verweis!!!!!!!!!!!!!!!!!!!!!!!!!!!!!!!!!].\\

Bei der Erstellung der Einleitung zeigte sich, dass eine genaue Beschreibung der benötigten Abschnitte entscheidend war. Ohne diese klare 
Vorgabe neigten die Tools dazu, eigene Strukturen und Inhalte zu erstellen, die nicht den Anforderungen entsprachen.\\

Außerdem traten bei Gemini teilweise Fehler in den Namen auf. Einmal wurde beispielsweise geschrieben, dass Frau Schmidt (Teamchefin) mit 
Herrn Müller (Kunde) Kontakt aufnahm, was eine fehlerhafte Zuordnung darstellt. %Dies zeigt, dass die Ausgaben der Tools unbedingt 
%probegelesen werden müssen, um solche Fehler zu korrigieren. -> Fazit/Bewertung

% - Zwei eingaben. Einmal für Einleitung mit dem Zweck des Dokuments, der Redaktion und dem Verteiler. Und einmal für die 
% Projektdefinition mit ausschließlich den Kapiteln Vorgeschichte und Inhaltliche Kurzdarstellung.
% - Muss genau Beschrieben werden was man für Abschnitte haben will, sonst wird ein eigenes Projekthandbuch erstellt, welches 
% nicht ganz den Anforderungen, welche in dieser Arbeit beschrieben wurde, entspricht.
% - Für die Einleitung werden spezifische Informationen benötigt, welche im Prompt angegeben werden müssen. In diesem Fall wurde 
% angegeben, dass die Projektleiterinfür das Projekt verantwortlich ist und als Verteiler ein Discord-Channel verwendet wird.
% - Abschnitte wurden gut erstellt. ChatGPT bennent im Abschnitt für den Verteiler einmal alle Teammitglieder, frage ob unbedingt nötig.
% - Zweite Abschnitt benötigt keine weiteren Informationen, da diese im Gespräch bereits vorgegeben sind. 
% - Abschnitte wurden hier auch gut erstellt. Die benötigten Informationen haben sich die Tools gut aus dem Gespräch raus gezogen.
% - Gemini hat teilweise dreher in den Namen drinne. Einmal hat Gemini geschrieben, dass Frau Schmidt (Teamcheffin) mit dem Herrn Müller 
% (Kunde) aufnahm. -> muss Probegelesen werden.

\section{Risikoliste} \index{Risikoliste} \label{CompRisikoliste}

Auch die Risikoliste wurde in zwei Schritten erstellt. Zunächst wurde nur die Risikotabelle erstellt und im zweiten 
Schritt die Tabelle mit den Maßnahmen. Die beiden Ergebnisse sind [Verweis!!!!!!!!!!!!!!].\\

Der Prompt für die Risikotabelle

\begin{prompt}[H]
    \begin{tcolorbox}[colback=gray!20, colframe=gray!20, boxrule=0pt, sharp corners] 
        Erstell mir für diese Projekt eine Risikoliste. Diese soll aus mehreren Spalten bestehen: "ID" für eine 
        eindeutige Identifikationsnummer, "Beschreibung" für eine ausführliche Beschreibung des Risikos und der 
        Auswirkungen, "Datum" für den Zeitpunkt, wann das Risiko identifiziert wurde, "Autor" für die Person die 
        das Risiko gemeldet hat, "Wahrscheinlichkeit (in\%)" für einen Schätzwert der Eintrittswahrscheinlichkeit 
        des Risikos, "Schaden (in €)" für eine Schätzung wie groß der Schaden ist, "Maß (in €)" was das Produkt aus 
        Wahrscheinlichkeit und Schaden ist, "Risikoklasse" für eine Priorisierung der potentiellen Risiken wo 
        zwischen Tolerierbar, Unerwünscht, Kritisch und Katastrophal Unterschieden wird und "Status" wo zwischen 
        aktiv, eingetreten und geschlossen unterschieden wird. Das Risiko ist Tolerierbar wenn das Risikomaß geringer 
        als 0,1\% des Projektvolumen ist, Unerwünscht wenn es größer als 0,1\% ist, Kritisch wenn es größer als 1\% 
        ist und Katastrophal wenn es größer als 10\% ist. In der Risikoliste sollen Team-, Technische-, Methodische-, 
        Kunden-, Fachliche-, Produkt-, Management- und Planungsrisiken betrachtet werden. Diese sollen mit einer 
        leer Zeile getrennt werden, in welchen die Oberbegriffe stehen. Das Projektvolumen beträgt 500000€.
        \vfill
    \end{tcolorbox}
    \caption{Prompt Risikotabelle}
    \label{Prompt Risikotabelle}
\end{prompt}

ist sehr umfangreich formuliert. Damit die Tabelle mit den richtigen Spalten erstellt wird, wurde im Prompt jede
einzelne Spalte aufgezählt und beschrieben. Ebenfalls ist es wichtig, die möglichen Werte für die Risikoklasse zu 
definieren, da hier sonst immer ``Hoch'', ``Mittel'' und ``Niedrig'' von den Tools verwendet wird. Es musste auch 
festgelegt werden, wann die einzelnen Risikoklassen auftreten, da die Zuweisung ansonsten recht schwammig ausfällt und
schwerwiegende Risiken eine geringere Risikoklasse erhalten als eher unwichtige Risiken. Außerdem sollte beschrieben 
werden, welche Arten von Risiken betrachtet werden sollen und dass diese Arten in einer leer Zeile stehen, welche 
die dazugehörigen Risiken von den anderen Arten trennen. Ansonsten kann es passieren, dass die Risiken durcheinander 
geschrieben werden und damit nicht den Arten zuzuordnen sind. Damit der Risikoschaden zwischen den Tools vergleichbar 
ist, sollte das Projektvolumen definiert werden. Ansonsten wird auch dies von Tools festgelegt und führt zu 
Ungenauigkeiten.\\

Grundsätzlich war die Erstellung der Risiken kein Problem. Schwierig war es eher die gewünschte Formatierung zu erhalten, 
sowie eine richtige Zuweisung der restlichen Attribute. Besonders Gemini und Le Chat haben dabei Schwierigkeiten. 
Außerdem war bei allen drei Tools auffällig, dass der Autor immer die Person ist, zu dem das Risiko in den 
Tätigkeitsbereich fällt. Also die Person, die auch der Verantwortliche ist. Fraglich ist dabei teilweise, wenn 
Kundenrisiken vom Kunden, also Herr Müller, erstellt werden, da dieser an der Erstellung der Risikoliste überhaupt 
nicht beteiligt ist.\\

ChatGPT hat die Tabelle sehr gut erstellt. Lediglich die Zuweisung der Risikoklasse war falsch. Nach einem Hinweis 
diesbezüglich wurden diese jedoch richtig korrigiert.\\

Ein Problem bei Gemini ist, dass das Risikomaß nicht richtig berechnet wird. Das Komma der Werte ist um eine Stelle zu 
weit links. Es muss alles einmal *10 gerechnet werden, damit die Werte stimmen. Kritisch bei Gemini ist, dass die Tabelle 
nicht in einem Zug erstellt werden kann. Es wird während der Generierung einfach aufgehört. Auch wenn man Gemini fragt, 
ob die vollständige Tabelle generiert werden kann, wird mitten drin aufgehört. Man muss explizit nur nach den noch offenen
Risikoarten fragen. Diese werden dann erstellt, jedoch soll man den Schaden und das Maß selbst eintragen und berechnen.
Nach anschließender Frage, ob er den Risikoschaden und das Schadensmaß festlegen kann, sagt er, dass er detaillierte 
Informationen über das Projekt und die potenziellen Risiken benötigt um genaue Werte festzulegen.

Le Chat hatte ein ähnliches Problem wie bereits im Besprechungsprotokoll [\autoref{CompBesprechungsprotokoll}], dass das 
Datum immer auf den 01.04.2023 gesetzt wird. Außerdem hatte Le Chat Probleme damit, die Tabelle in das gewünschte Format
zu bringen, dass die Risikoarten in einer Zeile stehen und die dazugehörigen Risiken darunter aufgelistet werden. Auch 
hier wurden die Risikoklassen falsch bestimmt und auch nach einem Hinweis waren diese falsch. Zudem wurde dabei der 
Risikoschaden der einzelnen Risiken geändert.\\

Die Maßnahmentabelle wurde mit folgendem Prompt erstellt:

\begin{prompt}[H]
    \begin{tcolorbox}[colback=gray!20, colframe=gray!20, boxrule=0pt, sharp corners] 
        Erstelle mir nun eine dazu passende Maßnahmentabelle. Auch diese besteht aus mehreren Spalten: "Typ" beschreibt 
        ob die Maßnahme das Risiko verhindert, lindert oder überträgt. In "Beschreibung" wird die Maßnahme beschrieben. 
        Falls ein Risiko als nicht mehr relevant eingestuft wird, wird in der Spalte "Beschreibung" eine Begründung 
        eingetragen und die Maßnahme auf "beendet" gesetzt. "Trigger" ist das Ereignis, das den Start der Maßnahme 
        veranlasst, falls diese nicht sofort eingeleitet werden soll. "Verantwortlicher" ist die zuständige Person für 
        die Durchführung der Maßnahme. "Status" unterscheidet zwischen geplant, aktiv und beendet. Anschließend gibt es 
        jeweils eine Spalte für "Restwahrscheinlichkeit (in \%)", "Restschaden (in €)", "Restmaß (in €)" und "Restklasse" 
        was die geschätzte Wahrscheinlichkeit, geschätzter Schaden, Maß und Klasse des Restrisikos, nach Durchführung der 
        Maßnahme entsprechen. Für jedes Risiko sollen zwei Maßnahmen erstellt werden. Dazu wird über die zwei Maßnahmen 
        eine Zeile mit der ID von dem Risikon beschrieben, auf die sich die Maßnahmen beziehen.
        \vfill
    \end{tcolorbox}
    \caption{Prompt Risikotabelle}
    \label{Prompt Risikotabelle}
\end{prompt}

Der Prompt für die Maßnahmen ist ähnlich wie der, für die Risikotabelle. Es wird jede Spalte der Tabelle einmal beschrieben,
damit das Format der Tabelle mit den Vorgaben übereinstimmt. Anschließend wird gesagt, dass die ID des dazugehörigen Risikos
mit in die Tabelle übernommen werden soll, damit man die Maßnahmen den Risiken zuordnen kann. Die Probleme sind hier 
ähnlich zu denen bei der Risikotabelle, jedoch erstellen alle drei Tools grundsätzlich anständige Maßnahmen für die Risiken.
Bei allen dreien fühlt sich allerdings die Zuweisung der Attribute zu den Maßnahmen repetitiv an. Häufig wechseln sich die 
auswählbaren Parameter ab und auch die Wahrscheinlichkeiten und der Restschaden sind meistens immer wieder die gleichen Zahlen.\\

ChatGPT erstellt auch die Maßnahmentabelle sehr gut. Lediglich die Zuweisung der Restklasse stimmt nicht überein.\\

Bei Gemini wird für die Begründung, warum ein Risiko als nicht mehr relevant eingestuft wird, eine eigene Spalte erstellt.
Auch wenn man die Beschreibung dafür im Prompt ändert, wird die Spalte erstellt. Außerdem ist die Formatierung bei Gemini
manchmal nicht so gut, da bei jeder 2. Zeile alle Daten ab der Spalte "Begründung" um eine Stelle nach links verschoben wird.
Lässt man sich die Antwort neu generieren und auch wenn man Gemini sagt er soll die Formatierung korrigieren bleibt das 
Problem bestehen. Ein weiteres Problem ist, dass die Spalte "Trigger" nicht gefüllt wird und auch wieder das Problem, dass
die Tabelle nicht ganz vollständig generiert wird sondern mitten drin aufhört, tritt wieder auf. Teilweise ist das Restmaß 
falsch berechnet und auch die Zuweisung der Restklasse ist nicht immer richtig.\\

Le Chat [[[[[[[[[[[[[[[[[[[[[[[[[[[[[[[[[[[[[[[[[[[[[]]]]]]]]]]]]]]]]]]]]]]]]]]]]]]]]]]]]]]]]]]]]]

% - Auch hier zwei einzelne Eingaben. Einmal für die Tabelle mit den Risiken und einmal für die mit den Maßnahmen.
% - Prompt für Risikotabelle sehr umfangreich. Jede Spalte erklärt damit die Tabelle, die erstellt wird, den Vorgaben entspricht. 
% Auch das Risiken jeweils zu den Oberbegriffen Team-, Technische-, Methodische-, Kunden-, Fachliche-, Produkt-, Management- und 
% Planungsrisiken erstellt werden sollen und die Risiken dazu mit einer leer Zeile getrennt werden sollen für eine bessere Übersicht 
% und um die Risiken leichter Zuordnen zu können.
% - Die Tools haben die Zuweisung der Risikoklasse zunächst mit den Werten niedrig, mittel und hoch erstellt. Daher wurde der 
% Prompt um die Anweisung, dass zwischen tolerierbar, unerwünscht, kritisch und katastrophal gewählt werden soll ergänzt.
% Insgesamt war die Zuordnung jedoch ohne richtiges Konzept und schwerwiegende Risiken eine geringe Risikoklasse als eher unwichtigere
% Risiken. Daher wurde die definition, wann welche Risikoklasse zu wählen ist, ebenfalls ergänzt.
% - Damit Schaden und die Risikoklassen richtig gewählt werden und die Tools vergleichbar sind, wurde das Projektvolumen auf 500000€
% festgelegt. Ansonsten wählen die Tools ein eigenes, was zu Ungenauigkeiten führt.
% - Die Erstellung der Risiken war kein Problem, eher bei der Formatierung der Tabelle und der Zuweisung der restlichen Attirbute haben
% Gemini und Le Chat Schwierigkeiten.

% - ChatGPT hat bei der ersten eingabe die Risikoklassen falsch zugewiesen. Bei Hinweis darauf wurde dies jedoch richtig korriegiert.
% Autor eher der Verantwortliche für das Risiko. Macht aber irgendwo Sinn, da die "Experten" auf ihrem Gebiet, am ehesten die Risiken
% für ihr Gebiet kennen und erstellen können.

% - Gemini legt den Autor, auf den am ehesten Verantwortlichen für das Risiko, fest. Also Teamrisiken werden von Frau Schmidt definiert,
% während zum Beispiel Anforderungsrisiken vom Anforderungsanalysten (Herr Schneider) und Kundenrisiken vom Kunden (Herr Müller) geschrieben
% werden. Außerdem berechnet Gemini das Risikomaß nicht richtig. Das Komma der Werte ist um eine Stelle zu weit links. Es muss alles *10 
% gerechnet werden. Die ID setzt Gemini einfach auf 1, 2, 3, ... was grundsätzlich aber nicht schlimm ist, da kein großer Mehraufwand, dies 
% selbst zu ändern. Gemini kann nicht gesamte Tabelle aufeinmal Generieren. Es wird einfach mitten drinn aufgehört. Bei nachfrage ob er nur 
% noch Produkt-, Management- und Planungsrisiken erstellen kann, macht er dies, fügt aber keinen Schaden und Maß mehr ein. Wenn ich ihm sag, 
% das er den Schaden und Maß noch festlegen soll, sagt er nur, dass er das ohne detaillierte Informationen über das Projekt und die potenziellen
% Risiken nicht kann.

% - Le Chat hatte wieder das Problem mit dem Datum, dass dieses auf den 01.04.2023 gesetzt wurde. Manchmal war außerdem das Format der Tabelle 
% nicht richtig und die Risiken wurde manchmal den Oberbegriffen nicht zugeordnet sondern zufällig erstellt. Musste schritt für schritt beschreiben 
% was Le Chat ändern soll. Auch Le Chat bestimmt die Risikoklasse falsch. Auch nach Hinweis, dass diese nicht stimmen sind diese noch falsch 
% und es wird der Schaden bei den Risiken geändert. Auch hier ist der Autor eher der Verantwortliche.


- Prompt für Maßnahmen ähnlich, dass jede Spalte der Tabelle einmal beschrieben wird und anschließend gesagt wird, dass die ID des dazugehörigen
Risikos übernommen werden soll.

- ChatGPT nur ein Fehler bei Restklasse. Sonst passt alles.

- Gemini erstellt für die Begründung, warum ein Risiko als nicht mehr relevant eingestuft wird eine eigene Spalte. Ist aber nicht so schlimm, da 
zu beginn eh leer. Formatierung schlecht, da bei jeder 2. Zeile alle Daten ab Spalte "Begründung" um eine Stelle nach links verschoben wird. Neu 
generierung und auch Hinweis darauf ändert nichts. Trigger auch nicht gefüllt. Außerdem ist der Typ immer abwechselnd "Verhindern" und "Lindern".
Auch hier ist Teilweise das Restmaß falsch berechnet. Hört bei Risiko 4.2 auf mit der generierung der Maßnahmen. Maßnahmen passen aber zu Risiken.

- Le Chat erstellt die Maßnahmen nicht immer zu den Risiken in der Risikotabelle, sonder zu den Oberbegriffen Team-, Technische-, Methodische-, 
Kunden-, Fachliche-, Produkt-, Management- und Planungsrisiken. Maßnahme 1-4 scheinen sehr allgemein formuliert, Maßnahmen 5-8 würden
zu den 1. Risiken des jeweiligen Oberbegriffes passen. Auch hier scheinen die Attribute sehr repetitiv. Wahrscheinlichkeit immer bei 5\%,
die Risikoklasse außer bei der ersten Maßnahme immer Unerwünscht und der Restschaden auch meistens bei den 2 Maßnahmen für ein Risiko gleich.

\section{Anforderungsspezifikation} \index{Anforderungsspezifikation} \label{CompAnforderungsspezifikation}

Einleitung:
- Beschreibung, welche Absätze erstellt werden sollen. Da ``Zweck und Umfang des Dokuments'' recht spezifisch, genauer definieren was für ein 
Inhalt benötigt wird.

- ChatGPT stellt nur textuellen Verweis auf Spielregeln. Müssen noch selber eingefügt werden, falls gewünscht. Fraglich ob Begriff "Punktesystem"
erklärt werden muss. Sonst gut.

- Gemini:?????????????????????????????????

- Le Chat verweist auf Link aus dem Internet. Macht man das in einem offizielen Dokument? Sonst gut.


Systemarchitektur:
- Prompt: Inhalt der Systemarchitektur spezifizieren. Mit Nachrichten vorher im Chat wird diese gut erstellt. (Gemini ausstehend.......)

- ChatGPT stellt die Client-Server und Server-Datenbank Verbindung zunächst unidirektional da, nach Hinweis korrigiert 
er dies aber richtig. Hat in die Komponenten schon genauere Teile definiert, schon zu detailiert? Beschreibung gut und 
hat in Chat Skizze "gezeichnet".

- Gemini: [[[[[[[[[[[[[[[[[[[[[[[[[[[[[[[[[[[[[[[[[[[[[[]]]]]]]]]]]]]]]]]]]]]]]]]]]]]]]]]]]]]]]]]]]]]]

- Le Chat: Kann die Abbildung nicht zeichnen. Gibt Text nach erstem Prompt aber gut aus. Nach Eingabe "Beschreib mir, 
wie die Abbildung aussehen soll." Erhält man beschreibung, welche Komponenten vorhanden sind, jedoch nicht die 
Beziehungen zueinander. Bei Frage "Wie ist die Beziehung zwischen der Client-Anwendungs, Server-Anwendung und der 
Datenbank?" Wird erklärt wie der Datenfluss abläuft (Bidirektional). Infos reichen um skizze selber zu erstellen.


Fachliches Datenmodell:
- Prompt: Information, das ein UML-Klassendiagramm verwendet werden soll, um entsprechendes Layout zu erhalten. 
Außerdem beschreibung, welche Infos benötigt werden. Inhalt muss nicht beschrieben werden.

- ChatGPT: 
    - Pfeile zeigen in gleiche Richtung -> sagt "stimmt sollten Bidirektional sein,..." und generiert neu. Fehler bleibt aber
    - Spiel-Spieler Beziehung 4..*? Sollen doch 0..4 Teilnehmen oder garkeine. -> Ändert auf 1..4, richtige Bezeichnung? weiß ich auch nicht.
    - spielBrett als String? -> Wird entfernt bei Nachfrage
    - Wofür zugNummer? -> Reihenfolge der Spielzüge innerhalb eines Spiels festzuhalten
    - Beziehungen von Spielfigur im Diagramm nicht richtig dargestellt. -> Bei Hinweis wird nicht korrigiert. ``Kann'' Beziehungen immer mit 1..
    - Feld braucht Attribut ob belegt oder nicht -> Korrigiert
    - Chat nicht richtig im UML dargestellt, Beziehungen falsch -> Bei Hinweis wird nicht korrigiert
    - Spieler in User und Player aufteilen? -> Macht er richtig. Frage nur ob dann pos in Spieler oder figur?
    - status in Spiel zu einem Enum statt String?
    - farbe in Spieler zu Enum statt String?
    - Spielzug: figurID statt spielfigur? Beziehung zu Spieler?
    - Spielfigur: braucht farbe wenn Spieler schon farbe hat?
    - Bei Prompt Nr. 7, bekommt Spieler ein Attribut pos: int?? und Spieler hat keine Farbe mehr? -> Bei Hinweis Farbe wieder eingefügt und Position entfernt
    - Ändert bei Neugenerierung immer wieder ein oder zwei Beziehungen. -> Lassen sich aber Händisch vervollständigen
    - Für Würfel wird zunächst eine eigene Entität erstellt, nach Frage ob nicht auch einfach Attribut in Spiel reicht, wird dies geändert

    - Hat keine Einschränkungen zu den Attributen festgelegt. -> Wird nach Hinweis gemacht
    - Legt nicht alle Beziehungen fest.

    Prompts:
        - Erstell mir nun für die Anforderungsspezifikation das Fachliche Datenmodell. Dieses soll mit Hilfe eines UML Klassendiagrammes mit ergänzenden Beschreibungen bzw. EInschränkungen spezifiziert werden. Das Modell soll alle Entitätstypen mit deren Eigenschaften, Beziehungen und Einschränkungen besitzen.
        - Im UML-Diagramm zeigen immer beide Pfeile in eine richtung, müssen diese nicht bidirektional zeigen?
        - Du hast die Beziehung "Ein Spiel hat mehrere Spieler" auf 4..* festgelegt. Sollte diese nicht 1 - 4 betragen? Es sollen ja maximal 4 Spieler teilnehmen können.
        - Braucht das Feld nicht noch ein Attribut ob es belegt ist oder nicht?
        - Macht es nicht auch sinn Spieler in zwei Entitäten aufzuteilen? Einmal "User" und einmal "Player".  "User" beinhaltet dabei den Namen, die Email, das Passwort und die Punkte. "Spieler" beinhaltet dann die nötigen Attribute zum Spielen wie SpielerID, die Farbe und die position seiner Spielfiguren.
        - Erstell mir zu den Attributen noch Einschränkungen, in dem du zu jedem Attribut eine Erklärung schreibst und ein Format definierst.
        - Benötigt man für jede entität unbedingt eine ID? Außerdem sind einige Beziehungen falsch: Ein User kann ein Spieler sein (0..*), Ein Spiel kann mehrere Spielzüge haben (0..*), ein Feld kann von einer Spielfigur besetzt sein (0..1), Ein Spieler kann an einem Spiel teilnehmen (1) und es fehlen die Beziehungsbeschreibung von Spieler zu Spielfigur, Spieler zu Chat, Spiel zu Chat, Spielfigur zu Spielzug. Dadurch, dass die Spielfigur nur genau einem Spieler zugeordnet ist, braucht die Spielfigur kein Attribut Farbe oder nicht?
        - Wieso hat Spieler keine Farbe mehr
        - Wieso hat Spieler ein Attribut Position
        - Wie setzt du das Spielbrett als String um? Und warum muss dieses gespeichert werden? Kann man das nicht einfach als Graphische Oberfläche implementieren?
        - Muss man den Würfel in das Fachliche Datenmodell übernehmen? 
        - Reicht es nicht den Würfel als Attribut in Spiel aufzunehmen?

        - Gemini: [[[[[[[[[[[[[[[[[[[[[[[[[[[[[[[[[[[[[[[[[[[[[]]]]]]]]]]]]]]]]]]]]]]]]]]]]]]]]]]]]]]]]]]]]]


- Le Chat:
    - Nur beschreibung
    - Hat auch nur Spieler, "user" und "player" würden sinn machen.
    - Spielfigur an einem oder mehreren Spielen teilnehmen?
    - Spieler-Spielfigur: Ein Spieler hat genau 4 Figuren, nicht weniger und nicht mehr
    - Beziehung zwischen Spiel-Spielfigur? wenn ja, Spieler * 4 Spielfiguren.
    - Punkte werde nicht gespeichert.
    - Player hat position? Player kann an mehreren Spielen teilnehmen? -> Nach Hinweis geändert
    - Player und Spielfigur haben positionen.
    - Spiel-Spielfigur Beziehung sinnvoll?

    - Probleme mit der Beschreibung der Beziehungen zwischen den Entitäten
    - Ein paar grundsätzliche Einschränkungen aufgelistet, aber nicht zu jedem Attribut -> Sagt nur ob Ganzzahl oder Text,
    keine Begrenzung in der Länge.

    Prompts:
        - Erstell mir nun für die Anforderungsspezifikation das Fachliche Datenmodell. Dieses soll mit Hilfe eines UML Klassendiagrammes mit ergänzenden Beschreibungen bzw. EInschränkungen spezifiziert werden. Das Modell soll alle Entitätstypen mit deren Eigenschaften, Beziehungen und Einschränkungen besitzen.
        - Macht es nicht auch sinn "Spieler" in zwei Entitäten aufzuteilen? Einmal "User" und einmal ``Player''.  "User" beinhaltet dabei den Namen, die Email, das Passwort und die Punkte. "Spieler" beinhaltet dann die nötigen Attribute zum Spielen wie SpielerID, die Farbe und die position seiner Spielfiguren.
        - Ein Player hat 4 Figuren. Außerdem hat der Player keine Positionen, diese haben ja die Figuren. Ein Player soll auch nicht an mehreren Spielen teilnehmen, sondern der User kann mehrere Player sein und ein Player spielt ein Spiel. Ein Spiel besteht aus 1 bis 4 Playern.


Anwendungsfalldiagramm:

- Prompt: Reicht zu sagen, dass das Diagramm alle Anwendungsfälle beinhalten soll und den Rollen zugewiesen werden soll.

- ChatGPT:
    - Spiel erstellen: wird zum Gastgeber. Besondere rechte? wenn ja, warum keine neue Rolle. -> Nach Hinweis neue Rolle
    - Als User Chat nutzen? In Anforderungen wurde nur von Chat während des Spiels gesprochen. -> Nach Hinweis wurde entfernt.
    - Was gehört zu Zug machen, wenn Spielfigur bewegen eigener Anwendungsfall ist -> gesamte Ablauf wenn Spieler am Zug ist
    (Würfeln, Auswählen und Bewegen der Spielfigur, Schlagen gegnerischer Figuren und Beenden des Zuges), AF "Spielfigur bewegen" 
    war doppelt. Wurde aber nach Hinweis entfernt. Sehr großer Anwendungsfall, vielleicht lieber in kleinere AF aufteilen??
    - Erstellt Code um in einem PlantUML-Editor oder einem Online Renderer das Diagramm als Bild zu erstellen
    - Erstellt Skizze in PlantUML ganz gut, sieht lediglich aus als ob auch ``Spieler'' von ``User'' erbt.

- Gemini: [[[[[[[[[[[[[[[[[[[[[[[[[[[[[[[[[[[[[[[[[[[[[]]]]]]]]]]]]]]]]]]]]]]]]]]]]]]]]]]]]]]]]]]]]]

- Le Chat:
    - Rollen werden gut erstellt.
    - Passende Anwendungsfälle, stimmen auch zu den Rollen
    - "Benutzer verwalten" von Admin vielleicht etwas ungenau? -> Wird geändert und entsprechend ergänzt
    - möglich mit PlantUML erstellen zu lassen, Diagramm jedoch recht durcheinander und Anwendungsfall ``Passwort vergessen'' wurde ``User'' 
    zugewiesen und nicht wie in der Beschreibung ``Gast''


AF:

- Prompt: Beschreibung, welcher Anwendunsfall beschrieben werden soll, welche Informationen zunächst erstellt werden soll und anschließend das ein 
Aktivitätsdiagramm erstellt werden soll. Beschrieben das die Aktivitäten dem Server oder dem Client zugewiesen werden sollen, um die spätere 
implementierung der Server-Client-Applikation einfacher fällt. Außerdem erwähnt das keine Dialog eigenschaft beschrieben werden sollen.

- ChatGPT-3.5:
    - Häufigkeit keine Zahl: "Häufig verwendet (mehrmals täglich)
    - Erstellt diagramm direkt mit, kein Endzustand. Verbindungen führen nach ``Anmeldung erfolgreich'' und ``Anmeldung fehlgeschlagen'' wieder zusammen.
    - Beschreibungen gut.
    - ChatGPT-4o macht zu viel. Implementierung wird bisschen mit beschrieben (Benutzersession erstellen) sowie Dialog 
    eigenschaften wie Anzeigen von Fehlermeldungen werden beschrieben. GPT-3.5 hier tatsächlich besser. Außerdem wurde Code
    von PlantUML falsch erstellt (Syntax error)


- Gemini: [[[[[[[[[[[[[[[[[[[[[[[[[[[[[[[[[[[[[[[[[[[[[[[]]]]]]]]]]]]]]]]]]]]]]]]]]]]]]]]]]]]]]]]]]]]]]]

- Le Chat:
    - Beschreibt für die Endzustände trotz Hinweis, das keine Dialog eigenschaften beschrieben werden sollen, was der Dialog anzeigt.
    - Häufigkeit keine Zahl
    - Kann auf Nachfrage Code für PlantUML erstellen.
    - Diagramm hat nur einen Endzustand, wo Verbindung von ``Anmeldung erfolgreich'' und ``Anmeldung fehlgeschlagen'' enden.


Benutzungsschnittstelle:

- Prompt:

- ChatGPT:
    - Man startet im Hauptmenü und sieht alle Buttons. ``Spiel erstellen'' und ``Spiel beitreten'' können jedoch nur 
    genutzt werden, wenn man Angemeldet ist.
    - Erstellt Skizze und Textuelle Beschreibung. Skizze passt aber nicht ganz zu Beschreibung. Auch nach Hinweis wird 
    selber Fehler nochmal gemacht. Wenn man auf die wiederholt Falsche Antwort negativ Reagiert mit dem Feedback, dass 
    die Hinweise nicht richtig umgesetzt wurden, und dann im als Prompt die Eingabe ``Die Skizze ist unverändert. 
    Setzt den "Spiel beitreten"-Dialog mit dem Hauptmenü in Verbindung.'' gibt, wird die Skizze richtig erstellt.
    - Es wird auch beschrieben, welche Buttons und Eingabefelder die einzelnen Dialoge besitzen. Dies ist jedoch 
    an dieser Stelle nicht nötig.
    - Skizze kann auch nicht zu einem Dialog zurück gehen. Wird dann nochmal aufgeführt (ohne Buttons,...)


- Gemini:

- Le Chat:
    - Erstellt für jeden Anwendungsfall einen Dialog, grundsätzlich gut.
    - Spielbrett benötigt auch ein Dialog?
    - Verbindungen zwischen den Dialogen macht keinen Sinn -> Nach Hinweis wird erneut ausgegeben, Beschreibung besser
    aber Skizze immer noch unsinnig.
    - Beschreibung beinhaltet aber nicht alle Anwendungsfälle (Admin wird ausgelassen). Außerdem werden Dialoge erstellt,
    für die es keine Anwendungsfälle gibt (z.B. "Profil bearbeiten" um Benutzerdaten selber zu ändern)
    - Auch nach genauer Beschreibung wird Skizze nicht richtig erstellt.
    


Dialog:

- Prompt: Beschreibung des Inhalts reicht.

- ChatGPT:
    - Erstellt Skizze
    - beschreibt Validierung. Richtigen Stelle dafür?

- Gemini:

- Le Chat: 
    - kleiner Formulierungsfehler bei ``Abbrechen''-Button. "Nutzer bleibt angemeldet. -> Wird bei Hinweis korrigiert
    - Erklärt Skizze gut bei Nachfrage.


Nichtfunktionale Anforderungen:

- Prompt: 

- ChatGPT:
    - passt

- Gemini:

- Le Chat: 
    - passt