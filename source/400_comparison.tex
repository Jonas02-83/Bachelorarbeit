
\chapter{Praxisergebnisse und Vergleich} \index{Praxisergebnisse und Vergleich}

TODO!!!!!!!!

Generell:
    - Schwierigkeiten in einem gorßen/langen Chat, die Tools von einer Formulierung weg zu bringen. 
    - Le Chat erstellt immer Link mit Bild aber /Xyxyxyxy ?
    - Le Chat warnt manchmal vor Eingabe: "Der Inhalt kann gefährliche oder sensible Themen beinhalten" für Eingabe:
    "Formulier die Beschreibung der Aktivitäten etwas genauer. Beschreibe welcher Wert wo gespeichert wird mithilfe des Fachlichen Datenmodells."


\section{Bewertung} \index{Bewertung} \label{BewertungLLMTools}

Um die Ausgaben der Tools besser zu vergleichen und ein nachvollziehbares Fazit zu ziehen, wurde ein Bewertungssystem 
entwickelt, das fünf Stufen umfasst. Diese Bewertungsstufen ermöglichen eine strukturierte und transparente Methode 
zur Beurteilung der Qualität der Ausgaben von LLM Tools. Die fünf Bewertungsstufen sind wie folgt definiert:\\

``Sehr gut'' wird vergeben, falls der Inhalt der Ausgabe Fachlich richtig ist und auch das Format den Vorgaben entspricht.
Also, dass die Ausgabe genauso in das entsprechende Dokument eingefügt werden kann.\\
Die Ausgabe wird als ``Gut'' eingestuft, wenn der Inhalt etwas fehlerhaft ist, man aber mit einpaar wenigen Eingaben 
den Inhalt korrigieren kann und das Format den Vorgaben entspricht. Eine Ausgabe wird auch als ``Gut'' bewertet, wenn 
der Inhalt richtig ist, aber das Format etwas fehlerhaft ist. Mann sollte aber den richtigen Inhalt einfach aus der 
Ausgabe kopieren können und in das entsprechende Dokument mit dem richtigen Format einfügen können.\\
Als ``Befriedigend'' wird eine Ausgabe bewertet, wenn der Inhalt und das Format etwas fehlerhaft sind, sich diese aber 
mit einpaar wenigen Eingaben korrigieren lassen oder man den richtigen Inhalt aus der Ausgabe kopieren kann.\\
``Ausreichend'' wird vergeben, wenn in der Ausgabe Inhaltlich Teile fehlen, man aber mit ganz genauen Eingaben, von dem 
was man will, die Ausgabe etwas verbessert, aber diese immer noch nicht vollständig ist.\\
Zuletzt wird eine Ausgabe als ``Mangelhaft'' bewertet, wenn der Inhalt der Ausgabe falsch oder nicht verständlich ist
und es sich auch nicht korrigieren lässt.\\

Außerdem fließen auch spezifische Aspekte, die nur für das jeweilige Aspekt wichtig sind, in die Bewertung mit ein. 
Diese werden in den entsprechenden Kapiteln für die Dokumente im Fazit erläutert.


\section{Besprechungsprotokoll} \index{Besprechungsprotokoll} \label{CompBesprechungsprotokoll}

Für die Erstellung des Besprechungsprotokoll wurde ein verschriftliches Gespräch verwendet [\autoref{Kundengespräch1} 
und \autoref{Kundengespräch2}]. Mit diesem Gespräch und dem Prompt:
    
\begin{prompt}[H]
    \begin{tcolorbox}[colback=gray!20, colframe=gray!20, boxrule=0pt, sharp corners] 
        Erstell mir aus folgendem Gespräch ein erstes Besprechungsprotokoll für ein Projekt. Das Protokoll soll nur 
        die wichtigen Punkte in Form einer Tabelle mit den Spalten "Nummer": was eine eindeutige Nummer zur 
        Identifikation ist, "Art": eine Auswahl ob es eine Information, ein Auftrag, eine Feststellung oder eine 
        Beschluss ist, "Beschreibung": Was den Punkt kurz und präzise zusammenfasst, "Termin": ein genaues Datum bis 
        wann der Auftrag erledigt sein muss und "Verantwortlich": Welches Teammitglied verantwortlich ist, enthalten. 
        Dabei müssen Aufträge mit einem genauen Fälligkeitsdatum und einem Verantwortlichen versehen sein. Beschlüsse 
        müssen klar und unmissverständlich formuliert werden. Feststellungen sind Beschlüsse, die keine Abstimmung 
        benötigen und Informationen bieten den Projektmitgliedern wichtige Hinweise: 
        [Hier folgte das Gespräch]
        \vfill
    \end{tcolorbox}
    \caption{Prompt Besprechungsprotokoll}
    \label{Prompt Besprechungsprotokoll}
\end{prompt}

wurden dann bei ChatGPT, Gemini und Le Chat die Eingabe getätigt. Die Ergebnisse [[[[[[[[[[[[[[[[]]]]]]]]]]]]]]]].\\

Zunächst fiel auf, dass trotz der Verwendung desselben Prompts die Ausgaben der verschiedenen Tools teilweise erheblich abwichen. 
Dies betraf sowohl die Anzahl der erstellten Einträge im Besprechungsprotokoll als auch die Art und Weise, wie die Informationen 
zusammengefasst wurden. In einigen Fällen wurden mehrere Punkte zu einem Eintrag zusammengefasst, während in anderen Fällen mehrere 
Einträge daraus entstanden. Dies trat insbesondere bei der Anforderung auf, dass die Anwendung mit JavaFX als Server-Client-Architektur 
mit RMI entworfen werden sollte. Hier haben die Tools teilweise einen Eintrag dafür erstellt und manchmal drei einzelne Einträge. 
Dieses Verhalten zeigte sich auch, wenn derselbe Prompt in neuen Chats mit demselben Tool verwendet wurde. Besonders bei Gemini trat 
dieses Problem sehr häufig und extrem auf.\\

Ein wichtiger Aspekt bei der Erstellung des Prompts war die Notwendigkeit einer genauen Beschreibung der Spalten und der verschiedenen 
Arten (Information, Auftrag, Feststellung, Beschluss). Ohne diese genaue Beschreibung benutzten die Tools ein eigenes Format, was eher  
einer Stichpunktlisten ähnelte. Doch auch die genaue Definition der Arten führt nicht immer zu konsistenten Ergebnissen, da die Zuweisung 
der Kategorien häufig unterschiedlich vorgenommen werden.\\

Auf die Spalte ``Termin'' musste ebenfalls ein besonderes Augenmerk gelegt werden. Wenn im Prompt nicht explizit angegeben war, dass ein 
genaues Datum erforderlich ist, arbeiteten die Tools oft mit relativen Angaben wie ``+2 Wochen''. Selbst mit der Klarstellung, dass der 
Termin ein genaues Datum sein sollte, traten bei Le Chat Probleme auf: Es wurden immer Termine gewählt, die in der Vergangenheit lagen. 
Dazu wurde angegeben, dass die Termine beispielhaft gewählt und auf das aktuelle Datum, den 29.03.2023, bezogen seien.\\

Google Gemini hatte, wie bereits erwähnt, erhebliche Schwierigkeiten, nur die wichtigsten Punkte aus dem Gespräch herauszufiltern. 
Häufig wurden Besprechungsprotokolle mit etwa 30 Punkten erstellt, wobei jede einzelne Information separat und auch unwichtige 
Informationen aufgeführt wurden. Gemini fügte zudem eine neue Art, ``Frage'', selbstständig hinzu, wodurch im Besprechungsprotokoll 
teilweise mehrere Punkte für eine Information erstellt wurden. Bei erneuten Eingaben variierte die Länge des Besprechungsprotokolls 
stark. Trotz des Hinweises, nur die wichtigen Punkte in das Protokoll aufzunehmen, neigte Gemini weiterhin dazu, sehr detaillierte und 
kleinteilige Protokolle zu erstellen.\\

Le Chat hingegen hielt sich häufig zu knapp. Dadurch wurden immer wieder wichtige Anforderungen nicht in das Besprechungsprotokoll 
aufgenommen, und es musste besonders darauf geachtet werden, ob alle relevanten Informationen enthalten waren. Besonders, dass mehrere
Spiele gleichzeitig gespielt werden können und das ansprechende Animationen verwendet werden sollen, wurden nicht mit aufgeführt.\\

Zusammenfassend lässt sich sagen, dass die Erstellung des Besprechungsprotokolls durch die LLM Tools grundsätzlich schon ganz gut 
funktioniert. Besonders ChatGPT hat wenig Schwankungen in seinen generierten Ausgaben und hat eigentlich immer die gleichen 
Punkte im Besprechungsprotokoll, formuliert diese nur ein bisschen unterschiedlich. Daher wird ChatGPT mit ``Sehr gut'' bewertet.
Die anderen beiden haben damit etwas mehr Schwierigkeiten, doch wenn man sich das Besprechungsprotokoll immer wieder neu generieren 
lässt, kommt irgendwann ein anständig erstelltes Protokoll. Da die Zusammenfassung des Gesprächs auf die wesentlichen Punkte, die 
Hauptaufgabe eines Besprechungsprotokolls ist, erhalten Le Chat und Gemini lediglich die Bewertung ``Ausreichend''. Das Le Chat und 
Gemini nicht alle Anforderungen in das Protokoll aufnehmen, kann, falls dies nicht auffällt, im Nachhinein das Projekt verzögern 
und die Kosten in die Höhe treiben. Geminis erstelle Protokolle mit etwa 30 Punkten verfehlen dahingegen komplett denn Sinn eines
Besprechungsprotokoll.

\section{Projekthandbuch} \index{Projekthandbuch} \label{CompProjekthandbuch}

Das Projekthandbuch wurde in zwei Schritten erstellt. Zunächst wurde die Einleitung mit dem Zweck des Dokuments, der Redaktion 
und dem Verteiler verfasst. Hierzu wurde der folgende Prompt im selben Chat eingegeben, in dem auch das Besprechungsprotokoll 
erstellt wurde:

\begin{prompt}[H]
    \begin{tcolorbox}[colback=gray!20, colframe=gray!20, boxrule=0pt, sharp corners] 
        Erstell mir für dieses Projekt die Einleitung für das Projekthandbuch. Die Einleitung besteht aus einem Abschnitt für 
        den Zweck des Dokuments, einen Abschnitt zur Redaktion, in welchem geklärt wird, wer für das Dokument verantwortlich ist 
        und einen Abschnitt zu dem Verteiler, also wer bei Änderungen zu informieren ist. Verantwortlich für das Dokument ist der 
        Projektleiter und über Änderungen wird das gesamte Team informiert. Dazu wird eine entsprechende Nachricht in den Discord 
        Channel geschrieben.
        \vfill
    \end{tcolorbox}
    \caption{Prompt Einleitung Projekthandbuch}
    \label{Prompt Einleitung Projekthandbuch}
\end{prompt}

Für die Einleitung sind spezifische Informationen erforderlich, die im Prompt angegeben werden müssen. In diesem Projekt sind das die 
Verantwortlichkeit des Projektleiters und der Verteiler über den Discord-Channel. Es fiel lediglich auf, dass ChatGPT im Abschnitt 
``Verteiler'' alle Teammitglieder nannte, was nicht unbedingt nötig ist. Ansonsten wurden die Abschnitte gut erstellt [Verweis!!!!!!!!!!!!!!!!!!!!!!].\\

Der zweite Teil betrifft das Kapitel ``Projektdefinition'' mit den Abschnitten ``Vorgeschichte'' und ``Inhaltliche Kurzdarstellung''. Dazu wurde 
der folgende Prompt verwendet, welcher nach der Ausgabe der Einleitung im Chat eingegeben wurde:

\begin{prompt}[H]
    \begin{tcolorbox}[colback=gray!20, colframe=gray!20, boxrule=0pt, sharp corners] 
        Erstelle mir nun das Kapitel "Projektdefinition" des Projekthandbuches. Der erste Abschnitt soll die Vorgeschichte des Projekts 
        beschreiben und anschließend soll ein Abschnitt eine inhaltliche Kurzdarstellung beschreiben.
        \vfill
    \end{tcolorbox}
    \caption{Prompt Projektdefinition Projekthandbuch}
    \label{Prompt Projektdefinition Projekthandbuch}
\end{prompt}

Diese zweite Eingabe erforderte keine weiteren Informationen, da diese im Gespräch bereits vorgegeben waren und die Tools darauf zugreifen 
können sollten. Auch hier wurden die Abschnitte gut erstellt, und die Tools konnten die benötigten Informationen aus dem Gespräch gut 
extrahieren [Verweis!!!!!!!!!!!!!!!!!!!!!!!!!!!!!!!!!].\\

Bei der Erstellung der Einleitung zeigte sich, dass eine genaue Beschreibung der benötigten Abschnitte entscheidend war. Ohne diese klare 
Vorgabe neigten die Tools dazu, eigene Strukturen und Inhalte zu erstellen, die nicht den Anforderungen entsprachen.\\

Außerdem traten bei Gemini teilweise Fehler in den Namen auf. Einmal wurde beispielsweise geschrieben, dass Frau Schmidt (Teamchefin) mit 
Herrn Müller (Kunde) Kontakt aufnahm, was eine fehlerhafte Zuordnung darstellt.\\

Abschließend lässt sich sagen, dass die Abschnitte von allen drei Tools gut waren. Der Fehler von Gemini zeigt allerdings, dass man sich 
die Ausgaben trotzdem nochmal achtsam durchlesen sollte. Aufgrund der Bewertungskriterien wird Gemini durch seinen Fehler mit ``Gut''
bewertet. Le Chat und Gemini werden beide mit ``Sehr gut'' benotet.

\section{Risikoliste} \index{Risikoliste} \label{CompRisikoliste}

Auch die Risikoliste wurde in zwei Schritten erstellt. Zunächst wurde nur die Risikotabelle erstellt und im zweiten 
Schritt die Tabelle mit den Maßnahmen. Die beiden Ergebnisse sind [Verweis!!!!!!!!!!!!!!].\\

Der Prompt für die Risikotabelle

\begin{prompt}[H]
    \begin{tcolorbox}[colback=gray!20, colframe=gray!20, boxrule=0pt, sharp corners] 
        Erstell mir für diese Projekt eine Risikoliste. Diese soll aus mehreren Spalten bestehen: "ID" für eine 
        eindeutige Identifikationsnummer, "Beschreibung" für eine ausführliche Beschreibung des Risikos und der 
        Auswirkungen, "Datum" für den Zeitpunkt, wann das Risiko identifiziert wurde, "Autor" für die Person die 
        das Risiko gemeldet hat, "Wahrscheinlichkeit (in\%)" für einen Schätzwert der Eintrittswahrscheinlichkeit 
        des Risikos, "Schaden (in €)" für eine Schätzung wie groß der Schaden ist, "Maß (in €)" was das Produkt aus 
        Wahrscheinlichkeit und Schaden ist, "Risikoklasse" für eine Priorisierung der potentiellen Risiken wo 
        zwischen Tolerierbar, Unerwünscht, Kritisch und Katastrophal Unterschieden wird und "Status" wo zwischen 
        aktiv, eingetreten und geschlossen unterschieden wird. Das Risiko ist Tolerierbar wenn das Risikomaß geringer 
        als 0,1\% des Projektvolumen ist, Unerwünscht wenn es größer als 0,1\% ist, Kritisch wenn es größer als 1\% 
        ist und Katastrophal wenn es größer als 10\% ist. In der Risikoliste sollen Team-, Technische-, Methodische-, 
        Kunden-, Fachliche-, Produkt-, Management- und Planungsrisiken betrachtet werden. Diese sollen mit einer 
        leer Zeile getrennt werden, in welchen die Oberbegriffe stehen. Das Projektvolumen beträgt 500000€.
        \vfill
    \end{tcolorbox}
    \caption{Prompt Risikotabelle}
    \label{Prompt Risikotabelle}
\end{prompt}

ist sehr umfangreich formuliert. Damit die Tabelle mit den richtigen Spalten erstellt wird, wurde im Prompt jede
einzelne Spalte aufgezählt und beschrieben. Ebenfalls ist es wichtig, die möglichen Werte für die Risikoklasse zu 
definieren, da hier sonst immer ``Hoch'', ``Mittel'' und ``Niedrig'' von den Tools verwendet wird. Es musste auch 
festgelegt werden, wann die einzelnen Risikoklassen auftreten, da die Zuweisung ansonsten recht schwammig ausfällt und
schwerwiegende Risiken eine geringere Risikoklasse erhalten als eher unwichtige Risiken. Außerdem sollte beschrieben 
werden, welche Arten (Team-, Technische-, Methodische-, Kunden-, Fachliche-, Produkt-, Management- und Planungsrisiken) 
von Risiken betrachtet werden sollen und dass diese Arten in einer leer Zeile stehen, welche die dazugehörigen Risiken von 
den anderen Arten trennen. Ansonsten kann es passieren, dass die Risiken durcheinander geschrieben werden und damit nicht 
den Arten zuzuordnen sind. Damit der Risikoschaden zwischen den Tools vergleichbar ist, sollte das Projektvolumen definiert 
werden. Ansonsten wird auch dies von Tools festgelegt und führt zu Ungenauigkeiten im Vergleich der einzelnen Ausgaben.\\

Grundsätzlich war die Erstellung der Risiken kein Problem. Schwierig war es eher die gewünschte Formatierung zu erhalten, 
sowie eine richtige Zuweisung der restlichen Attribute. Besonders Gemini und Le Chat haben dabei Schwierigkeiten. 
Außerdem war bei allen drei Tools auffällig, dass der Autor immer die Person ist, zu dem das Risiko in den 
Tätigkeitsbereich fällt. Also die Person, die auch der Verantwortliche ist. Fraglich ist dabei teilweise, wenn 
Kundenrisiken vom Kunden, also Herr Müller, erstellt werden, da dieser an der Erstellung der Risikoliste überhaupt 
nicht beteiligt ist.\\

ChatGPT hat die Tabelle sehr gut erstellt. Lediglich die Zuweisung der Risikoklasse war falsch. Nach einem Hinweis 
diesbezüglich wurden diese jedoch, mit einer Ausnahme, richtig korrigiert.\\

Ein Problem bei Gemini ist, dass das Risikomaß nicht richtig berechnet wird und auch die Risikoklassen nicht korrekt 
zugewiesen werden. Das Komma der Werte ist um eine Stelle zu weit links. Es muss alles einmal *10 gerechnet werden, damit die 
Werte stimmen. Kritisch bei Gemini ist, dass die Tabelle nicht in einem Zug erstellt werden kann. Es wird während der 
Generierung einfach aufgehört. Auch wenn man Gemini fragt, ob die vollständige Tabelle generiert werden kann, wird mitten 
drin aufgehört. Man muss explizit nur nach den noch offenen Risikoarten fragen. Diese werden dann erstellt, jedoch soll man 
den Schaden und das Maß selbst eintragen und berechnen. Nach anschließender Frage, ob er den Risikoschaden und das Schadensmaß 
festlegen kann, sagt er, dass er detaillierte Informationen über das Projekt und die potenziellen Risiken benötigt um genaue Werte 
festlegen zu können.\\

Le Chat hatte ein ähnliches Problem wie bereits im Besprechungsprotokoll [\autoref{CompBesprechungsprotokoll}], dass das 
Datum immer auf den 01.04.2023 gesetzt wird. Außerdem hatte Le Chat Probleme damit, die Tabelle in das gewünschte Format
zu bringen, dass die Risikoarten in einer Zeile stehen und die dazugehörigen Risiken darunter aufgelistet werden. Sie wurden 
lediglich mit 1.x bis 8.x beschriftet wodurch Sie sich zu den Arten zuweisen ließen. Auch 
hier wurden die Risikoklassen falsch bestimmt und auch nach einem Hinweis wurden diese nicht korrigiert. Jedoch wurde 
dabei der Risikoschaden der einzelnen Risiken geändert. Auch hier ist die Zuweisung der Risikoklassen nicht richtig.\\

Die Maßnahmentabelle wurde mit folgendem Prompt erstellt:

\begin{prompt}[H]
    \begin{tcolorbox}[colback=gray!20, colframe=gray!20, boxrule=0pt, sharp corners] 
        Erstelle mir nun eine dazu passende Maßnahmentabelle. Auch diese besteht aus mehreren Spalten: "Typ" beschreibt 
        ob die Maßnahme das Risiko verhindert, lindert oder überträgt. In "Beschreibung" wird die Maßnahme beschrieben. 
        Falls ein Risiko als nicht mehr relevant eingestuft wird, wird in der Spalte "Beschreibung" eine Begründung 
        eingetragen und die Maßnahme auf "beendet" gesetzt. "Trigger" ist das Ereignis, das den Start der Maßnahme 
        veranlasst, falls diese nicht sofort eingeleitet werden soll. "Verantwortlicher" ist die zuständige Person für 
        die Durchführung der Maßnahme. "Status" unterscheidet zwischen geplant, aktiv und beendet. Anschließend gibt es 
        jeweils eine Spalte für "Restwahrscheinlichkeit (in \%)", "Restschaden (in €)", "Restmaß (in €)" und "Restklasse" 
        was die geschätzte Wahrscheinlichkeit, geschätzter Schaden, Maß und Klasse des Restrisikos, nach Durchführung der 
        Maßnahme entsprechen. Für jedes Risiko sollen zwei Maßnahmen erstellt werden. Dazu wird über die zwei Maßnahmen 
        eine Zeile mit der ID von dem Risikon beschrieben, auf die sich die Maßnahmen beziehen.
        \vfill
    \end{tcolorbox}
    \caption{Prompt Maßnahmentabelle}
    \label{Prompt Maßnahmentabelle}
\end{prompt}

Der Prompt für die Maßnahmen ist ähnlich wie der, für die Risikotabelle. Es wird jede Spalte der Tabelle einmal beschrieben,
damit das Format der Tabelle mit den Vorgaben übereinstimmt. Anschließend wird gesagt, dass die ID des dazugehörigen Risikos
mit in die Tabelle übernommen werden soll, damit man die Maßnahmen den Risiken zuordnen kann. Die Probleme sind hier 
ähnlich zu denen bei der Risikotabelle, jedoch erstellen alle drei Tools grundsätzlich anständige Maßnahmen für die Risiken.
Bei allen dreien fühlt sich allerdings die Zuweisung der Attribute zu den Maßnahmen repetitiv an. Häufig wechseln sich die 
auswählbaren Parameter ab und auch die Wahrscheinlichkeiten und der Restschaden sind meistens immer wieder die gleichen Zahlen.\\

ChatGPT erstellt auch die Maßnahmentabelle sehr gut. Lediglich die Zuweisung der Restklasse stimmt nicht überein.\\

Bei Gemini wird für die Begründung, warum ein Risiko als nicht mehr relevant eingestuft wird, eine eigene Spalte erstellt.
Auch wenn man die Beschreibung dafür im Prompt ändert, wird die Spalte erstellt. Außerdem ist die Formatierung bei Gemini
manchmal nicht so gut, da bei jeder 2. Zeile alle Daten ab der Spalte "Begründung" um eine Stelle nach links verschoben wird.
Lässt man sich die Antwort neu generieren und auch wenn man Gemini sagt er soll die Formatierung korrigieren bleibt das 
Problem bestehen. Ein weiteres Problem ist, dass die Spalte "Trigger" nicht gefüllt wird und auch das Problem, dass
die Tabelle nicht ganz vollständig generiert wird sondern mitten drin aufhört, tritt wieder auf. Teilweise ist das Restmaß 
falsch berechnet und auch die Zuweisung der Restklasse ist nicht immer richtig.\\

Besonders aufgefallen bei Le Chat ist, dass die Attribute für die Maßnahmen sehr repetitiv festgelegt wurden. Die Restwahrscheinlichkeit
beträgt für jede Maßnahme 5\% und auch der Restschaden ist häufig für die Maßnahmen für eine Risikoart gleich. Auch hier ist die 
Restklasse nicht entsprechend zugewiesen worden sondern ist, bis auf bei der ersten Maßnahme, auf Unerwünscht festgelegt. Auch 
die Festlegung des Typs der Maßnahmen wirkt sehr repetitiv, da hier immer zwischen ``verhindern'' und ``lindern'' gewechselt wird.\\

Zusammenfassend lässt sich sagen, das die Erstellung von Risiken und dazu gehörigen Maßnahmen gut funktioniert, allerdings die 
Festlegung der zusätzlichen Attribute nicht den Vorgaben entspricht und häufig zu Fehlern führt. Auch die richtige Formatierung 
der Tabelle ist häufig etwas schwierig, lässt sich aber in den meisten fällen korrigieren. Da ChatGPT deutlich weniger Probleme 
bei der Erstellung hatte und auch die Zuweisung der Attribute, abgesehen von der Risikoklasse, und die Formatierung gut funktioniert
hat wird es mit ``Gut'' bewertet.\\
Le Chat hatte etwas mehr Probleme gemacht. Bei der Risikotabelle war die Formatierung insofern kein Problem, da man die einzelnen 
Spalten einfach in eine eigene Tabelle übernehmen konnte und die Risikoarten selbst dazwischen schreibt. Erst bei der Maßnahmentabelle
hat LeChat Probleme mit der Zuweisung der Attribute bekommen. Daher wird Le Chat mit ``Befriedigend'' bewertet.\\
Gemini hat am meisten Probleme gemacht. Alleine schon die Tatsache, dass Gemini mitten in der Generierung diese als fertig betrachtet 
und dadurch die Tabellen unvollständig erstellt werden, sorgt dafür, dass Gemini für solche Erstellungen eher ungeeignet ist. Die 
anderen Probleme wie die fehlerhafte Formatierung und das die Attribute falsch berechnet und festgelegt werden kommen dabei noch hinzu.
Dies sorgt dafür, dass Gemini hier die Bewertung ``Mangelhaft'' erhält.\\

Wenn man allerdings sich nur die Erstellung von Risiken mit dazugehörigen Maßnahmen anschaut, erhalten alle drei Tools die Bewertung 
``Sehr gut''. Die erstellten Risiken und Maßnahmen waren zwar alle eher welche, die man vermutlich als erstes benennt und auch 
Projekt unabhängig sind, wurden aber dennoch sehr gut erstellt. Auch bei Gemini werden einzelne Risiken mit dazugehörigen Maßnahmen 
in einem rutsch erstellt, weshalb man dieses Problem bei der Erstellung der Risiken und Maßnahmen nicht ankreiden kann.


\section{Anforderungsspezifikation} \index{Anforderungsspezifikation} \label{CompAnforderungsspezifikation}

Für die Anforderungsspezifikation wurde für jeden Inhalt ein eigener Prompt erstellt. Daher wird im folgenden die Erstellung der Einleitung, 
der Systemarchitektur, des Fachliche Datenmodells, des Anwendungsfalldiagramms, der detaillierten Anwendungsfall Beschreibung mit Aktivitätsdiagramm, 
der Benutzungsschnittstelle mit der Dialog Navigation, der detaillierten Ausarbeitung von Dialogen und der nichtfunktionalen Anforderungen einzelnen
Beschrieben und anschließend bewertet.\\

\subsection*{Einleitung}

Für die Einleitung wurde der folgende Prompt verwendet:

\begin{prompt}[H]
    \begin{tcolorbox}[colback=gray!20, colframe=gray!20, boxrule=0pt, sharp corners] 
        Erstell mir für die Anforderungsspezifikation eine Einleitung. Diese soll kurz, mit 2-3 Sätzen, in das Dokument einführen. Anschließend soll 
        ein Abschnitt zum "Zweck und Umfang des Dokuments" kommen, welcher eine Beschreibung der Notwendigkeit des Systems für den Kunden sowie eine 
        Kurzbeschreibung der Funktionalität und der Nachbarsysteme beinhaltet. Danach wichtige "Begriffe und Abkürzungen" erklärt werden. Zum Schluss 
        kommt ein Abschnitt "Verweise auf sonstige Ressourcen", wo auf die Spielregeln verwiesen wird.
        \vfill
    \end{tcolorbox}
    \caption{Prompt Einleitung Anforderungsspezifikation}
    \label{Prompt Einleitung Anforderungsspezifikation}
\end{prompt}

Bei der Erstellung des Prompts musste darauf geachtet werden, den Abschnitt ``Zweck und Umfang des Dokuments'' genauer zu beschreiben, da hier ansonsten
die generierte Ausgabe sehr ausschweifend und unspezifisch erstellt werden. Grundsätzlich hat die Erstellung gut geklappt.\\
Bei ChatGPT ist lediglich auffällig, dass er beim Abschnitt ``Zweck und Umfang des Dokuments'' nicht den Zweck der Anwendung beschreibt, sondern den Zweck 
des Dokuments. Die Kurzbeschreibung der Funktionalität wurde dafür gut erstellt. Bei der Kurzbeschreibung der Nachbarsysteme wurden die einzelnen Teile 
der Anwendungs erläutert, also das Spielsystem, das Kommunikationsprotokoll für RMI und die Benutzeroberfläche, und keine anderweitigen System, da die 
Anwendungs alleinstehend ist.\\
Bei Gemini [[[[[[[[[[[[[[[[[]]]]]]]]]]]]]]]]]\\
Le Chat hat die Beschreibung gut erstellt. Der einzige Aspekt der auffällig war ist, dass beim Abschnitt ``Verweis auf sonstige Ressourcen'' ein Link 
auf eine Webseite für die Regeln von Mensch ärgere dich nicht eingefügt wird.\\

Bewertet wird ChatGPT mit ``Gut'', da hier der Zweck des Dokuments beschrieben wurde, obwohl im Prompt explizit nach einer Beschreibung der Notwendigkeit
des Systems für den Kunden gefragt wurde.\\
Le Chat wird mit ``Sehr gut'' benotet, da hier keine Inhatlichen Fehler vorhanden sind und soweit alles gepasst hat. 

\subsection*{Systemarchitektur}

Die Systemarchitektur wurde mit folgenden Prompt erstellt:

\begin{prompt}[H]
    \begin{tcolorbox}[colback=gray!20, colframe=gray!20, boxrule=0pt, sharp corners] 
        Erstell mir für die Anforderungsspezifikation die Systemarchitektur. Diese beinhaltet einen groben Überblick über die erwartete Systemarchitektur 
        und Einordnung des Systems in die Systemlandschaft des Kunden. Falls Schnittstellen zu Nachbarsystemen bestehen, dann müssen diese abgebildet werden.
        \vfill
    \end{tcolorbox}
    \caption{Prompt Systemarchitektur}
    \label{Prompt Systemarchitektur}
\end{prompt}

Auch hier muss beschrieben werden, was genau erstellt werden soll. Wichtig hier ist, dass man in den selben Chat schreibt, in dem man die Dokumente vorher
schon erstellt hat. Dadurch wissen die Tools was zu erstellen ist.\\
Bei ChatGPT wurde sogar ein Diagramm in den Chat ``gezeichnet''. Die Beschreibung und das Diagramm ist jedoch für die Systemarchitektur schon viel zu genau.
Es hätte gereicht den Client, den Server und die Datenbank darzustellen und diese richtig miteinander zu verbinden. Bei der ersten Erstellung wurden 
die Verbindungen zwischen Client und Server und Server und Datenbank unidirektional dargstellt. Nach einer weiteren Eingabe, ob das so gewollt ist, wurde 
dies jedoch korrigiert und die Verbindung wurde bidirektional eingezeichnet.\\
Gemini[[[[[[[[[[[[[[[[[[[[[[[[[]]]]]]]]]]]]]]]]]]]]]]]]]\\
Nach der Eingabe kam bei Le Chat eine kurze Textuelle Beschreibung der Client- und Server-Anwendung sowie der Datenbank. Außerdem wurde beschrieben, 
dass Schnittstellen zu Nachbarsystemen ebenfalls mit aufgenommen werden sollen. Diese Beschreibungen haben soweit gut gepasst. Auf nachfrage, wie denn 
dann die Abbildung aussehen soll, wurde beschrieben, dass der Client, der Server und die Datenbank jeweils eigenständige Blöcke sein sollen. Dies ist soweit
richtig. Es wurde jedoch nicht genau Beschrieben wie die Beziehungen zwischen den Komponenten aussehen. Daher wurde nochmal eine Eingabe getätigt, wie denn 
die Beziehung zwischen den Komponenten aussieht. Hier wurde richtig dargestellt, dass der Client und der Server kommunizieren und der Server mit der Datenbank, 
jedoch nicht der Client mit der Datenbank. Die restliche Beschreibung, wie diese Komponenten kommunizieren, ist wieder etwas zu ausführlich. Die Beschreibung 
reicht jedoch, um die Skizze selber zu erstellen, was kein negatives Kriterium ist, da bei den kostenfreien Versionen nicht davon ausgegangen wird, dass diese 
das können.\\

Zusammenfassend lässt sich hier sagen das alle [zwei???] Tools die Systemarchitektur etwas zu genau erzeugen. Jedoch lassen sich die richtigen Inhalte 
aus der Ausgabe kopieren und die zu detaillierten Beschreibungen können einfach rausgelöscht werden, weshalb die Tools mit der Note ``Gut'' bewertet werden.

\subsection*{Fachliches Datenmodell}

Das erste größere und anspruchsvollere Dokument ist das Fachliche Datenmodell. Dieses wurde mit folgendem Prompt erstellt:

\begin{prompt}[H]
    \begin{tcolorbox}[colback=gray!20, colframe=gray!20, boxrule=0pt, sharp corners] 
        Erstell mir nun für die Anforderungsspezifikation das Fachliche Datenmodell. Dieses soll mit Hilfe eines UML Klassendiagrammes mit ergänzenden Beschreibungen 
        bzw. EInschränkungen spezifiziert werden. Das Modell soll alle Entitätstypen mit deren Eigenschaften, Beziehungen und Einschränkungen besitzen.
        \vfill
    \end{tcolorbox}
    \caption{Prompt Fachliches Datenmodell}
    \label{Prompt Fachliches Datenmodell}
\end{prompt}

Dabei muss explizit erwähnt werden, dass ein UML-Klassendiagramm gewünscht ist, damit die Aussgabe das richtige Format besitzt. Außerdem sollte erwähnt werden,
dass ergänzende Beschreibungen mit erstellt werden soll, damit diese Vorgabe auch direkt mit erfüllt ist und die Tools ihr Diagramm einmal näher erläutern.\\

Bei ChatGPT war direkt auffällig, dass für jede Entität eine ID erstellt wurde. Dies führt zum einen dazu, dass die Anwendung sehr Datenbanklastig ist und zum 
anderen ist dies auch einfach bei manchen Entitäten nicht nötig. Zum Beispiel beim User könnte man, je nach logIn-Daten, den Namen oder die E-Mail-Adresse 
einzigartig machen und damit den User identifizieren. Des weiteren haben die Beziehungen nicht immer gepasst. Zum Beispiel die 1:1 Beziehung zwischen 
``Spielfigur'' und ``Feld''. Es muss natürlich nicht auf jedem Feld eine Spielfigur stehen sondern auf einem Feld kann (0..1) eine Spielfigur stehen. Dieses Problem
zieht sich auch durch die gesamten Versuche, das Fachliche Datenmodell zu korrigieren. Häufig passiert es auch, dass wenn ein anderer Fehler korrigiert wird, 
eine richtige Beziehung zu etwas falschem geändert wird. Bei der erstellten Skizze waren ebenfalls die Beziehungen falsch eingezeichnet. Außerdem zeigten Beide Pfeile 
zwischen zwei Entitäten in die gleiche Richtung. Bei Nachfrage, ob die Beziehung nicht in beide richtungen zeigen sollten, wurde gesagt, dass das stimmt und die 
Beziehungen Bidirektional sein sollten, jedoch wurde an der Zeichnung nichts geändert. Desweiteren ist aufgefallen, dass bei der Entität ``Feld'' ein Attribut gefehlt
hat, das beschreibt, ob das Feld belegt ist oder nicht. Außerdem wurde in einem String in ``Spiel'' das Spielbrett gespeichert. Auf Nachfrage wie dies Implementiert
werden soll, wurde geantwortet das dies nicht die beste Lösung ist, sondern das Spielbrett dynamisch in der Anwendung generiert und angezeigt werden sollte.
Weitere Probleme hatte ChatGPT mit der Speicherung der Farbe und der Position. Manchmal wurden diese Attribute in ``Spieler'', manchmal in ``Spielfigur'' und teilweise
auch in beiden gespeichert. Damit ese später einfacher zu realisieren ist an mehreren Spielen teilnehmen zu können, wurde die Eingabe, dass es mehr Sinn macht den 
``Spieler'' in ``User'' und ``Spieler'' aufzuteilen, getätigt. ChatGPT hat dies dann genau wie beschrieben gemacht. Desweiteren wurde der Würfel vergessen, welcher 
jedoch nach Hinweis diesbezüglich in ``Spiel'' hinzugefügt wurde. Nach der letzten Ausgabe des Fachlichen Datenmodells wurde, nach Aufforderung, ein PlantUML-Diagramm 
erstellt. In diesem wurde an den Assoziationen Leseunterstützung angefügt. Diese Leseunterstützung sind jedoch nur von oben nach unten hilfreich. Zum Beispiel ``Feld 
steht auf Spielfigur'' ist keine Hilfreiche Beschriftung. Des weiteren sind alle Festlegungen der Vielfachheit auf der falschen Seite. Als Beispiel muss an der 
Assoziation zwischen ``Spieler'' und ``Spielfigur'' die ``1'' auf der Seite von ``Spieler'' sein und die ``1..4'' auf der Seite von ``Spielfigur''. Desweiteren würden 
Pfeile an den Assoziationen helfen, die Flussrichtung zu verdeutlichen.\\

Gemini[[[[[[[[[]]]]]]]]]\\

Auch Le Chat hatte zu beginn nur eine Entität ``Spieler''. Für eine Anwendung wo ein Spieler an mehreren Spielen teilnehmen sollen kann, macht es jedoch Sinn, diese 
Entität auf zwei Entitäten, ``User'' und ``Player'', aufzuteilen. Desweiteren wurde bei Le Chat, wie es eigentlich in einem UML-Klassendiagramm gehört, die Typen der
Attribute nicht festgelegt. Außerdem ist die Frage, wenn die E-Mail-Adresse eines Spieler eindeutig sein muss, warum dann trotzdem eine eindeutige ID benötigt wird.
Daneben wurde auch die Anforderung des Kunden, das ein Punktesystem eingebunden werden soll, ignoriert, da es kein Attribut zum Speichern der Punktzahl gibt.
Die Bezeichung der Beziehungen zwischen den Entitäten ist, in der Textuellen Beschreibung für die Beziehung, eher an einem ER-Modell orientiert und nicht an einem 
UML-Klassendiagramm. Die genaue Bezeichnung für die Vielfachheit muss aus der Beschreibung selbst erstellt werden. Diese war jedoch auch teilweise fehlerhaft. Zum 
Beispiel, dass ein Player an einem oder mehreren Spielen teilnehmen kann, macht kein Sinn, da die Entität ``Player'' die Eigenschaften eines Spielers
im Spiel beschreibt. Diese kann schlecht in mehreren Spielen gleich sein. Auch einfachere Sachen wie die Zuweisung der Spielfiguren war ein Problem. Zum Beispiel bei 
der Beschreibung der Beziehung zwischen Player und Spielfigur wurde definiert, dass ein Player eine oder mehrere Spielfiguren besitzen kann. Dies entspricht aber nicht 
den normalen Regeln von Mensch ärgere dich nicht, wo jeder Spieler immer genau vier Spielfiguren besitzt. Daneben wurde auch hier der ``Player''-Entität ein Attribut
für die Position auf dem Spielbrett festgelegt. Nach der Eingabe [Verweis!!!!] wurden genau diese Beziehungen richtig geändert, jedoch waren immer noch Fehler enthalten
wie zum Beispiel, dass ein Spiel eine oder mehrere Spielfiguren enthalten kann. Anschließend wurde gefragt ob der Würfel auch in das Fachliche Datenmodell mit aufgenommen
werden sollte, woraufhin sowohl eine eigene Entität ``Dice'' als auch ein Attribut ``dice: Dice'' in der ``Game''-Entität erstellt wurde. Anschließend wurde gefragt ob 
das Fachliche Datenmodell in PlantUML erstellt werden kann. Hier fällt besonders auf, dass die Verbindungen zwischen den Entitäten alles Kompositionen sind, was aber soweit
auch Sinn ergibt. Außerdem fällt auch das hier bereits die Primär- und Fremdschlüssel dargestellt werden. Dies ist für das Fachliche Datenmodell allerdings zu spezifisch.
Ein weiterer Fehler ist die Beziehung zwischen ``Game'' und ``Figure''. Hier wird festgelegt, dass ein Spiel aus 16 Figuren besteht. Allerdings gibt es auch ein 
Attribut ``max\_players'' und es wurde bereits festgelegt, dass ein Spiel aus 1..4 Spieler besteht. Daher kann es auch sein, falls nicht immer auf 4 Spieler mit KI-Gegner aufgefüllt 
wird, das weniger als 16 Figuren auf dem Spielfeld sind. Hier lässt sich allerdings eh kritisch hinterfragen ob die Komposition zwischen ``Game'' und ``Figure'' überhaupt 
notwendig ist, da durch die Verbindung zwischen ``Figure'' und ``Player'' und der zwischen ``Game'' und ``Player'' die Figuren sowieso genau einem Spiel zugeordnet sind.\\

Alle drei Tools schaffen es, ein grobes Grundgerüst für das Fachliche Datenmodell zu erstellen.


- Le Chat:
    - Nur beschreibung
    - Hat auch nur Spieler, "user" und "player" würden sinn machen.
    - Spielfigur an einem oder mehreren Spielen teilnehmen?
    - Spieler-Spielfigur: Ein Spieler hat genau 4 Figuren, nicht weniger und nicht mehr
    - Beziehung zwischen Spiel-Spielfigur? wenn ja, Spieler * 4 Spielfiguren.
    - Punkte werde nicht gespeichert.
    - Player hat position? Player kann an mehreren Spielen teilnehmen? -> Nach Hinweis geändert
    - Player und Spielfigur haben positionen.
    - Spiel-Spielfigur Beziehung sinnvoll?

    - Probleme mit der Beschreibung der Beziehungen zwischen den Entitäten
    - Ein paar grundsätzliche Einschränkungen aufgelistet, aber nicht zu jedem Attribut -> Sagt nur ob Ganzzahl oder Text,
    keine Begrenzung in der Länge.

    Prompts:
        - Erstell mir nun für die Anforderungsspezifikation das Fachliche Datenmodell. Dieses soll mit Hilfe eines UML Klassendiagrammes mit ergänzenden Beschreibungen bzw. EInschränkungen spezifiziert werden. Das Modell soll alle Entitätstypen mit deren Eigenschaften, Beziehungen und Einschränkungen besitzen.
        - Macht es nicht auch sinn "Spieler" in zwei Entitäten aufzuteilen? Einmal "User" und einmal ``Player''.  "User" beinhaltet dabei den Namen, die Email, das Passwort und die Punkte. "Spieler" beinhaltet dann die nötigen Attribute zum Spielen wie SpielerID, die Farbe und die position seiner Spielfiguren.
        - Ein Player hat 4 Figuren. Außerdem hat der Player keine Positionen, diese haben ja die Figuren. Ein Player soll auch nicht an mehreren Spielen teilnehmen, sondern der User kann mehrere Player sein und ein Player spielt ein Spiel. Ein Spiel besteht aus 1 bis 4 Playern.


Anwendungsfalldiagramm:

- Prompt: Reicht zu sagen, dass das Diagramm alle Anwendungsfälle beinhalten soll und den Rollen zugewiesen werden soll.

- ChatGPT:
    - Spiel erstellen: wird zum Gastgeber. Besondere rechte? wenn ja, warum keine neue Rolle. -> Nach Hinweis neue Rolle
    - Als User Chat nutzen? In Anforderungen wurde nur von Chat während des Spiels gesprochen. -> Nach Hinweis wurde entfernt.
    - Was gehört zu Zug machen, wenn Spielfigur bewegen eigener Anwendungsfall ist -> gesamte Ablauf wenn Spieler am Zug ist
    (Würfeln, Auswählen und Bewegen der Spielfigur, Schlagen gegnerischer Figuren und Beenden des Zuges), AF "Spielfigur bewegen" 
    war doppelt. Wurde aber nach Hinweis entfernt. Sehr großer Anwendungsfall, vielleicht lieber in kleinere AF aufteilen??
    - Erstellt Code um in einem PlantUML-Editor oder einem Online Renderer das Diagramm als Bild zu erstellen
    - Erstellt Skizze in PlantUML ganz gut, sieht lediglich aus als ob auch ``Spieler'' von ``User'' erbt.
    -> Grundgerüst wird ganz gut erstellt, muss aber mit ein paar nachfragen in richtige Richtung lenken.

- Gemini: [[[[[[[[[[[[[[[[[[[[[[[[[[[[[[[[[[[[[[[[[[[[[]]]]]]]]]]]]]]]]]]]]]]]]]]]]]]]]]]]]]]]]]]]]]

- Le Chat:
    - Rollen werden gut erstellt.
    - Passende Anwendungsfälle, stimmen auch zu den Rollen
    - "Benutzer verwalten" von Admin vielleicht etwas ungenau? -> Wird geändert und entsprechend ergänzt
    - möglich mit PlantUML erstellen zu lassen, Diagramm jedoch recht durcheinander und Anwendungsfall ``Passwort vergessen'' wurde ``User'' 
    zugewiesen und nicht wie in der Beschreibung ``Gast''


AF:

- Prompt: Beschreibung, welcher Anwendunsfall beschrieben werden soll, welche Informationen zunächst erstellt werden soll und anschließend das ein 
Aktivitätsdiagramm erstellt werden soll. Beschrieben das die Aktivitäten dem Server oder dem Client zugewiesen werden sollen, um die spätere 
implementierung der Server-Client-Applikation einfacher fällt. Außerdem erwähnt das keine Dialog eigenschaft beschrieben werden sollen.

- ChatGPT-3.5:
    - Häufigkeit keine Zahl: "Häufig verwendet (mehrmals täglich)
    - Erstellt diagramm direkt mit, kein Endzustand. Verbindungen führen nach ``Anmeldung erfolgreich'' und ``Anmeldung fehlgeschlagen'' wieder zusammen.
    -> Auch nach Allgemeiner Beschreibung nicht korrigiert und auch nicht bei sehr genauer Beschreibung.
    - Beschreibungen gut.
    - ChatGPT-4o macht zu viel. Implementierung wird bisschen mit beschrieben (Benutzersession erstellen) sowie Dialog 
    eigenschaften wie Anzeigen von Fehlermeldungen werden beschrieben. GPT-3.5 hier tatsächlich besser. Außerdem wurde Code
    von PlantUML falsch erstellt (Syntax error)


- Gemini: [[[[[[[[[[[[[[[[[[[[[[[[[[[[[[[[[[[[[[[[[[[[[[[]]]]]]]]]]]]]]]]]]]]]]]]]]]]]]]]]]]]]]]]]]]]]]]

- Le Chat:
    - Beschreibt für die Endzustände trotz Hinweis, das keine Dialog eigenschaften beschrieben werden sollen, was der Dialog anzeigt.
    - Häufigkeit keine Zahl
    - Kann auf Nachfrage Code für PlantUML erstellen.
    - Diagramm hat nur einen Endzustand, wo Verbindung von ``Anmeldung erfolgreich'' und ``Anmeldung fehlgeschlagen'' enden.
    - Diagramm häufig Syntax Error, welcher sich nicht beheben lässt. -> Dafür sehr gute textuelle Beschreibung, auch richtig mit 
    Endpunkten und dem Entscheidungsknoten.


    Zug machen:

        ChatGPT:
            - Aktivität 6, berechne mögliche Züge. -> Nicht grade effizient?
            - Server würfelt alleine, kein gutes Spielerlebnis
            - Generell keine wirklche Verweise aufs FDM
            - Entität Spielzug aus FDM wird nicht betrachtet.
            - Entscheidungsknoten laufen im UML Diagramm immer wieder zusammen
            - Aber ansich ablauf der Aktivitäten ergibt Sinn 
            - Hilft, das FDM ihm nochmal zum merken zu geben. Beschreibung der Aktivität geht dann bisschen mehr aufs FDM ein
            - Erstellt aufeinmal ``Spieler'' und ``Server'' Spalten. -> Wird zu ``Client'' und ``Server'' geändert
            - Hat wieder Dialog Aktivitäten mit drinne, Bsp.: "Ergebnis des Würfelwurfs anzeigen"
            - Aktivitäten schon etwas zu genau formuliert?
            - Entscheidungsknoten laufen auch hier wieder zusammen -> Wird geändert
            - Nur ein Endknoten -> Wird geändert
            - Nach Frage ob er Aktivitätenbeschreibung etwas genauer Formulieren kann, wird zwei mal in ``Spielfigur.position''
            gespeichert. Einmal in ``Bewege Spielfigur'' und einmal in ``Speichere aktuelle Position der Spielfigur in Spielfigur.position''
                -> wird geändert
            - Spiel.status soll aktualisiert werden, wird aber nix genaues Beschrieben was geprüft wird
            - Feld.belegt wird nicht angepasst
            -


        LeChat:
            - Formuliert aktivitäten erst sehr grob. Sagt nicht was wie gespeichert wird.
            - Benutzt andere Attribute als im FDM festgelegt. Bsp. Feld.x und Feld.y
            - Entspricht nicht den normalen Regeln? "...die geschlagene Figur aus der Liste der Figuren des Spielers entfernt"
            - Spielstand aktualisieren als eigenen AF? Wurde vorher nicht aufgeführt
            - Führt Würfeln als Aktivität auf? Ist aber doch eigener Anwendungsfall
            - Auf nachfrage, das aktuelle FDM auszugeben, wird ein völlig anderes erstellt. Auch wenn ich 
            das zuletzt erstellte als Eingabe verwende, ändert er dies geringfügig.



Benutzungsschnittstelle:

- Prompt:

- ChatGPT:
    - Man startet im Hauptmenü und sieht alle Buttons. ``Spiel erstellen'' und ``Spiel beitreten'' können jedoch nur 
    genutzt werden, wenn man Angemeldet ist.
    - Erstellt Skizze und Textuelle Beschreibung. Skizze passt aber nicht ganz zu Beschreibung. Auch nach Hinweis wird 
    selber Fehler nochmal gemacht. Wenn man auf die wiederholt Falsche Antwort negativ Reagiert mit dem Feedback, dass 
    die Hinweise nicht richtig umgesetzt wurden, und dann im als Prompt die Eingabe ``Die Skizze ist unverändert. 
    Setzt den "Spiel beitreten"-Dialog mit dem Hauptmenü in Verbindung.'' gibt, wird die Skizze richtig erstellt.
    - Es wird auch beschrieben, welche Buttons und Eingabefelder die einzelnen Dialoge besitzen. Dies ist jedoch 
    an dieser Stelle nicht nötig.
    - Skizze kann auch nicht zu einem Dialog zurück gehen. Wird dann nochmal aufgeführt (ohne Buttons,...)


- Gemini:

- Le Chat:
    - Erstellt für jeden Anwendungsfall einen Dialog, grundsätzlich gut.
    - Spielbrett benötigt auch ein Dialog?
    - Verbindungen zwischen den Dialogen macht keinen Sinn -> Nach Hinweis wird erneut ausgegeben, Beschreibung besser
    aber Skizze immer noch unsinnig.
    - Beschreibung beinhaltet aber nicht alle Anwendungsfälle (Admin wird ausgelassen). Außerdem werden Dialoge erstellt,
    für die es keine Anwendungsfälle gibt (z.B. "Profil bearbeiten" um Benutzerdaten selber zu ändern)
    - Auch nach genauer Beschreibung wird Skizze nicht richtig erstellt.
    


Dialog:

- Prompt: Beschreibung des Inhalts reicht.

- ChatGPT:
    - Erstellt Skizze
    - beschreibt Validierung. Richtigen Stelle dafür?

- Gemini:

- Le Chat: 
    - kleiner Formulierungsfehler bei ``Abbrechen''-Button. "Nutzer bleibt angemeldet. -> Wird bei Hinweis korrigiert
    - Erklärt Skizze gut bei Nachfrage.


Nichtfunktionale Anforderungen:

- Prompt: 

- ChatGPT:
    - passt

- Gemini:

- Le Chat: 
    - passt