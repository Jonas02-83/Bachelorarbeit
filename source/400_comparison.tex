
\chapter{Praxisergebnisse und Vergleich} \index{Praxisergebnisse und Vergleich}

TODO!!!!!!!!

Generell:
    - Schwierigkeiten in einem gorßen/langen Chat, die Tools von einer Formulierung weg zu bringen. 

\section{Besprechungsprotokoll} \index{Besprechungsprotokoll} \label{CompBesprechungsprotokoll}

%Angefangener Text in Rest.tex

- [Prompt] und Gespräch als Eingabe
- Teilweise sehr unterschiedliche Ausgaben obwohl selbe Prompts. Auch in neuen Chats bei gleichem Tool.
Einzelne Einträge mal zu einem Eintrag zusammengefasst, mal in mehreren geschrieben. (Anz. immer unterschiedlich).
Beispiel Server-Client, JavaFX und RMI.
- genaue Beschreibung der Spalten wird benötigt, da sonst eher eine Stichpunkteliste erstellt wird
- Einzelnen Arten müssen beschrieben werden, da diese sonst sehr ungenau festgelegt werden. Aber auch mit
Beschreibung der Arten ein Problem. Tools machen das unterschiedlich und auch in neuen Chats wird die Zuweisung immer 
bisschen unterschiedlich gemacht. Obwohl gleiche Eingabe.
- Im Prompt muss stehen das "Termin" ein genaues Datum benötigt. Sonst wird z.B. mit "+2 Wochen" gearbeitet. Le Chat 
macht das dann trotzdem nicht, da muss als neue Eingabe "Der Termin soll ein genaues Datum sein" geschrieben werden.
Datum dann allerdings in der Vergangenheit (z.B. 31.10.2023)?? Schreibt aber das die Termine beispielhaft gewählt sind
und sich am aktuellen Datum, 29.03.2023??, gewählt sind.
- Gemini hat großes Problem nur die wichtigsten Punkte aus dem Gespräch rauszulesen. Häufig Besprechungsprotokolle mit etwa 
30 Punkten. Dabei erstellt er als neue Art "Frage" und füllt diese z.B. mit "Maximale Spieleranzahl" um anschließend eine 
Information aufzulisten mit der Beschreibung "Maximale Spieleranzahl: 4". Wenn selben Prompt in neuem Chat immer wieder 
eingegeben wird, kommt manchmal eine "kurze" Version, welche immernoch länger ist als die Protokolle der anderen beiden Tools.
Gemini schreibt sehr kleinkariert und mach für jede einzelne Information einen neuen Punkt. Beispiel Server-Client, JavaFX und RMI.
- Auch zusätzlicher Hinweis im Prompt, nur die wichtigen Punkte in der Tabelle aufzunehmen, ändert bei Gemini nix daran auszuschweifen.
- Fazit?? Selber drüber lesen und schauen ob das passt. Vorallem auf "Art" gucken und ob der Punkt wirklich wichtig ist. Le Chat
hält sich eher zu knapp mit den Punkten, daher hier schauen ob alle wichtigen Informationen enthalten sind.

\section{Projekthandbuch} \index{Projekthandbuch} \label{CompProjekthandbuch}

- Zwei eingaben. Einmal für Einleitung mit dem Zweck des Dokuments, der Redaktion und dem Verteiler. Und einmal für die 
Projektdefinition mit ausschließlich den Kapiteln Vorgeschichte und Inhaltliche Kurzdarstellung.
- Muss genau Beschrieben werden was man für Abschnitte haben will, sonst wird ein eigenes Projekthandbuch erstellt, welches 
nicht ganz den Anforderungen, welche in dieser Arbeit beschrieben wurde, entspricht.
- Für die Einleitung werden spezifische Informationen benötigt, welche im Prompt angegeben werden müssen. In diesem Fall wurde 
angegeben, dass die Projektleiterinfür das Projekt verantwortlich ist und als Verteiler ein Discord-Channel verwendet wird.
- Abschnitte wurden gut erstellt. ChatGPT bennent im Abschnitt für den Verteiler einmal alle Teammitglieder, frage ob unbedingt nötig.
- Zweite Abschnitt benötigt keine weiteren Informationen, da diese im Gespräch bereits vorgegeben sind. 
- Abschnitte wurden hier auch gut erstellt. Die benötigten Informationen haben sich die Tools gut aus dem Gespräch raus gezogen.
- Gemini hat teilweise dreher in den Namen drinne. Einmal hat Gemini geschrieben, dass Frau Schmidt (Teamcheffin) mit dem Herrn Müller 
(Kunde) aufnahm. -> muss Probegelesen werden.

\section{Risikoliste} \index{Risikoliste} \label{CompRisikoliste}

- Auch hier zwei einzelne Eingaben. Einmal für die Tabelle mit den Risiken und einmal für die mit den Maßnahmen.
- Prompt für Risikotabelle sehr umfangreich. Jede Spalte erklärt damit die Tabelle, die erstellt wird, den Vorgaben entspricht. 
Auch das Risiken jeweils zu den Oberbegriffen Team-, Technische-, Methodische-, Kunden-, Fachliche-, Produkt-, Management- und 
Planungsrisiken erstellt werden sollen und die Risiken dazu mit einer leer Zeile getrennt werden sollen für eine bessere Übersicht 
und um die Risiken leichter Zuordnen zu können.
- Die Tools haben die Zuweisung der Risikoklasse zunächst mit den Werten niedrig, mittel und hoch erstellt. Daher wurde der 
Prompt um die Anweisung, dass zwischen tolerierbar, unerwünscht, kritisch und katastrophal gewählt werden soll ergänzt.
Insgesamt war die Zuordnung jedoch ohne richtiges Konzept und schwerwiegende Risiken eine geringe Risikoklasse als eher unwichtigere
Risiken. Daher wurde die definition, wann welche Risikoklasse zuwählen ist, ebenfalls ergänzt.
- Damit Schaden und die Risikoklassen richtig gewählt werden und die Tools vergleichbar sind, wurde das Projektvolumen auf 500000€
festgelegt. Ansonsten wählen die Tools ein eigenes, was zu ungenauigkeiten führt.
- Die Erstellung der Risiken war kein Problem, eher bei der Formatierung der Tabelle und der Zuweisung der restlichen Attirbute haben
[[[[[[[[[Die Tools???????]]]]]]]]] schwierigkeiten.

- Gemini legt den Autor, auf den am ehsten Verantwortlichen für das Risiko, fest. Also Teamrisiken werden von Frau Schmidt definiert,
während zum Beispiel Anforderungsrisiken vom Anforderungsanalysten (Herr Schneider) und Kundenrisiken vom Kunden (Herr Müller) geschrieben
werden. Außerdem berechnet Gemini das Risikomaß nicht richtig. Das Komma der Werte ist um eine Stelle zu weit links. Es muss alles *10 
gerechnet werden. Die ID setzt Gemini einfach auf 1, 2, 3, ... was grundsätzlich aber nicht schlimm ist, da kein großer Mehraufwand, dies 
selbst zu ändern. Gemini kann nicht gesamte Tabelle aufeinmal Generieren. Es wird einfach mitten drinn aufgehört. Bei nachfrage ob er nur 
noch Produkt-, Management- und Planungsrisiken erstellen kann, macht er dies, fügt aber keinen Schaden und Maß mehr ein. Wenn ich ihm sag, 
das er den Schaden und Maß noch festlegen soll, sagt er nur, dass er das ohne detaillierte Informationen über das Projekt und die potenziellen
Risiken nicht kann.

- Le Chat hatte wieder das Problem mit dem Datum, dass dieses auf den 01.04.2023 gesetzt wurde. Manchmal war außerdem das Format der Tabelle 
nicht richtig und die Risiken wurde manchmal den Oberbegriffen nicht zugeordnet sondern zufällig erstellt. Musste schritt für schritt beschreiben 
was Le Chat ändern soll. Auch Le Chat bestimmt die Risikoklasse falsch. Auch nach Hinweis, dass diese nicht stimmen sind diese noch falsch 
und es wird der Schaden bei den Risiken geändert. Auch hier ist der Autor eher der Verantwortliche.


- Prompt für Maßnahmen ähnlich, dass jede Spalte der Tabelle einmal beschrieben wird und anschließend gesagt wird, dass die ID des dazugehörigen
Risikos übernommen werden soll.

- Gemini erstellt für die Begründung, warum ein Risiko als nicht mehr relevant eingestuft wird eine eigene Spalte. Ist aber nicht so schlimm, da 
zu beginn eh leer. Formatierung schlecht, da bei jeder 2. Zeile alle Daten ab Spalte "Begründung" um eine Stelle nach links verschoben wird. Neu 
generierung und auch Hinweis darauf ändert nichts. Trigger auch nicht gefüllt. Außerdem ist der Typ immer abwechselnd "Verhindern" und "Lindern".
Auch hier ist Teilweise das Restmaß falsch berechnet. Hört bei Risiko 4.2 auf mit der generierung der Maßnahmen. Maßnahmen passen aber zu Risiken.

- Le Chat erstellt die Maßnahmen nicht immer zu den Risiken in der Risikotabelle, sonder zu den Oberbegriffen Team-, Technische-, Methodische-, 
Kunden-, Fachliche-, Produkt-, Management- und Planungsrisiken. Maßnahme 1-4 scheinen sehr allgemein formuliert, Maßnahmen 5-8 würden
zu den 1. Risiken des jeweiligen Oberbegriffes passen. Auch hier scheinen die Attribute sehr repetitiv. Wahrscheinlichkeit immer bei 5\%,
die Risikoklasse außer bei der ersten Maßnahme immer Unerwünscht und der Restschaden auch meistens bei den 2 Maßnahmen für ein Risiko gleich.
