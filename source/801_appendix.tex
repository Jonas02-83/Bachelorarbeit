
\chapter{Anhang} \label{Anhang}

\section*{Kundengespräch}

\begin{prompt}[H]
    \begin{tcolorbox}[colback=gray!20, colframe=gray!20, boxrule=0pt, sharp corners] 
        **Kunde (Herr Müller):** Guten Tag, ich bin Herr Müller. Wie geht es ihnen?

        **Entwickler (Frau Schmidt):** Hallo, ich bin Frau Schmidt. Mir geht es gut. Wie geht es ihnen?
        
        **Kunde (Herr Müller):** Mir geht es auch gut. Ich möchte ein digitales "Mensch ärgere dich nicht"-Spiel entwickeln lassen und brauche Ihre Expertise dafür.
        
        **Entwickler (Frau Schmidt):** Es freut mich, dass Sie sich für unsere Dienstleistungen interessieren. Können Sie mir bitte mehr über Ihr Projekt erzählen? Welche Funktionen und Eigenschaften sollte das Spiel haben?
        
        **Kunde:** Natürlich. Zunächst möchte ich, dass das Spiel als JavaFX Anwendung auf dem Computer läuft. Die Anwendung soll eine Server-Client-Applikation, die mit Hilfe von Java realisiert wird, sein. Server und Client sollen über RMI kommunizieren. Es sollte das klassische "Mensch ärgere dich nicht"-Spielbrett und die entsprechenden Regeln enthalten. Außerdem soll es auch möglich sein, mehrere Spiele gleichzeitig zu spielen.
        
        **Entwickler:** Verstehe. Sollen die Spieler gegen den Computer, gegen andere Spieler online oder beides spielen können?
        
        **Kunde:** Beides wäre ideal. Es sollte einen Einzelspielermodus gegen KI-Gegner sowie einen Mehrspielermodus geben, in dem man online gegen Freunde oder zufällige Gegner spielen kann.
        
        **Entwickler:** Gut, dann werden wir einen Multiplayermodus integrieren. Wie viele Spieler sollen maximal teilnehmen können?
        
        **Kunde:** Wie im klassischen Spiel sollten es bis zu vier Spieler sein.
        
        **Entwickler:** Perfekt. Haben Sie Vorstellungen bezüglich des Designs? Soll das Spiel ein klassisches, traditionelles Aussehen haben oder möchten Sie ein modernes Design?
        
        **Kunde:** Ein klassisches Design mit einer ansprechenden, modernen Benutzeroberfläche wäre gut. Es sollte an das traditionelle Spiel erinnern, aber gleichzeitig frisch und ansprechend wirken.
        
        **Entwickler:** Verstanden. Gibt es besondere Funktionen oder Features, die Ihnen wichtig sind? Zum Beispiel Chat-Funktionalitäten, Punktesysteme oder spezielle Animationen?
        
        **Kunde:** Ein Chat während des Spiels wäre nett. Ein Punktesystem könnte die Motivation erhöhen, also wäre das auch eine gute Idee. Animationen sollten einfach, aber ansprechend sein - nichts Übertriebenes.
        
        **Entwickler:** Das klingt alles machbar. Wie sieht es mit dem Zeitrahmen aus? Haben Sie eine bestimmte Deadline, bis wann das Spiel fertig sein soll?
        
        **Kunde:** Ich denke, ein Zeitraum von sechs Monaten sollte machbar sein. Glauben Sie, dass das realistisch ist? Darf ich außerdem Fragen wie ihr Team aufgestellt ist?
        
        **Entwickler:** Ja, sechs Monate sind ein realistischer Zeitraum für dieses Projekt, vorausgesetzt, wir haben regelmäßige Feedback-Schleifen und Sie können uns zeitnah Ihre Anforderungen und Rückmeldungen geben. Wir sind ein Team aus 4 Personen. Ich bin die Projektleiterin, der Herr Schneider ist unser Anforderungsanalyst und fachlicher Chefdesigner, die Frau Fischer ist unsere Systemarchitektin und der Herr Becker ist der Testmanager.
        
        **Kunde:** Das klingt gut. Was wären die nächsten Schritte?
        \vfill
    \end{tcolorbox}
    \caption{Kundengespräch}
    \label{Kundengespräch1}
\end{prompt}

\begin{prompt}[H]
    \begin{tcolorbox}[colback=gray!20, colframe=gray!20, boxrule=0pt, sharp corners] 
        **Entwickler:** Wir würden zunächst einen detaillierten Projektplan erstellen, der alle Funktionen und Meilensteine festlegt. Danach beginnen wir mit der Entwicklung des Prototyps. Während der gesamten Entwicklung würden wir regelmäßige Meetings abhalten, um den Fortschritt zu besprechen und eventuelle Anpassungen vorzunehmen.
        
        **Kunde:** Das klingt nach einem guten Plan. Wie sind sie eigentlich dazu gekommen, Entwickler zu werden?
        
        **Entwickler:** Ich habe mich schon als Kind für Computerspiele interessiert und auch wie diese im Hintergrund funktionieren. Daher habe ich recht früh angefangen das Programmieren zu lernen. Das hat mir immer viel Spaß gemacht und so habe ich nach der Schule eine Ausbildung in der Branche angefangen.
        
        **Kunde:** Ah ich verstehe. Sie sind jetzt mittlerweile Projektleiterin oder?
        
        **Entwickler:** Ja, genau.
        
        **Kunde:** Fehlt ihnen die Arbeit als Programmiererin? Als Projektleitern machen Sie ja wahrscheinlich nicht mehr ganz so viel in dem Bereich oder?
        
        **Entwickler:** Ne, das stimmt. Aber man ist ja trotzdem noch damit in Berührung. Wenn mal irgendwo Schwierigkeiten bei der Implementierung auftreten, kann ich auch helfen diese zu lösen. Ein wenig fehlen tut es mir schon, aber meine jetzige Tätigkeit als Projektleiterin macht mir auch sehr viel Spaß.
        
        **Kunde:** Alles klar. Ich freue mich darauf, mit Ihnen zusammenzuarbeiten.
        
        **Entwickler:** Vielen Dank, Herr Müller. Wir freuen uns auch auf die Zusammenarbeit. Wir werden Ihnen in Kürze einen detaillierten Projektplan und ein Angebot zukommen lassen.
        
        **Kunde:** Perfekt, ich warte auf Ihre Nachricht. Einen schönen Tag noch!
        
        **Entwickler:** Ihnen auch, Herr Müller. Bis bald!
        \vfill
    \end{tcolorbox}
    \caption{Kundengespräch}
    \label{Kundengespräch2}
\end{prompt}


\section*{Tabelle der auf dem USB-Stick enthaltenen Dateien} \label{BeigelegteDateien}

\begin{longtable}{|p{5cm}|p{10cm}|}
    \hline
    \textbf{Datei} & \textbf{Beschreibung} \\
    \hline
    \endfirsthead

    01\_Besprechungsprotokolle.pdf & Enthält die erstellten Besprechungsprotokolle der drei Tools. \\
    \hline
    02\_Projekthandbuch.pdf & Enthält die erstellten Abschnitte vom Projekthandbuch der drei Tools. \\
    \hline
    03\_Gemini\_Risikoliste.xlsx & Enthält die erstellte Risiko- und Maßnahmentabelle von Gemini. \\
    \hline
    03\_GPT\_Risikoliste.xlsx & Enthält die erstellte Risiko- und Maßnahmentabelle von ChatGPT. \\
    \hline
    03\_LeChat\_Risikoliste.xlsx & Enthält die erstellte Risiko- und Maßnahmentabelle von Le Chat. \\
    \hline
    04\_Anforderungsspezifikation.pdf & Enthält die erstellten Dokumente der Anforderungsspezifikation der drei Tools. \\
    \hline
    05\_Architekturdokument.pdf & Enthält die erstellten Dokumente des Architekturdokumentes der drei Tools. \\
    \hline

\end{longtable}

\cleardoublepage

%% Abkürzungsverzeichnis

\printglossary[type=\acronymtype]
\printglossary

\listoffigures
\cleardoublepage 

%% Tabellenverzeichnis
\listoftables
\cleardoublepage 

%% Quellcodeverzeichnis
\renewcommand\lstlistlistingname{Quellcodeverzeichnis} 
\lstlistoflistings 
\renewcommand*\lstlistingname{Quellcode}
\cleardoublepage 

%% Stichwortverzeichnis, Index
\renewcommand{\indexname}{Stichwortverzeichnis}
\printindex
\cleardoublepage 

\printbibliography
\cleardoublepage 